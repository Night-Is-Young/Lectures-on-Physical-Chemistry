\documentclass{ctexart}
\usepackage{PhysicalChemistryNote}

\begin{document}\pagestyle{plain}
\noindent\tbf{\LARGE 0A 函数,极限与导数}\vspace{15pt}\\
\indent 客观世界中的许多变量都存在一定的关系,这就是我们需要定义函数来描述这一关系的原因.%
特别地,研究这些变量之间的变化关系——我们称之为“变化率”,不仅是一个自然的想法,也在之后有着举足轻重的作用,%
而严谨地定义这一概念就需要用到极限的概念.\vspace{12pt}\\
\Section{0A.1 函数}
\indent 想必你已经与函数这一概念打过不少交道了.函数最初是需要刻画一个变量如何依赖另一个变量而被发明的.%
例如,行星的位置可以视为对时间的函数.在此,我们再一次地给出函数的定义.
\begin{definition}[0A.1.1 函数]
    设$x$与$y$两个变量分别在集合$X$与$Y$中取值.如果有一种规则$f$,使得对于每个$x\in X$都有唯一确定的$y\in Y$与之对应,%
    那么我们称$f$是一个\tbf{函数}(有时也称为\tbf{映射}),记作$f:X\to Y$,并称$X$为$f$的\tbf{定义域},%
    所有$x$能对应到的$y$构成的集合称为$f$的\tbf{值域}.\\
    通常,上面的关系也可以写作$y=f(x)$,有时也直接写作$f(x)$.
\end{definition}
一个典型的函数是$f(x)=x^2(x\in\R)$,它把任何一个实数对应至自己的平方,其值域即为正实数集$R^\ast$.%
另一个稍微抽象一些(在数学中也很重要)的函数是Dirichlet函数,它的定义为
\[f(x)=\left\{\begin{array}{l}
    1,x\in\mathbb{Q}\\0,x\notin\mathbb{Q}
\end{array}\right.\]
其中$\mathbb{Q}$为有理数集.这样,它把所有有理数对应至$1$,而把所有无理数对应至$0$,其值域为集合$\left\{0,1\right\}$.\\
\indent 如果这样的对应关系$f$能用关于$x$的代数式表示出来,那么这个表达式就是$f$的解析式,例如上面的$f(x)=x^2$.\\
\indent 在我们的学习中,大部分的物理量之间都有良好的函数关系,例如大部分时候它们的定义域都是一整个区间而非分散的点集,并且如果把它的图像看起来总是一条光滑的曲线%
\footnote{我们将在后面详细地讨论将光滑这一性质是如何被严格定义的.}.\vspace{12pt}\\
\Section{0A.2 极限}
\indent 我们先来考虑一个看起来有些古怪的函数:$f(x)=x+1,x\neq0$.它的图像在大部分时候都与$f(x)=x+1$表现得别无二致,%
除了在$x=0$处:它没有被定义,因此在图像上对应一个空心点.\\
\indent 然而我们仍然希望获知在$x=0$附近时$f(x)$的性质.由于$x=0$在$f(x)$的定义域之外,%
我们可以转而考虑$x$十分接近$0$时$f(x)$的值,这样就有
\[f(0.01)=1.01\ \ \ \ \ f(0.0001)=1.0001\ \ \ \ \ f(0.000001)=1.000001\cdots\]
\[f(-0.01)=0.99\ \ \ \ \ f(-0.0001)=0.9999\ \ \ \ \ f(-0.000001)=0.999999\cdots\]
看起来比较明显的一点当$x$与$0$的差距越来越小时,$f(x)$与$1$的差距也越来越小.似乎可以这么说:%
当$x$“趋近”于$0$(但不等于$0$)时,$f(x)$“趋近”于$1$(但不等于$1$).\\
\indent 尽管直觉上确实如此,但我们希望用严谨的数学语言规范上面的定义.为此,我们考虑$1$附近的一个区间$\left(1-\ep,1+\ep\right)$,%
其中$\ep$是一个任意给定的正数.可以看出,无论我们如何选取$\ep$,都能找到$x=0$附近的区间$\left(-\delta,\delta\right)$(其中$0<\delta<\ep$),%
在这个区间里总有$1-\ep<f(x)<1+\ep$.%
由于$\ep$是任意选取的,并且总是有区间$\left(-\delta,\delta\right)$满足$\left|f(x)-1\right|<\ep$,因此无论要求$f(x)$多么接近$1$,%
总存在这样的区间满足这一要求.因此我们说$f(x)$可以无限地接近$1$,这就是\tbf{函数极限}的定义.
\begin{definition}[0A.2.1 函数极限]
    设函数$f(x)$在一点$x_0$的去心邻域$\left(x_0-a,x_0\right)\cup\left(x_0,x_0+a\right)$上有定义.%
    如果存在$l\in\R$,使得对任意$\ep>0$,都存在$\delta>0$,使得对任意满足$\left|x-x_0\right|<\delta$的$x$都有
    \[\left|f(x)-l\right|<\ep\]
    则称$x$趋于$x_0$时$f(x)$以$l$为极限,记作
    \[\lim_{x\to x_0}f(x)=l\]

\end{definition}
有时,对于左边和右边两种趋近$x_0$的方式能得到不同的结果,例如对于函数
\[f(x)=\left\{\begin{array}{l}
    x+1,x>0\\x-1,x<0
\end{array}\right.\]
从$x=0$的左边接近,所得的极限值为$-1$,而从$x=0$右边接近所得的极限值为$1$.%
因此,为了完善我们的理论,需要定义单侧极限.
\begin{definition}[0A.2.2 单侧极限]
    设函数$f(x)$在区间$\left(x_0,a\right)$上有定义.%
    如果存在$l\in\R$,使得对任意$\ep>0$,都存在$\delta>0$,使得对任意$x\in\left(x_0,x_0+\delta\right)$都有
    \[\left|f(x)-l\right|<\ep\]
    则称$x$趋于$x_0^+$时$f(x)$以$l$为右极限,记作
    \[\lim_{x\to x_0^+}f(x)=l\]
    对于左极限,只需设$f(x)$在$\left(a,x_0\right)$上有定义,并且把上述定义中的区间$x\in\left(x_0,x_0+\delta\right)$换成$\left(x\in\left(x_0-\delta,x_0\right)\right)$
\end{definition}
不难发现,只有两边的单侧极限都存在且相等时,\tbf{0A.1.1}中的极限(通常称为双侧极限)才能存在.\\
\indent 函数在某一处的极限与此处的值并不一定相等.例如,对于函数
\[f(x)=\left\{\begin{array}{l}
    x+1,x\neq0\\0,x=0
\end{array}\right.\]
根据前面的讨论,我们有
\[\lim_{x\to0}f(x)=1\]
这和$f(x)$在$0$处的值(即$f(0)=0$)并不相同.在图像上,这表现为一条直线上出现了一个间断点.%
反之,如果$f(0)=1$,那么图像就将变为一条完整的,连续的直线.这就是连续函数的定义.
\begin{definition}[0A.2.3 连续函数]
    设函数$f(x)$在$(a,b)$有定义.对于$x_0\in(a,b)$,如果
    \[\lim_{x\to x_0}=f\left(x_0\right)\]
    就称$f(x)$在$x_0$处\tbf{连续}.如果$f(x)$在定义域中的任意一点都连续,就称$f(x)$为\tbf{连续函数}.
\end{definition}
我们在今后研究的各种函数几乎都是连续函数.我们再也不会遇到前面例子中的奇怪的函数,这一点你可以放心.\vspace{12pt}\\
\Section{0A.3 导数}
\Part{平均速率和瞬时速率}
\indent 我们总是很关心交通工具的速率的.一般而言,物体的速率$v$可以通过以下的式子计算
\[v=\dfrac{s}{t}\]
其中$s$是物体在时间$t$内运动的距离.例如,一辆行驶在直道上的汽车在一小时内行驶了$100\text{ km}$,%
那么它的速率即为$100\text{ km/h}$.这一速率并不能反应司机在这段时间内踩了油门或刹车的具体情况,%
它只能反映车辆在这一个小时内平均的前进情况,因此事实上上面的式子只能描述物体的\tbf{平均速率}.\\
\indent 然而,你会注意到仪表盘上显示的速率却总是随着汽车前进的情况发生变化.%
它事实上反映了某一时刻时汽车的速率,即\tbf{瞬时速率}.我们先暂时不考虑仪表盘得出瞬时速率的原理,%
而是把注意力放在我们(作为人类)得出物体在某一时刻的瞬时速率的方法.
\begin{derivation}
    我们考虑物体在某时刻$t$的瞬时速率,并测量从时间$t$开始时一段时间内物体的平均速率.\\
    一种简单的想法是把测量平均速率时的时间放的尽可能短.例如,把上面的例子中的一小时改成%
    $1\text{ s}$,$0.01\text{ s}$乃至$0.0001\text{ s}$.汽车在如此短的时间内运动状态的变化%
    显然会比一小时内小得多,因此在更短的时间里测得的平均速率应当与它的瞬时速率更为接近.
\end{derivation}
我们刚才正好说到了一种准确地衡量“尽可能接近”的方法,即函数极限.
\end{document}