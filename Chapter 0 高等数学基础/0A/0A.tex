\documentclass{ctexart}
\usepackage{PhysicalChemistryNote}

\begin{document}\pagestyle{plain}
\noindent\tbf{\LARGE 0A 导数与微分}\vspace{15pt}\\
\indent 发展微积分的最初灵感之一来自试图去理解运动物体的速度,距离和时间的关系.%
人们经过数百年的探索建立了微积分这一学科,而这也是我们学习物理化学所必须掌握的数学基础知识.\vspace{12pt}\\
\Section{0A.1 导数}
\Part{导数的简单定义}
\indent 出于简单考虑,我们不准备介绍极限的定义和严格定义导数的方法\footnote{尽管这一概念是高等数学的基础.}.%
或者更确切地说,这一章节的内容都是为了让你对微积分有一个简单的,偏向直观的了解,而非与那些性质古怪的函数打交道.
\begin{hint}
    本书所考虑的所有函数几乎都是光滑的.这在图像上表现为一条“看起来”光滑的曲线,没有间断的地方,没有粗糙的折点.%
    一个函数光滑意味着它具有良好的性质,我们可以不加考虑地对它进行求导,无需担心可能出现的差错或意外.
\end{hint}
正如前言中所说,人们是从物体的运动状态开始研究微积分的.在初中时,你应当学过简单的运动学知识:在一段时间$t$内,物体运动的距离\footnote{准确而言应当是位移.此处我们假定物体做直线运动,因而位移就与距离的数值相同.}%
$x$与$t$之比就是它在这段时间内的平均速度$v$,即
\[v=\dfrac xt\]
我们让这段时间从某一时间点$t_0$开始,并结束于$t_0+\Delta t$,在这段时间内物体运动的距离为$\Delta x$.于是这段时间内的平均速度为
\[\bar{v}=\dfrac{\Delta x}{\Delta t}\]
现在让$\Delta t$越来越小.在此过程中,物体的平均速度$\bar{v}$应当\tbf{趋近}于某一定值,这就是某一时间$t_0$时该物体的瞬时速度$v_0$,可以记为
\[v_0=\lim_{\Delta t\to 0}\dfrac{\Delta x}{\Delta t}\]
这就是导数的定义.上面的$\lim$代表limit,意为“极限”.由于本书并不打算涉及极限的定义,因此上述式子只需意会即可:%
某一时间$t_0$的瞬时速度$v_0$等于当$\Delta t$趋近于$0$时$\left(t_0,t_0+\Delta t\right)$时间段内的运动距离$\Delta x$与$\Delta t$之比值.\\
\indent 我们知道,速度即为运动距离对时间的变化率.上面的叙述就可以引出导数的(浅显而形象的)定义.
\begin{definition}[0A.1.1 导数]
    函数$y=f(x)$在$x=x_0$处的导数定义为
    \[f'\left(x_0\right)=\lim_{\Delta x\to 0}\dfrac{f\left(x+\Delta x\right)-f(x)}{\Delta x}\]
    在各点处的$f'\left(x_0\right)$亦可以构成一个新的函数,这一函数$f'(x)$记为$f(x)$的导函数.%
    今后,我们说某一函数的导数,一般就指其导函数.
\end{definition}
至此,基础的介绍就已经完结了\footnote{上述内容似有听君一席话如听一席话的感受,这实在是由于笔者不才而不能很好地介绍导数所致.}.%
如果你想对高等数学的内容有更加深入的了解,笔者建议你阅读普林斯顿微积分读本以获得对微积分的初步认识,然后再阅读各种高等数学书籍.\vspace{4pt}\\
\Part{常见函数的导数}
\indent 接下来,我们将从定义出发推导一些函数的导函数\footnote{这会不可避免地引入极限的四则运算,你只需意会即可.%
总之,尽管$\Delta x$在很多结果里都看似是$0$,但它总是不为$0$.我们需要思考的是当它接近$0$时整个式子所接近的值,%
这和直接把$\Delta x=0$代入是有区别的,毕竟分母不可能为$0$.},首先是简单的幂函数.
\begin{derivation}
    我们先从最简单的$f(x)=x$入手.我们有
    \[f'(x)=\lim_{\Delta x\to0}\dfrac{f(x+\Delta x)-f(x)}{\Delta x}
    =\lim_{\Delta x\to0}\dfrac{x+\Delta x-x}{\Delta x}=\lim_{\Delta x\to0}1=1\]
    因此$f(x)=x$的导函数处处为$1$.\\
    现在考虑$f(x)=x^2$.情况要稍微复杂一些.
    \[\begin{aligned}
        f'(x)
        &= \lim_{\Delta x\to0}\dfrac{f(x+\Delta x)-f(x)}{\Delta x}=\lim_{\Delta x\to0}\dfrac{(x+\Delta x)^2-x^2}{\Delta x} \\
        &= \lim_{\Delta x\to0}\dfrac{2x\Delta x-(\Delta x)^2}{\Delta x}=\lim_{\Delta x\to0}(2x+\Delta x)=2x
    \end{aligned}\]
    而$f(x)=\dfrac1x$也是相似的.
    \[\begin{aligned}
        f'(x)
        &= \lim_{\Delta x\to0}\dfrac{f(x+\Delta x)-f(x)}{\Delta x}=\lim_{\Delta x\to0}\dfrac{\frac{1}{x+\Delta x}-\frac1x}{\Delta x} \\
        &= \lim_{\Delta x\to0}\dfrac{1}{\Delta x}\cdot\dfrac{x-(x+\Delta x)}{x(x+\Delta x)} \\
        &= -\lim_{\Delta x\to0}\dfrac{1}{x^2+x\Delta x} \\
        &= -\dfrac1{x^2}
    \end{aligned}\]
    现在再来考虑$f(x)=\sqrt{x}$.我们有
    \[\begin{aligned}
        f'(x)
        &= \lim_{\Delta x\to0}\dfrac{f(x+\Delta x)-f(x)}{\Delta x}=\lim_{\Delta x\to0}\dfrac{\sqrt{x+\Delta x}-\sqrt{x}}{\Delta x} \\
        &= \lim_{\Delta x\to0}\dfrac{(x+\Delta x)-x}{\left(\sqrt{x+\Delta x}+\sqrt{x}\right)\Delta x} \\
        &= \lim_{\Delta x\to0}\dfrac{1}{\sqrt{x+\Delta x}+\sqrt{x}} \\
        &= \dfrac{1}{2\sqrt{x}}
    \end{aligned}\]
    我们可以归纳出以下结论:幂函数$f(x)=x^a$的导函数为$f'(x)=ax^{a-1}$.这一结论的严格证明过程请查阅资料.
\end{derivation}
\begin{theorem}[0A.1.2 幂函数的导数]
    幂函数$f(x)=x^a(a\in\R)$的导函数为$f'(x)=ax^{a-1}$.
\end{theorem}
接下来是常见的三角函数.在推到之前,我们需要知道一个重要极限.
\begin{theorem}[0A.1.3 重要极限I]
    我们有重要极限
    \[\lim_{x\to0}\dfrac{\sin x}{x}=1\]
    你可以自行查阅这一极限的证明方法.
\end{theorem}
\begin{derivation}
    先考虑$f(x)=\sin x$.我们有
    \[\begin{aligned}
        f'(x)
        &= \lim_{\Delta x\to0}\dfrac{f(x+\Delta x)-f(x)}{\Delta x}=\lim_{\Delta x\to0}\dfrac{\sin(x+\Delta x)-\sin{x}}{\Delta x} \\
        &= \lim_{\Delta x\to0}\dfrac{\sin x\cos\Delta x+\sin\Delta x\cos x-\sin x}{\Delta x} \\
        &= \lim_{\Delta x\to0}\dfrac{\sin x(1-\cos\Delta x)}{\Delta x}+\lim_{\Delta x\to0}\dfrac{\cos x\sin\Delta x}{\Delta x} \\
        &= 0+\cos x=\cos x
    \end{aligned}\]
    其中
\end{derivation}
\Part{导(函)数的运算法则}
\indent 导函数的运算法则是十分重要且基础的工具.
\begin{theorem}[0A.1.2]
    导函数的运算法则主要有以下几条.
    \begin{enumerate}[label=\tbf{\roman*.},topsep=0pt,parsep=0pt,itemsep=0pt,partopsep=0pt]
        \item \tbf{导数的加减法}\\如果$h(x)=f(x)\pm g(x)$,那么$h'(x)=f'(x)\pm g'(x)$.
        \item \tbf{导数的乘法}\\如果$h(x)=f(x)g(x)$,那么$h'(x)=f'(x)g(x)+f(x)g'(x)$.
        \item \tbf{导数的除法}\\如果$h(x)=f(x)$
        \item 如果$h(x)=f(g(x))$,那么$h'(x)$
    \end{enumerate}
\end{theorem}
\end{document}