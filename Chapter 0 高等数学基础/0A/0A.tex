\documentclass{ctexart}
\usepackage{PhysicalChemistryNote}

\begin{document}\pagestyle{plain}
\noindent\tbf{\LARGE 0A 极限与导数}\vspace{15pt}\\
\indent 极限论是研究微积分的基本工具.\vspace{12pt}\\
\Section{0A.1 函数与极限}
\Part{函数}
\indent 想必你已经与函数这一概念打过不少交道了.我们在此再一次地给出函数的定义,以开始我们对高等数学的学习.
\begin{definition}[0A.1.1 函数]
    设$x$与$y$两个变量分别在集合$X$与$Y$中取值.如果有一种规则$f$,使得对于每个$x\in X$都有唯一确定的$y\in Y$与之对应,%
    那么我们称$f$是一个\tbf{函数}(有时也称为\tbf{映射}),记作$f:X\to Y$,并称$X$为$f$的\tbf{定义域},%
    所有$x$能对应到的$y$构成的集合称为$f$的\tbf{值域}.
\end{definition}
如果这样的对应关系能用关于$x$的表达式表示出来,那么这个表达式就是$f$的解析式.%
我们可以把它写成$y=f(x)$的形式.\\
\indent 在我们的学习中,大部分的物理量之间都有良好的函数关系(至少大部分时候它们的定义域都是一个区间而非分散的点集,%
更多时候是正实数集$\R^*$).因此你不必为这个看起来比较严格而奇怪的定义伤脑筋,%
而大可以用你早先就知道的函数的定义.
\Part{极限}
\indent 微积分中的两个基本概念,即导数(微商)和定积分,都是建立在%
极限的概念的基础之上的.尽管我们在以后的学习中主要用到的是微积分的知识,而且%
并不需要像理论数学一样进行严格地证明,但是简单地学习极限论以了解微积分的基本思想却是必要的.
\end{document}