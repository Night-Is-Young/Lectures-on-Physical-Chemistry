\documentclass{ctexart}
\usepackage{PhysicalChemistryNote}

\begin{document}\pagestyle{plain}
\noindent\tbf{\LARGE 1B 气体分子动理论}\vspace{15pt}\\
%
\indent 据说,从前有个农民,他养了一些鸡,但是它们不下蛋.于是他去找物理学家求救.%
这个物理学家便做了些计算,然后说:"我有个办法,不过只适用于真空中的球形鸡."\\
\indent 虽然这个笑话讽刺了物理学家们做的研究似乎总是不太切实际(恕我直言,现代化学也有相当一部分是这样),但是所有的理论总是建立在一个又一个理想化的假设上的.%
当我们知道了这些假设和它们的推论,也就离描述真实的世界又近了一步.毕竟,真空中的球形鸡也是一种鸡.\vspace{12pt}\\
\Section{1B.1 气体分子动理论的基本公式}
\Part{气体分子动理论的基本方程}
\indent 我们从气体分子运动的微观模型开始本节.
\begin{theorem}[1B.1.1 气体分子运动的微观模型]
    (理想)气体分子运动的微观模型可表述为:
    \begin{enumerate}[label=\tbf{\arabic*.}]
        \item 气体是大量分子的集合体.相较于气体分子之间的距离和所占据的体积,%
            其本身的体积是很小而可以忽略不计的.因此,可以把气体分子当作质点处理.
        \item 气体分子总是做无规则的运动,均匀分布在容器中.
        \item 分子之间的碰撞或分子与容器壁的碰撞是完全弹性的.
    \end{enumerate}
\end{theorem}
应当思考的是,一定量的气体在一定体积下具有恒定的压强.%
从微观角度解释,这一压强来源于大量气体分子均匀地碰撞容器壁而产生.%
因此,压强和体积这两个宏观物理量应当与气体分子的微观性质(例如运动速度,质量,数目等等)相关联.%
下面我们从上述假设和一些基本的物理知识推导联系上述宏观量和微观量的\tbf{气体分子动理论的基本方程}.
\begin{derivation}
    为了简单考虑,我们不妨设容器为一个边长为$l$的正方体.一个质量为$m$的粒子$X_i$以$v_i$的速度沿$x$轴撞击某一壁.\\
    根据\tbf{1B.1.1.3}可知该分子与容器壁的碰撞是完全弹性的,于是该过程的动量变化为
    \[\Delta p=mv_{x,i}-\left(-mv_{x,i}\right)=2mv_{x,i}\]
    其中$v_{x,i}$为$X_i$沿$x$方向的速度分量.$X_i$每隔$\dfrac{2l}{v_{x,l}}$的时间便撞击一次该壁,于是此面受的力为
    \[F_{x,i}=\dfrac{\Delta p}{\Delta t}=\dfrac{2mv_{x,i}}{\frac{2l}{v_{x,i}}}=\dfrac{mv_{x,i}^2}{l}\]
    考虑容器内的所有粒子,假定它们质量相同,总数目为$N$,则有
    \[\sum F_x=\dfrac{\displaystyle \sum_{i=1}^{N}mv_{x,i}^2}{l}=\dfrac ml\sum_{i=1}^{N}v_{x,i}^2\]
    令$\displaystyle \overline{v_x^2}=\dfrac{1}{N}\sum_{i=1}^{N}v_{x,i}^2$,$F_x=\dfrac{mN\overline{v_x^2}}{l}$,其中$F_x$即为所有分子对$x$方向某侧器壁的作用力.同理可得
    \[F_y=\dfrac{mN\overline{v_y^2}}{l}\ \ \ \ \ F_z=\dfrac{mN\overline{v_z^2}}{l}\]
    又因为$v_{x,i}^2+v_{y,i}^2+v_{z,i}^2=v_i^2$,于是可以定义
    \[\overline{v^2}=\dfrac{1}{N}\sum_{i=1}^{N}v_i^2=\overline{v_x^2}+\overline{v_y^2}+\overline{v_z^2}\]
    根据\tbf{1B.1.1.2}和正方体的对称性可知,各个方向上器壁的受力应当相同,于是$F_x=F_y=F_z$,从而
    \[\overline{v_x^2}=\overline{v_y^2}=\overline{v_z^2}\ \ \ \ \ \overline{v^2}=3\overline{v_x^2}\]
    每个面受到的压强$p$为
    \[p=\dfrac{F}{S}=\dfrac{\frac{mN\overline{v_x^2}}{l}}{l^2}=\dfrac{mN\overline{v_x^2}}{l^3}=\dfrac{mN\overline{v^2}}{3V}\]
    移项即可得
    \[pV=\dfrac13mN\overline{v^2}\]
    如果令$u=\sqrt{\dfrac1N\displaystyle\sum_{i=1}^{N}v_i^2}$,称为\tbf{均方根速率},则有
    \[pV=\dfrac13mNu^2\]

\end{derivation}
我们就得到了联系宏观量和微观量的一个方程.
\begin{theorem}[1B.1.2 气体分子动理论的基本方程]
    设气体体积为$V$,压强为$p$,气体分子质量为$m$,数目为$N$,均方根速率为$u$,则有
    \[pV=\dfrac13mNu^2\]

\end{theorem}
\Part{压强和温度的统计学意义}
\indent 由\tbf{1B.1.2},我们可以得到压强$p$的统计学意义.在上述推导中,压强实际上就是大量分子撞击器壁产生的总结果.
\begin{theorem}[1B.1.3 压强的统计学意义]
    大量气体分子撞击器壁,总体来说器壁受到的作用力(压力)具有持续性,稳定性.在宏观上这就表现为压强.
\end{theorem}
那么温度究竟是什么呢?我们在\tbf{1A}中对温度做过一个粗浅的定义,用于衡量两个物体接触时热量的传递方向.%
对于两个温度不同的气体(不妨假定为A和B),假定它们的能量仅来源于其动能,那么交换能量的方式就是通过碰撞而互相传递能量.%
在宏观上就表现为平均动能\footnote{这里指的是平动能,即分子整体移动的动能,而有别于转动能.}.高的气体分子向平动能低的气体分子传递能量.%
因此,如果它们的温度不同,就应当具有平动能上的的区别.否则,如果平动能相同,那么宏观上两种气体没有能量传递,也就不应当具有温度的差别.\\
\indent 因此,可以认为平动能$\overline{E_t}=\dfrac12u^2$与温度具有正相关性,平动能越高则气体温度越高,向别的物体传热的能力就越强.%
一种直接而合理的想法是将平动能直接称为温度,不过为了和现行的温度的尺度相符,我们可以添加一个比例系数,即
\[\dfrac12mu^2=\dfrac{3}{2}k_\text BT\]
这里的$k_\text B$即Boltzmann常数,而$\dfrac32$则是方便计算而引入的系数(我们马上就会看到这样写的合理性).这就是基于分子动理论的温度的定义.
\begin{definition}[1B.1.4 理想气体温标]
    理想气体的温度$T$被定义为
    \[\dfrac12mu^2=\dfrac32k_\text BT\]
    其中$\dfrac12mu^2$是气体分子的平均平动能.
\end{definition}
基于这样的温度的定义,我们不难可以得到
\[pV=\dfrac13Nmu^2=Nk_\text BT=nRT\]
如此,你就从气体分子动理论得到了理想气体状态方程.基于这个式子可以得到
\[u=\sqrt{\dfrac{3k_\text BT}{m}}\]
这是一个比较重要的式子,在推导Maxwell速率分布时将要用到.
\begin{hint}
    我们将在\tbf{Chapter 3}中给出温度的另一种定义,即热力学温标,并且说明两种定义的等价性.
\end{hint}\vspace{8pt}
\Section{1B.2 分子运动的速率分布}
\Part{Maxwell速率分布的推导}
\indent 气体包含数量众多的分子.它们在容器内每时每刻都在互相碰撞,其速度都在发生变化.那么,是否有一种方法能总体地描述气体分子运动速率呢?%
我们将从简单的概率论出发推导分子运动的速率的分布情况.\\
\indent 我们首先需要知道一些概率论的知识.当随机变量是离散的时候,我们都知道它的的每一个取值都对应一个大于等于0的概率值.%
例如,均匀的六面骰子掷出1到6中任意一点的概率均为$\frac16$.\\
\indent 然而,当随机变量是连续的时候,我们不再能逐一例举每一个随机变量,而且很显然的一个结论是随机变量的每一个取值的概率都是0.%
例如,在区间$(0,1)$上任意取实数,取到某个特定的实数的概率都是$0$(你可以粗略地理解为$\frac{1}{\infty}$).%
这时,真正有意义的是随机变量落在某一个区间的概率,例如你所取的实数落在区间$\left(0,\frac12\right)$内的概率(直观的)为$\frac12$.
\begin{definition}[1B.2.1 连续随机变量的概率密度函数]
    对于随机变量$x$,我们假定一个概率函数$P(x)$,使得对于某个特定的取值$x_0$,随机变量落在一个无穷小区间$(x_0,x_0+\di x)$内的概率(这当然也是一个无穷小量)满足
    \[P\left(x_0<x<x_0+\di x\right)=\di P(x_0)\]
    如果要求$x$落在区间$(a,b)$内的概率,应当进行积分
    \[P(a<x<b)=\int_a^b\di P(x)\]
    将$\di P(x)$写成$f(x)\di x$,就有
    \[P(a<x<b)=\int_a^bf(x)\dx\]
    我们将这样的函数$f(x)$称作随机变量$x$的\tbf{概率密度函数}.\\
    对于某个特定的$x_0$,$f\left(x_0\right)\dx$就等于$x\in\left(x_0,x_0+\dx\right)$的概率.\\
    概率密度函数需要满足\tbf{归一化}条件.如果$x$的所有可能的取值范围为$(A,B)$,那么我们有
    \[\int_A^Bf(x)\dx=1\]
    这代表$x$落在$(A,B)$内的概率是$1$.
\end{definition}
类似地,分子速率也是一个连续随机函数.我们可以类似地得到分子速率分布函数的定义.
\begin{definition}[1B.2.2 分子速率分布函数]
    在一定温度$T$下,某个气体分子的运动速率$v$是一个连续随机变量.如果关于速率$v$的函数$f(v)$满足对任意的$0<v_a<v_b$都有
    \[P\left(v_a<v<v_b\right)=\int_{v_a}^{v_b}f(v)\di v\]\
    成立,则称$f(v)$是该温度下气体的\tbf{分子速率分布函数}.
\end{definition}
于是,只要我们知道了$f(v)$的具体形式,就能研究分子处于任意一个速率范围内的概率,进而获知气体分子的速率分布.下面我们着手推导$f(v)$.
\begin{derivation}
    设分子的速率为$v$,其速度在空间直角坐标系上可以分解为$v_x,v_y,v_z$.\\
    以$v_x,v_y,v_z$为轴建立一个三维空间.这样,每个分子都可以根据其在三个方向上的速度(含正负)对应到这空间中的一个点.\\
    先考虑$v_x$方向.假定分子总数为$N$,速率处于$\left(v_x,v_x+\di v_x\right)$上的分子数目为$\di N_{v_x}$.这样,分子的$x$方向上速率处于该区间内的概率为$\dfrac{\di N_{v_x}}{N}$.\\
    分子在$x$方向上的速率当然也是连续随机变量,设其概率密度函数为$f_x\left(v_x\right)$.根据\tbf{1B.2.1}可知
    \begin{equation}
        \dfrac{\di N_{v_x}}{N}=f_x\left(v_x\right)\di v_x
    \end{equation}
    类似地,在$y$方向和$z$方向上有
    \begin{equation}
        \dfrac{\di N_{v_y}}{N}=f_y\left(v_y\right)\di v_y\ \ \ \ \ \dfrac{\di N_{v_z}}{N}=f_z\left(v_z\right)\di v_z
    \end{equation}
    根据气体分子运动的特性,我们认为分子在总体上并没有倾向于向任何方向运动(即"没有择优方向"),即速度的分布是各向同性的.%
    于是,我们有理由认为各个方向上的速率分布函数$f_x,f_y,f_z$具有相同的形式,记为$f$.\\
    同样地,我们可以假设各个方向上的速度是独立的.直观地想,分子在某个方向上的速率应当不会影响它在与该方向垂直方向上的速率.\\
    几件独立事件同时发生的概率应当等于它们各自发生的概率之积,%
    于是,在三个方向上的运动速率分别处于$\left(v_x,v_x+\di v_x\right)$,$\left(v_y,v_y+\di v_y\right)$和$\left(v_z,v_z+\di v_z\right)$中的分子的数目$\di N_{v_x,v_y,v_z}$为
    \begin{equation}
        \di N_{v_x,v_y,v_z}=Nf\left(v_x\right)f\left(v_y\right)f\left(v_z\right)\di v_x\di v_y\di v_z
    \end{equation}
    这个区域的体积为$\di V=\di v_x\di v_y\di v_z$.既然我们知道这区域内的分子数目和体积,就可以定义一个密度函数$\rho\left(v_x,v_y,v_z\right)$(注意这里的$\rho$不是概率密度函数,$\dfrac{\rho}{N}$才是),使得
    \begin{equation}
        \rho\left(v_x,v_y,v_z\right)=\dfrac{\di N_{v_x,v_y,v_z}}{\di V}=Nf\left(v_x\right)f\left(v_y\right)f\left(v_z\right)
    \end{equation}
    对$\rho$做偏微分有
    \begin{equation}
        \begin{aligned}
            \di\rho
        &= \left(\dfrac{\p\rho}{\p v_x}\right)\di v_x+\left(\dfrac{\p\rho}{\p v_y}\right)\di v_y+\left(\dfrac{\p\rho}{\p v_z}\right)\di v_z \\
        &= Nf'\left(v_x\right)f\left(v_y\right)f\left(v_z\right)\di v_x+Nf\left(v_x\right)f'\left(v_y\right)f\left(v_z\right)\di v_y+Nf\left(v_x\right)f\left(v_y\right)f'\left(v_z\right)\di v_z
        \end{aligned}
    \end{equation}
    仍然由于速度具有各向同性,因此指定速率为$v$时,只要满足$v_x^2+v_y^2+v_z^2=v^2$,$\rho\left(v_x,v_y,v_z\right)$的取值就应相同.%
    于是在这个球壳上,有$\di\rho=0$,即$\dfrac{\di\rho}{\rho}=0$.于是
    \begin{equation}
        \left\{\begin{array}{l}
            \dfrac{f'\left(v_x\right)}{f\left(v_x\right)}\di v_x+\dfrac{f'\left(v_y\right)}{f\left(v_y\right)}\di v_y+\dfrac{f'\left(v_z\right)}{f\left(v_z\right)}\di v_z=0 \\
            v_x\di v_x+v_y\di v_y+v_z\di v_z=0
        \end{array}\right.
    \end{equation}
    后者是对球面方程微分得到的,决定了$\di v_x,\di v_y,\di v_z$之间满足的关系.前者是速率分布的各向同性决定的$\rho$在球面上的取值不变而得到的.%
    现在,我们要在这两个限制条件下求解$f$的表达式.考虑Lagrange乘数法,将第二个方程乘以参数$\lambda$后与第一个方程相加可得
    \begin{equation}
        \left(\dfrac{f'\left(v_x\right)}{f\left(v_x\right)}+\lambda v_x\right)\di v_x
        +\left(\dfrac{f'\left(v_y\right)}{f\left(v_y\right)}+\lambda v_y\right)\di v_y
        +\left(\dfrac{f'\left(v_z\right)}{f\left(v_z\right)}+\lambda v_z\right)\di v_z=0
    \end{equation}
    选定$\lambda$使得
    \begin{equation}
        \dfrac{f'\left(v_x\right)}{f\left(v_x\right)}+\lambda v_x=0
    \end{equation}
    则上式变为
    \begin{equation}
        \left(\dfrac{f'\left(v_y\right)}{f\left(v_y\right)}+\lambda v_y\right)\di v_y
        +\left(\dfrac{f'\left(v_z\right)}{f\left(v_z\right)}+\lambda v_z\right)\di v_z=0
    \end{equation}
    现在,$\di v_y$和$\di v_z$不再有限定关系(无论这两者如何变化,我们都可以根据(6)求出$\di v_x$后使得约束条件成立).令$\di v_y=0$可得
    \begin{equation}
        \dfrac{f'\left(v_z\right)}{f\left(v_z\right)}+\lambda v_z=0
    \end{equation}
    同理令$\di v_z=0$可得
    \begin{equation}
        \dfrac{f'\left(v_y\right)}{f\left(v_y\right)}+\lambda v_y=0
    \end{equation}
    于是我们得到了三个形式相同的微分方程.以(8)为例,变形可得
    \[\dfrac{\di f\left(v_x\right)}{f\left(v_x\right)}=-\lambda v_x\di v_x\]
    两边积分可得
    \[\ln f\left(v_x\right)=-\dfrac12\lambda v_x^2+\ln\alpha\]
    式中$\alpha$为积分常数.令$\beta^2=\dfrac12\lambda$,并取指数有
    \begin{equation}
        f\left(v_x\right)=\alpha\e^{-\beta^2v_x^2}
    \end{equation}
    同理
    \begin{equation}
        f\left(v_y\right)=\alpha\e^{-\beta^2v_y^2}
    \end{equation}
    \begin{equation}
        f\left(v_z\right)=\alpha\e^{-\beta^2v_z^2}
    \end{equation}
    将(12)(13)(14)代入(4)中可得
    \begin{equation}
        \rho\left(v_x,v_y,v_z\right)=N\alpha^3\exp\left(-\beta^2\left(v_x^2+v_y^2+v_z^2\right)\right)=N\alpha^3\exp\left(-\beta^2v^2\right)
    \end{equation}
    于是密度函数是以$v$为自变量的函数.\\
    考虑速度落在$(v,v+\di v)$区间上的分子,这是一个球壳层,其体积为$\di V=4\pi v^2\di v$.在这个球壳层内,分子分布的密度$\rho$近似相等,于是在这个球壳上分布的分子数目$\di N_v$为
    \begin{equation}
        \di N_v=\rho(v)\di V=N\alpha^3\exp\left(-\beta^2v^2\right)\cdot4\pi v^2\di v
    \end{equation}
    由于分子速率一定处于$(0,+\infty)$上,于是
    \begin{equation}
        \int_{0}^{+\infty}\di N_v=\int_{0}^{+\infty}N\alpha^3\exp\left(-\beta^2v^2\right)\cdot4\pi v^2\di v=N
    \end{equation}
    化简可得
    \[4\pi\alpha^3\int_0^{+\infty}v^2\exp\left(-\beta^2v^2\right)\di v=1\]
    查阅积分表可得
    \[\int_0^{+\infty}v^2\exp\left(-\beta^2v^2\right)\di v=\dfrac{\sqrt\pi}{4\beta^3}\]
    代入上式可得
    \begin{equation}
        \sqrt\pi\alpha=\beta
    \end{equation}
    又因为均方根速率
    \begin{equation}
        u=\sqrt{\int_{0}^{+\infty}v^2F(v)\di v}=\sqrt{\dfrac{3k_\text BT}{m}}
    \end{equation}
    于是
    \[\dfrac{3k_\text BT}{m}=\dfrac{4\beta^3}{\sqrt\pi}\int_0^{+\infty}v^4\exp\left(-\beta^2v^2\right)\di v\]
    查阅积分表可得
    \[\int_0^{+\infty}v^4\exp\left(-\beta^2v^2\right)\di v=\dfrac{3\sqrt\pi}{8\beta^5}\]
    于是
    \begin{equation}
        \dfrac{3k_\text BT}{m}=\dfrac{3}{2\beta^2}
    \end{equation}
    根据(18)与(20)可得
    \[\alpha=\sqrt{\dfrac{m}{2\pi k_\text BT}}\ \ \ \ \ \beta=\sqrt{\dfrac{m}{2k_\text BT}}\]
    代入(16)可得
    \begin{equation}
        \di N_v=\dfrac{4N}{\sqrt\pi}\left(\dfrac{m}{2k_\text BT}\right)^{\frac32}v^2\text{exp}\left(-\dfrac{mv^2}{2k_\text BT}\right)\di v
    \end{equation}
    根据$\di N_v$的定义,$\dfrac{\di N_v}{N}$表示气体速率在$(v,v+\di v)$内的概率,因而根据\tbf{1B.2.1}可知
    \begin{equation}
        \dfrac{\di N_v}{N}=f(v)\di v
    \end{equation}
    其中$f(v)$为气体的速率分布函数.将(21)代入(22)有
    \[f(v)=\dfrac{4}{\sqrt\pi}\left(\dfrac{m}{2k_\text BT}\right)^{\frac32}v^2\text{exp}\left(-\dfrac{mv^2}{2k_\text BT}\right)\]
    这就是Maxwell速率分布函数.
\end{derivation}\setcounter{equation}{0}
总结我们的推导过程,大致按照
\begin{enumerate}[label=\tbf{\arabic*.}]
    \item 写出各个方向上的速率分布函数$f\left(v_x\right)$,$f\left(v_y\right)$,$f\left(v_z\right)$.
    \item 用运动的各向同性写出在速率空间内的密度函数$\rho\left(v_x,v_y,v_z\right)$.
    \item 考虑到$\rho\left(v_x,v_y,v_z\right)$在球壳上的取值不变,写出关于$f$的微分方程.
    \item 求解$f$和$\rho$的含待定参量$\alpha,\beta$的形式.
    \item 根据概率密度函数的归一化性质和分子的均方根速率求得待定的参量$\alpha,\beta$.
    \item 得出速率分布函数.
\end{enumerate}
的过程进行推导.这样就得到了Maxwell速率分布.
\begin{theorem}[1B.2.2 Maxwell速率分布]
    气体分子运动的速率函数$f(v)$具有如下形式
    \[f(v)=\dfrac{4}{\sqrt\pi}\left(\dfrac{m}{2k_\text BT}\right)^{\frac32}v^2\text{exp}\left(-\dfrac{mv^2}{2k_\text BT}\right)\]

\end{theorem}
下面绘制了不同温度和分子质量下的Maxwell速率分布的图像,以供你理解这一分布的性质.%
可以看到,分子质量越大,温度越低,速率分布的越集中.
\begin{figure}[H]
    \centering\documentclass{standalone}
\usepackage{PhysicalChemistryNote}
\begin{document}
\begin{tikzpicture}
    \begin{axis}[
        width = 12cm,
        height = 8cm,
        legend pos = north east,
        xlabel = {速率 $v/\left(\mathrm{m}\cdot\mathrm{s}^{-1}\right)$},
        ylabel = {概率密度 $f(v)/10^{-3}$},
        axis lines = left,
        ymax = 3.5,
        domain = 0:3000,
        samples = 400
    ]
    \addplot [thick, blue] {(4 / sqrt(pi)) *0.0000133* x^2 * exp(-0.0000056*x^2)};
    \addplot [thick, red] {(4 / sqrt(pi)) *0.000069* x^2 * exp(-0.0000168*x^2)};
    \addplot [thick, yellow] {(4 / sqrt(pi)) *0.000000254* x^2 * exp(-0.0000004*x^2)};
    \addplot [thick, cyan] {(4 / sqrt(pi)) *0.00000132* x^2 * exp(-0.0000012*x^2)};
    \legend {$\text{N}_2\left(300\text{K}\right)$,
        $\text{N}_2\left(100\text{K}\right)$,
        $\text{H}_2\left(300\text{K}\right)$,
        $\text{H}_2\left(100\text{K}\right)$,}
    \end{axis}
\end{tikzpicture}
\end{document}
\end{figure}
\Part{描述分子速率分布的三个特征量}
\indent 不难发现,上图的每条曲线都有一个最高点,表示具有该速度的分子所占比例最大.于是可以做如下定义.
\begin{definition}[1B.2.3 最概然速率]
    分子的\tbf{最概然速率}$v_m$是气体分子在一定条件下最有可能具有的速率,对应速率分布函数$f(v)$取到最大值时$v$的取值.
\end{definition}
我们可以尝试对$v_m$的具体值进行推导.
\begin{derivation}
    $f(v)$图像的最高点满足
    \[\dfrac{\di f(v)}{\di v}=0\]
    于是
    \[\dfrac{\di}{\di v}\left(v^2\exp\left(-\dfrac{mv^2}{2k_{\text BT}}\right)\right)=0\]
    化简可得
    \[\left(2v-\dfrac{mv^3}{k_{\text BT}}\right)\exp\left(-\dfrac{mv^2}{2k_{\text BT}}\right)=0\]
    由此解得
    \[v=\sqrt{\dfrac{2k_\text BT}{m}}=\sqrt{\dfrac{2RT}{M}}\]

\end{derivation}
由此可见,分子的最概然速率与温度的平方根成正比,与摩尔质量的平方根成反比,这也符合我们对图像的观察.\\
\indent 除了最概然速率之外,还有一个值得考量的统计量,即所有分子的速率的数学平均值.
\begin{definition}[1B.2.4 数学平均速率]
    分子的\tbf{数学平均速率}$v_a$是所有分子速率的数学平均值.
\end{definition}
同样地,我们也可以计算$v_a$的具体值.
\begin{derivation}
    考虑所有$N$个分子,处于速率$v$的分子有$\di N_v$个,将它们求和(对无穷小量的求和实际上就是积分运算)有
    \[v_a=\dfrac{\displaystyle\sum v\cdot\di N_v}{N}=\dfrac1N\int_{0}^{+\infty}v\di N_v
    =\int_0^{+\infty}\dfrac{4}{\sqrt\pi}\left(\dfrac{m}{2k_\text BT}\right)^{\frac32}v^3\text{exp}\left(-\dfrac{mv^2}{2k_\text BT}\right)\di v\]
    做代换$x=\dfrac{mv^2}{2k_\text BT}$,则上式可以写作
    \[v_a=\sqrt{\dfrac{8k_\text BT}{\pi m}}\int_{0}^{+\infty}x\e^{-x}\di x\]
    查阅积分表可得
    \[\int_{0}^{+\infty}x\e^{-x}\di x=1\]
    于是
    \[v_a=\sqrt{\dfrac{8k_\text BT}{\pi m}}\]

\end{derivation}
至此,我们得到了关于分子运动速率的三个统计量:均方根速率$u$,最概然速率$v_m$和数学平均速率$v_a$.它们之间的比例关系如下
\[v_m:v_a:u=\sqrt2:\sqrt{\dfrac8\pi}:\sqrt3\]
\Section{1B.3 分子平动能的分布}
\indent 从Maxwell分布很容易就能推出分子能量(这里指的就是平动能)的分布.
\begin{derivation}
    各分子的平动能为$E=\dfrac12mv^2$,于是
    \[\di E=mv\di v\]
    代入\tbf{1B.2.2}有
    \[f(E)=\dfrac{2}{\sqrt\pi}\left(\dfrac{1}{k_\text BT}\right)^{\frac32}\sqrt{E}\exp\left(-\dfrac{E}{k_\text BT}\right)\]
    其实就是把$f(v)$改写成$E$的函数以得到关于能量的分布.
\end{derivation}
\begin{theorem}[1B.3.1 三维空间中的能量分布函数]
    分子平动能在三维空间中的\tbf{能量分布函数}$f(E)$为
    \[f(E)=\dfrac{2}{\sqrt\pi}\left(\dfrac{1}{k_\text BT}\right)^{\frac32}\sqrt{E}\exp\left(-\dfrac{E}{k_\text BT}\right)\]
    对于任意的$0<E_1<E_2$,分子能量$E$满足$E_1<E<E_2$的概率为
    \[P\left(E_1<E<E_2\right)=\int_{E_1}^{E_2}f(E)\di E\]

\end{theorem}
我们还可以考虑所有分子中能量大于$E_0$的分子的数目$N_{E_0\to\infty}$(这在化学反应动力学的研究中非常重要).根据\tbf{1B.3.1}就有
\[\dfrac{N_{E_0\to\infty}}{N}=P\left(E_0<E\right)=\int_{E_0}^{+\infty}f(E)\di E\]
用分部积分法对上面的积分做展开,可得
\[\dfrac{N_{E_0\to\infty}}{N}=\dfrac{2}{\sqrt\pi}\sqrt{\dfrac{E_0}{k_\text BT}}\exp\left(\dfrac{E_0}{k_\text BT}\right)\left[1+\left(\dfrac{k_\text BT}{2E_0}\right)-\left(\dfrac{k_\text BT}{2E_0}\right)^2+3\left(\dfrac{k_\text BT}{2E_0}\right)^3-\cdots\right]\]
一般情况下$E_0\gg k_\text BT$,因此求和项可以只保留第一项.于是我们有
\begin{theorem}[1B.3.2 大于某个能量值的分子数目]
    气体分子能量大于某个定值$E_0$的概率为
    \[P_{E_0\to\infty}=\dfrac{N_{E_0\to\infty}}{N}=\dfrac{2}{\sqrt\pi}\sqrt{\dfrac{E_0}{k_\text BT}}\exp\left(\dfrac{E_0}{k_\text BT}\right)\]

\end{theorem}
通常在物理化学中,我们只需用到能量分布的近似公式,这是通过二维空间中的Maxwell速率分布推出的.%
在我们学习推导三维空间中的情况后,也许此时你可以先合上书,在草稿纸上自己演算一遍.
\begin{derivation}
    设分子速率为$v$,其速度在平面直角坐标系上可以分解为$v_x,v_y$.\\
    同样地,建立一个二维的速度空间将每个分子对应到其上的一点.\\
    考虑$x$方向和$y$方向的速率分布函数$f_x\left(v_x\right)$和$f_y\left(v_y\right)$.与三维空间相同地,我们有
    \[f\left(v_x\right)=\alpha\e^{-\beta^2v_x^2}\]
    \[f\left(v_y\right)=\alpha\e^{-\beta^2v_y^2}\]
    于是概率密度函数
    \[\rho\left(v_x,v_y\right)=Nf\left(v_x\right)f\left(v_y\right)
    =\dfrac{Nm}{2\pi k_\text B T}\exp\left(-\dfrac{v^2}{2k_\text BT}\right)\]
    做极坐标变换,即$v_x=v\cos\theta,v_y=v\sin\theta$,则$\di v_x\di v_y=v\di v\di\theta$.%
    考虑到所有速率为$v$的分子,运动方向介于$0$到$2\pi$之间,于是
    \[\di N_v=\int_{v_x^2+v_y^2=v^2}\rho\di v_x\di v_y=\int_{0}^{2\pi}\rho v\di v\di \theta=2\pi\rho\di v\]
    从而速率分布函数
    \[f(v)=\dfrac{\di N_v}{\di v}=2\pi\rho=\left(\dfrac{Nm}{k_\text BT}\right)v\exp\left(-\dfrac{v^2}{2k_\text BT}\right)\]
    与三维空间中同理可以推出
    \[f(E)=\dfrac{N}{k_\text BT}\exp\left(-\dfrac{E}{k_\text BT}\right)\]
    进行无穷积分可得
    \[\dfrac{N_{E_0\to\infty}}{N}=\int_{E_0}^{+\infty}f(E)\di E=\exp\left(-\dfrac{E_0}{k_\text BT}\right)\]
    
\end{derivation}
这就是我们更常用的能量分布公式.事实上,这正是Boltzmann分布的结果.
\begin{theorem}[1B.3.3 能量分布的近似公式]
    在二维空间中,气体分子的能量分布函数
    \[f(E)=\dfrac{N}{k_\text BT}\exp\left(-\dfrac{E}{k_\text BT}\right)\]
    气体分子能量大于某个定值$E_0$的概率为
    \[P_{E_0\to\infty}=\dfrac{N_{E_0\to\infty}}{N}=\exp\left(-\dfrac{E_0}{k_\text BT}\right)\]
    这一结论在能量分布的推导中更加常用.
\end{theorem}
\vspace{8pt}
\Section{\Large 1B.4 分子的碰撞频率,平均自由程与隙流}
\Part{分子的互碰频率与平均自由程}
\indent 分子以很高的速度做无规则运动,每时每刻都在发生大量的碰撞.研究气体分子的碰撞过程对于了解气体的扩散,热传导等现象具有重要的意义.
\begin{definition}[1B.4.1 自由程与平均自由程]
    分子在两次连续碰撞之间所经过的路程称为\tbf{自由程},记为$l$.\\
    自由程在无规则地不断地改变着,其平均值称为\tbf{平均自由程},记为$\overline{l}$.
\end{definition}
我们知道,分子之间的碰撞实际上是分子靠近后相斥而使它们又互相远离的过程.为了更方便地讨论这一过程,我们定义\tbf{有效直径}.
\begin{definition}[1B.4.2 有效直径]
    分子碰撞时,两个分子的质心所能达到的最短距离称为\tbf{有效直径},记为$d$.
\end{definition}
下面我们着手推导互碰频率与平均自由程.
\begin{derivation}
    设某个分子$X$的半径为$r$,平均速率为$v_a$,在时间$t$内与其它分子碰撞的次数为$z'$,那么显然有
    \begin{equation}
        \overline{l}=\dfrac{v_at}{z'}
    \end{equation}
    我们假定在体积$V$内均匀分布着$N$个分子.首先,假设其它分子都是静止的,仅考虑$X$分子.%
    它在这段时间走过的路程为$v_at$.我们要计算有多少分子会与之发生碰撞,就要考虑有多少分子会落在与$X$的质心距离小于$d=2r$的范围$D_X$内.\\
    换一个角度思考这个问题,我们只需知道区域$D_X$在这段时间扫过的体积和单位体积内的分子数$n$,就可以知道与之发生碰撞的分子总数.\\
    显然,$D_X$是一个以$X$为球心,半径为$d$的球.由于我们假设分子体积忽略不计,那么$d$相对于$v_at$应当很小.于是,这个球扫过的体积可以近似为多段圆柱(这是因为$X$由于碰撞而多次改变方向)的体积.%
    这些圆柱的总长度为$v_at$,圆柱的半径为$d$,因此
    \begin{equation}
        V_X=\pi d^2v_at
    \end{equation}
    又因为单位体积内的分子数目为
    \begin{equation}
        n=\dfrac{N}{V}
    \end{equation}
    于是与$X$发生碰撞的分子总数(实际上就是碰撞次数$z'$)
    \begin{equation}
        z'=n\pi d^2v_at
    \end{equation}
    然后再考虑每个分子都在运动的情况.这时,应当用平均相对速率$v_r$代替上面式子中的$v_a$.\\
    考虑两个分子$X,Y$碰撞时速度(不妨记为$\overrightarrow{v_X},\overrightarrow{v_Y}$)的夹角为$\theta$,它们的相对速率$v_r$应当为
    \[v_r=\left|\overrightarrow{v_X}-\overrightarrow{v_Y}\right|
    =\sqrt{\left|\overrightarrow{v_X}-\overrightarrow{v_Y}\right|^2}
    =\sqrt{\left|\overrightarrow{v_X}\right|^2+\left|\overrightarrow{v_Y}\right|^2-2\left|\overrightarrow{v_X}\right|\left|\overrightarrow{v_X}\right|\cos\theta}\]
    先代入所有分子的平均速率$v_a$,即$\left|\overrightarrow{v_X}\right|=\left|\overrightarrow{v_Y}\right|=v_a$,则有
    \[v_r=v_a\sqrt{2\left(1-\cos\theta\right)}\]
    $\theta$应在$[0,\pi]$上均匀分布,则$\cos\theta$应对称地取遍$[-1,1]$上的值.于是$(1-\cos\theta)$的均值为$1$,于是
    \begin{equation}
        v_r=\sqrt{2}v_a
    \end{equation}
    将(5)代入(4)可得
    \begin{equation}
        z'=\sqrt2v_a\pi d^2nt
    \end{equation}
    于是平均自由程
    \begin{equation}
        \overline{l}=\dfrac{v_at}{z'}=\dfrac{1}{\sqrt2\pi d^2n}
    \end{equation}
    现在,考虑所有$N$个分子,每个分子在时间$t$内发生碰撞的次数均为$z'$.又因为每次碰撞都是两个分子参与的,因此实际次数应是上面计数方法所得的一半.这样,在单位体积内的碰撞频率
    \begin{equation}
        \nu=\dfrac{Nz'}{2Vt}=\dfrac{\pi d^2n^2v_a}{\sqrt2}
    \end{equation}
    将$v_a=\sqrt{\dfrac{8RT}{\pi M}}$代入(8)有
    \begin{equation}
        \nu=2n^2\pi d^2\sqrt{\dfrac{RT}{\pi M}}
    \end{equation}
    由两种气体构成的体系,其推导方法较为复杂.我们将公式列在后面,如果你有兴趣,可以自行查阅推导方法.
\end{derivation}\setcounter{equation}{0}
\begin{theorem}[1B.4.3 气体分子的互碰频率与平均自由程]
    单组分气体的互碰频率$\nu$为
    \[\nu=2n^2\pi d^2\sqrt{\dfrac{RT}{\pi M}}\]
    平均自由程$\overline{l}$为
    \[\overline{l}=\dfrac{v_at}{z'}=\dfrac{1}{\sqrt2\pi d^2n}\]
    上面的式子中的$d$为有效直径,$n$为单位体积内的分子数目.\\
    对于由A,B两种分子构成的双组分体系,我们不加证明地给出其互碰频率
    \[\nu=\pi d_{\text{AB}}^2\sqrt{\dfrac{8RT}{\pi\mu}}n_\text An_\text B\]
    其中
    \[d_{\text{AB}}=\dfrac{d_\text A+d_\text B}{2}\]
    为两种分子的有效直径的平均值.$\mu$代表\tbf{折合质量},满足
    \[\dfrac1\mu=\dfrac{1}{M_\text A}+\dfrac{1}{M_\text B}\]
    $n_\text A,n_\text B$分别为单位体积内A和B的分子数目.
\end{theorem}
\Part{分子碰撞器壁的频率与隙流}
同样地,分子与器壁也存在碰撞.我们可以用相似的方法推导气体与器壁的碰撞频率.
\begin{derivation}
    考虑空间直角坐标系内分子的运动.根据气体分子运动的对称性,我们考虑$x$方向上的速率$v_x$.%
    我们已经在Maxwell分布的推导中得出了一维方向上的速率分布函数
    \[f\left(v_x\right)=\sqrt{\dfrac{m}{2\pi k_\text BT}}\exp\left(-\dfrac{mv_x^2}{2k_\text BT}\right)\]
    这是一个尚未归一化的分布函数.据此可得
    \[\overline{v_x}=\dfrac{\displaystyle\int_0^{+\infty}v_xf\left(v_x\right)\di v_x}{\displaystyle\int_0^{+\infty}f\left(v_x\right)\di v_x}=\sqrt{\dfrac{2k_\text BT}{\pi m}}\]
    式中的积分可以查表求得.这样,我们就有
    \[\overline{v_x}=\dfrac12v_a\]
    仍设$n$为单位体积内的分子数目.考虑垂直于$x$方向上的一侧器壁的面积元$\di A$,那么应当只有一半的分子朝向此器壁运动.%
    在时间$t$内能撞击$\di A$的分子数目应当与以$\di A$为底,$\overline{v_x}t$为高的柱体内的分子数目的一半相同.因此单位时间单位面积内碰撞该壁的碰撞频率
    \[\nu'=\dfrac{\overline{v_x}t\di A\cdot\dfrac n2}{\di A}=\dfrac{n\overline{v_x}}{2}=n\sqrt{\dfrac{k_\text BT}{2\pi m}}\]
    已知$pV=Nk_\text BT$,即$n=\dfrac{p}{k_\text BT}$,则有
    \[\nu'=\dfrac{p}{\sqrt{2\pi mk_\text BT}}\]
\end{derivation}
\begin{theorem}[1B.4.4 气体分子与器壁的碰撞频率]
    气体分子与器壁的碰撞频率(以分子数记)(简称\tbf{碰撞频率})$\nu'$为
    \[\nu'=\dfrac{p}{\sqrt{2\pi mk_\text BT}}\]
    如果碰撞频率以物质的量记,那么上式即为
    \[\nu'=\dfrac{p}{\sqrt{2\pi MRT}}\]
\end{theorem}
不知道你有没有思考过一个带有小孔的气球缓慢向外漏气的情形.%
这是一个与气体分子碰撞器壁息息相关的一个物理过程,即\tbf{隙流}(顾名思义,就是从缝隙中流出).
\begin{definition}[1B.4.5 隙流]
    气体从一个小孔向容器外泄露的过程被称作\tbf{隙流}.
\end{definition}
试想这一小孔即为我们在上述推导中的面积元$\di A$,那么原来应当撞击到$\di A$上的气体分子就改为泄露出容器.于是,隙流速率就等于碰撞频率.
\begin{theorem}[1B.4.6 隙流定律$^\ast$]
    气体隙流的速率$v$为
    \[v=\dfrac{p}{\sqrt{2\pi MRT}}\]
    因此,在同一温度下,气体隙流的速率与其摩尔质量的平方根成反比.这一定律也被称为\tbf{Graham定律}.
\end{theorem}
利用隙流定律,我们可以通过测量隙流的速率以求算气体的摩尔质量:只需测定该气体与已知气体的隙流速率之比即可.此外,隙流还可以应用于同位素分离中.
\end{document}