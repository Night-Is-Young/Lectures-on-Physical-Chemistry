\documentclass{ctexart}
\usepackage{PhysicalChemistryNote}

\begin{document}\pagestyle{plain}
\noindent\tbf{\LARGE 2B 热力学第一定律}\vspace{15pt}\\
\indent “退休后打算做什么?”\\
\indent “去当博物馆保安,看守‘永动机原型机’——反正它们永远不需要充电.”\\
\indent “那是个空壳子啊喂!”\vspace{12pt}\\
\Section{2B.1 热力学第一定律}
\Part{内能}
\indent 一块烧红了的铁和一块常温下安静地躺在地面上的铁,相信你对它们蕴含的能量大小肯定有一个依赖于直觉的判断.%
在热力学中,“物质蕴含的能量”有一个准确的定义——\tbf{内能}.
\begin{definition}[2B.1.1 内能]
    系统的总能量称为\tbf{内能},又称为\tbf{热力学能},记为$U$.\\
    通常,系统的内能是系统内所有粒子的动能与势能之和.
\end{definition}
需要说明的是,系统总体的动能和势能并不包含在内能之中.你携带在高铁上的咖啡\footnote{原稿为可乐,但考虑到可乐中的CO$_2$容易逸出造成内能改变,因此改为咖啡.}%
和在家里喝的咖啡虽然速度不同,但是内能是相同的;同样,这罐咖啡的内能也不会因为你把它带到山上就发生改变.%
不过,要是你有兴趣加热或者冰镇这瓶咖啡,它的内能自然就会发生改变.\\
\indent 我们需要指出下一事实.
\begin{theorem}[2B.1.2 内能的性质]
    内能是状态函数.
\end{theorem}
这不难理解,毕竟无论经历怎样的改变,只要系统的状态确定,其中粒子的运动情况和相互作用就是确定的,内能也就是确定的.\vspace{4pt}\\
\Part{热力学第一定律}
\indent 热力学第一定律的诞生与人类对“永动机”的追求密切相关.%
自中世纪起,许多人试图设计无需外部能量输入的机械(如利用重力,浮力或磁力的“自驱动机”),但均以失败告终.%
18世纪末,工业革命推动了对蒸汽机效率的研究,科学家逐渐意识到热,功与能量之间存在深层联系.\\
\indent 大量事实(例如Joule做的精确测定热功当量的实验)表明,能量不会凭空产生或消失,只会以不同的形式发生转化.这就是我们熟知的\tbf{能量守恒定律}.\\
\indent 能量守恒定理表明,一个孤立系统的总能量不会发生改变(这一系统不与环境发生能量或物质的交换,因而它的能量不会增加或减少).%
考虑到我们研究的体系的总能量一般指内能(通常你也不会让它整体做奇怪的运动),因此我们就有\tbf{热力学第一定律}.
\begin{theorem}[2B.1.3 热力学第一定律]
    隔离系统的内能是守恒的.其数学形式为
    \[\Delta U=Q+W\text{\ \ \ 或\ \ \ }\di U=\delta Q+\delta W\]

\end{theorem}
我们将在接下来的很多地方用到热力学第一定律.不过,在此之前,我们先引入一些别的状态函数以更好地描述系统.\vspace{12pt}\\
\Section{2B.2 焓与热容}
\Part{焓}
\indent 假定系统在变化过程中不做其余功,则根据热力学第一定律有
\[\Delta U=Q+W\]
如果系统的变化是等容过程,那么$W=0$,于是
\[\Delta U=Q\]
如果系统的变化是等压过程,那么不妨设压力保持为$p$,则有
\[W=-p\left(V_2-V_1\right)\]
即
\[U_2-U_1=Q-pV_2+pV_1\]
移项可得
\[Q=\left(U_2+pV_2\right)-\left(U_1+pV_1\right)\]
这告诉我们,等压过程下系统的热量变化等于始态和终态的$(U+pV)$之差.这促使我们定义一个新的状态函数以更好地描述等压过程.
\begin{definition}[2B.2.1 焓]
    \tbf{焓}是定义为$U+pV$的状态函数,记为$H$.
\end{definition}
上面的推导告诉我们,在没有其它功的情况下,等容过程中的热$Q_V$全部用于系统热力学能$U$的增加,而等压过程中的热$Q_p$全部用于系统焓的增加.%
尽管我们不知道$U$和$H$的具体值,却可以通过测量上面两种过程中的热效应来衡量过程中的内能变化$\Delta U$或焓变$\Delta H$.\\
\indent 由于化学反应更常见的是恒压反应,因此在处理化学问题时,焓也许更加常用.\vspace{4pt}\\
\Part{热容}
\indent 我们知道,使不同的系统升高相同的温度,需要提供的热也不同.在阳光照射下的沙滩,你会明显感觉沙子的温度比海水要高.%
因此,物质吸收热而升高温度的能力是不同的,这促使我们定义\tbf{热容}以定量地表述这种能力.
\begin{definition}[2B.2.2 热容]
    \tbf{热容}的定义是系统升高单位热力学温度时吸收的热,记为$C$,即
    \[C=\dfrac{\delta Q}{\di T}\]
    显然,热容与物质的量有关,因此定义\tbf{摩尔热容}为
    \[C_\m=\dfrac{C}{n}=\dfrac1n\dfrac{\delta Q}{\di T}\]

\end{definition}
我们已经知道,在等压和等容过程中,分别有
\[\delta Q_V=\di U\ \ \ \ \ \delta Q_p=\di H\]
\indent 因此可以定义\tbf{定压热容}和\tbf{定容热容}.
\begin{definition}[2B.2.3 定压热容和定容热容]
    定压热容$C_p$和定容热容$C_V$分别定义为
    \[C_p=\dfrac{\delta Q_p}{\di T}=\pa HTp\ \ \ \ \ C_V=\dfrac{\delta Q_V}{\di T}=\pa UTV\]
    以及与\tbf{2B.2.2}类似地,也可以定义\tbf{定压摩尔热容}$C_{\text p,\m}$和定容摩尔热容$C_{V,\m}$.
\end{definition}
有了上述两种热容,我们就可以计算等压过程中的焓变和等容过程中的内能变,即
\[\Delta H=\int_{T_1}^{T_2}C_p\di T\ \ \ \ \ \Delta U=\int_{T_1}^{T_2}C_V\di T\]
其中$T_1,T_2$分别为始态和终态的温度.\\
\indent 热容是温度的函数,这意味着它也可以做如下的展开
\[C_{p,\m}=a+bT+cT^2+\cdots\ \ \ \ \ C_{V,\m}=a'+b'T+c'T^2+\cdots\]
实际计算中也常常用到这类展开式以求更精确的计算.
\end{document}