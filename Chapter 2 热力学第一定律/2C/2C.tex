\documentclass{ctexart}
\usepackage{PhysicalChemistryNote}

\begin{document}\pagestyle{plain}
\noindent\tbf{\LARGE 2C 热力学第一定律对气体的应用}\vspace{15pt}\\
\indent 作者很懒,于是什么都没留下.向各位征集导言.\vspace{12pt}\\
\Section{2C.1 热力学第一定律对理想气体的应用}
\Part{理想气体的$U$和$H$}
\indent Gay-Lussac和Joule分别独立完成了真空膨胀实验.%
他们在水浴的容器中添加一块隔板,一边充入高压气体,另一边抽成真空.随后,撤掉挡板,气体向真空膨胀.%
水浴内的温度计显示温度没有变化,因此这个过程满足$Q=W=0$,因此热力学能$U$没有变化.我们将从这一实验结果推出一个重要的推论.
\begin{derivation}
    对于定量的理想气体,其内能$U$可以用$p,V,T$三个变量中的两个进行表示.\\
    假如我们选择$T,V$作为$U$的变量,对$U$全微分可得
    \[\di U=\pa UTV\di T+\pa UVT\di V\]
    根据上述实验结果,有
    \[\di U=0\ \ \ \ \ \di T=0\ \ \ \ \ \di V\neq0\]
    于是
    \[\pa UVT=0\]
    换用$p,T$作为$U$的变量,同理可得
    \[\pa UpT=0\]
    这表明温度一定时,理想气体的$U$不随$p,V$的改变而改变.
    又因为
    \[\pa HVT=\pa{(U+pV)}VT=\pa UVT+\pa{pV}VT\]
    而对于定量的理想气体,温度一定时有$pV=nRT=\text{定值}$,于是$\pa {pV}VT=0$,从而
    \[\pa HVT=0\]
    同理可得
    \[\pa HpT=0\]
    这表明温度一定时,理想气体的$H$也不随$p,V$的改变而改变.
\end{derivation}
\begin{theorem}[2C.1.1 Joule定律]
    理想气体的热力学能和焓都是仅以温度为自变量的函数,与压力和体积无关,即
    \[U=U(T)\ \ \ \ \ H=H(T)\]

\end{theorem}
我们可以从分子动理论进行简单的解释:热力学能是分子的动能与相互作用的势能的总和.对于理想气体,动能仅与温度有关,%
而分子之间的相互作用力被忽略,即没有相互作用的势能这一项,因此其热力学能仅与温度有关.
\begin{hint}
    严格来讲,Gay-Lussac和Joule的实验并不精确,因为水的热容很大,一点小的热量变化也不会使温度发生明显改变.%
    不过,我们仍然可以通过外推$p\to0$的情形来说明Joule定律的合理性.\\
    在学习\tbf{Chapter 3}之后,我们可以根据Maxwell关系式来确证Joule定律的合理性.可以证明
    \[\pa UVT=T\pa pTT-p\ \ \ \ \ \pa HpT=V-T\pa VTp\]
    对于理想气体有$pV=nRT$,代入上式可得
    \[\pa UVT=\pa HpT=0\]
    而
    \[\pa UpT=\pa UVT\pa VpT=0\ \ \ \ \ \pa HVT=\pa HpT\pa pVT=0\]
    于是$U,H$均不随$p$和$V$的改变而改变.
\end{hint}
\Part{理想气体的$C_p$和$C_V$}
\indent 考虑到
\[C_p=\pa UTV\ \ \ \ \ C_V=\pa HTp\]
于是理想气体的定容热容也是仅与温度有关的函数.\\
\indent 对于等压过程,气体除了吸收热量升温之外,还要多吸收一部分热量膨胀对外做功,因此$C_p$总是比$C_V$大.%
我们现在来求两者的差值.
\setcounter{equation}{0}
\begin{derivation}
    首先考虑对于所有系统的普适情况.\\
    对于一个一般的系统,我们有
    \begin{equation}
        \begin{aligned}
            C_p-C_V
            &= \pa HTp-\pa UTV =\pa{(U+pV)}Tp-\pa UTV \\
            &= \pa UTp+p\pa VTp-\pa UTV
        \end{aligned}
    \end{equation}
    考虑$U$作为$V,T$的函数和$V$作为$p,T$的函数,即
    \[U=U(V,T)\ \ \ \ \ V=V(p,T)\]
    根据复合函数的链式求导法则有
    \begin{equation}
        \pa UTp=\pa UTV+\pa UVT\pa VTp
    \end{equation}
    将(2)代入(1)中可得
    \begin{equation}
        C_p-C_V=p\pa VTp+\pa UVT\pa VTp=\left[p+\pa UVT\right]\pa VTp
    \end{equation}
    上式就是一般气体的定压热容与定容热容之差的公式.对于理想气体,又有
    \begin{equation}
        \pa UVT=0\ \ \ \ \ \pa VTp=\dfrac{nR}{p}
    \end{equation}
    将(4)代入(3)可得
    \begin{equation}
        C_p-C_V=nR
    \end{equation}
    或
    \begin{equation}
        C_{p,\m}-C_{V,\m}=R
    \end{equation}

\end{derivation}
于是,我们得到了以下重要定理.
\begin{theorem}[2C.1.2 气体的$C_p$与$C_V$之差]
    对于一般的气体有
    \[C_p-C_V=\left[p+\pa UVT\right]\pa VTp\]
    对于理想气体有
    \[C_p-C_V=nR\text{\ \ \ 或\ \ \ }C_{p,\m}-C_{V,\m}=R\]

\end{theorem}
在以后,你也许会经常遇到要使用$C_p$或$C_V$却只给出了另外一种数据的情形,此时\tbf{2C.1.2}就将发挥它的作用.\vspace{4pt}\\
\Part{理想气体的绝热过程}
\indent 绝热过程是化学反应中的常见过程.对于气体的绝热膨胀,由于向外界做功的同时不能从外界吸收热,%
于是根据热力学第一定律可知气体的内能减小,温度必然降低.我们在此先讨论理想气体的绝热过程.
\begin{derivation}
    在绝热过程中,$Q=0$.在不做非膨胀功的情况下,我们有
    \[\di U=\delta W=-p\di V\]
    对$U$做全微分有
    \[\di U=\pa UTV\di T+\pa UVT\di V=C_V\di T\]
    由于理想气体满足$p=\dfrac{nRT}{V}$,于是代入有
    \[-\dfrac{nRT}{V}\di V=C_V\di T\]
    即
    \[\dfrac{\di T}{T}+\dfrac{nR}{C_V}\dfrac{\di V}{V}=0\]
    我们已经知道$C_p-C_V=nR$.令\tbf{热容比}$\gamma=\dfrac{C_p}{C_V}$,则有
    \[\dfrac{\di T}{T}+(\gamma-1)\dfrac{\di V}{V}=0\]
    如果$C_V$与温度无关(我们将来会知道对于理想气体,其热容确实是常数),那么上述式子积分可得
    \[TV^{\gamma-1}=\text{常数}\]
    将$T=\dfrac{pV}{nR}$代入上式可得
    \[pV^\gamma=\text{常数}\]

\end{derivation}
因此,我们可以得到理想气体绝热过程中满足的条件.
\begin{theorem}[2C.1.3 理想气体的绝热过程]
    理想气体从$T_1$变化至$T_2$的绝热过程满足
    \[W=\int_{T_1}^{T_2}C_V\di T=C_V\left(T_2-T_1\right)\]
    且
    \[p_1V_1^\gamma=p_2V_2^\gamma\]

\end{theorem}
在实际情况下,气体的变化一般介于等温过程和绝热过程之间,此时有%
$pV^\beta=\text{常数}(1<\beta<\gamma)$.这种过程被称为\tbf{多方过程}.\vspace{12pt}\\
\Section{2C.2 Carnot循环\footnote{音译为“卡诺循环”.}}
\indent 自从蒸汽机被发明以来,人们一直利用热机做功而应用于生产生活.
\begin{definition}[2C.2.1 热机]
    \tbf{热机}是能够将热源提供的一部分热量转化成为对外输出的机械能的机器.\\
    一般的热机由\tbf{高温热源}\footnotemark,\tbf{低温热源},\tbf{工作系统}构成,%
    在热源和工作系统之间用来进行能量传递和转化的媒介叫做\tbf{工作物质}.
\end{definition}
\footnotetext{热源是具有恒定温度且温度不随吸收或放出热量而发生的系统.}
\indent 19世纪初,蒸汽机的效率很低,只有3\%到5\%,大量的能量被浪费.%
热机工程师们望从理论上找出提高热机效率的办法.1824年,年轻的法国工程师N.L.S.Carnot设计了一个由两个等温过程和两个绝热过程构成的最简单的理想循环,后来被称为Carnot循环.%
这一热力学循环的提出为提高热机效率指明了方向,为推动热力学理论的发展做出了重要贡献.\vspace{4pt}\\
\Part{Carnot循环}
\indent 我们先介绍Carnot设计的循环过程.
\begin{figure}[H]
    \centering\documentclass{standalone}
\usepackage{PhysicalChemistryNote}
\begin{document}
\begin{tikzpicture}
    \draw[->] (0,0) -- (5,0) node[right]{$V$};
    \draw[->] (0,0) -- (0,5) node[above]{$p$};
    \draw[domain=1.0400:2.6207] plot[smooth](\x,4/\x);
    \draw[domain=2.6207:4.1602] plot[smooth](\x,{12*sqrt(2)/(\x^2.5)});
    \draw[domain=1.6510:4.1602] plot[smooth](\x,2/\x);
    \draw[domain=1.0400:1.6510] plot[smooth](\x,{3*sqrt(2)/(\x^2.5)});
    \coordinate[label=above:{$A\left(p_1,V_1,T_\h\right)$}] (1) at (1.0400,3.8460);
    \coordinate[label=above right:{$B\left(p_2,V_2,T_\h\right)$}] (2) at (2.6207,1.5263);
    \coordinate[label=right:{$C\left(p_3,V_3,T_\c\right)$}] (3) at (4.1602,0.4807);
    \coordinate[label={[xshift=-0.4cm,yshift=-0.85cm]$D\left(p_4,V_4,T_\c\right)$}] (4) at (1.6510,1.2114);
    \node[circle,fill,inner sep=1pt] at (1) {};
    \node[circle,fill,inner sep=1pt] at (2) {};
    \node[circle,fill,inner sep=1pt] at (3) {};
    \node[circle,fill,inner sep=1pt] at (4) {};
\end{tikzpicture}
\end{document}
\end{figure}
\begin{definition}[2C.2.2 Carnot循环]
    以$n\mol$理想气体为工作物质,温度为$T_\h$的高温热源和温度为$T_\c$的低温热源作为能量来源,进行如下四个步骤.
    \begin{enumerate}[label=\tbf{\arabic*.}]
        \item \tbf{等温可逆膨胀}\\
            系统与高温热源$T_\h$接触,作等温可逆膨胀,由状态$A\left(p_1,V_1,T_\h\right)$变为状态$B\left(p_2,V_2,T_\h\right)$.
        \item \tbf{绝热可逆膨胀}\\
            系统离开热源,作绝热可逆膨胀,由状态$B\left(p_2,V_2,T_\h\right)$变为状态$C\left(p_3,V_3,T_\c\right)$.
        \item \tbf{等温可逆压缩}\\
            系统与高温热源$T_\c$接触,作等温可逆压缩,由状态$C\left(p_3,V_3,T_\c\right)$变为状态$D\left(p_4,V_4,T_\c\right)$.
        \item \tbf{绝热可逆压缩}\\
            系统离开热源,作绝热可逆压缩,由状态$D\left(p_4,V_4,T_\c\right)$变为状态$A\left(p_1,V_1,T_\h\right)$.
        \end{enumerate}
\end{definition}
接下来,我们来推导Carnot循环中各步骤的功和热,以计算整个循环的效率.
\begin{derivation}
    \begin{enumerate}[label=\tbf{\arabic*.}]
        \item 等温可逆膨胀时$\Delta U_1=0$,且$\displaystyle W_1=-\int_{V_1}^{V_2}=nRT_\h\ln\dfrac{V_1}{V_2}$,于是$Q_\h=-W_1=nRT_\h\dfrac{V_2}{V_1}$.
        \item 绝热可逆膨胀时$Q=0$,则$\displaystyle W_2=\Delta U_2=\int_{T_\h}^{T_\c}C_V\di T$.
        \item 等温可逆压缩时$\Delta U_3=0$,且$\displaystyle W_3=-\int_{V_3}^{V_4}=nRT_\c\ln\dfrac{V_3}{V_4}$,于是$Q_\c=-W_3=nRT_\c\dfrac{V_4}{V_3}$.
        \item 绝热可逆压缩时$Q=0$,则$\displaystyle W_4=\Delta U_4=\int_{T_\c}^{T_\h}C_V\di T$.
    \end{enumerate}
    由于\tbf{2.}和\tbf{4.}均为绝热过程,根据\tbf{2C.1.3}可知
    \[T_\h V_2^{\gamma-1}=T_\c V_3^{\gamma-1}\ \ \ \ \ T_\c V_4^{\gamma-1}=T_\h V_1^{\gamma-1}\]
    于是
    \[\dfrac{V_1}{V_2}=\dfrac{V_4}{V_3}\]
    于是
    \[W_{\text{tot}}=W_1+W_2+W_3+W_4=nRT_\h\ln\dfrac{V_1}{V_2}+nRT_\c\ln\dfrac{V_3}{V_4}=nR\left(T_\h-T_\c\right)\ln\dfrac{V_1}{V_2}\]
    \[Q=-W=nR\left(T_\h-T_\c\right)\ln\dfrac{V_2}{V_1}\]

\end{derivation}
于是我们有
\begin{theorem}[2C.2.3 Carnot循环的功和热]
    系统完成一次\tbf{2C.2.1}所示的Carnot循环,做的总功为
    \[W=nR\left(T_\h-T_\c\right)\ln\dfrac{V_1}{V_2}\]
    从高温热源吸收的热$Q_\h$为
    \[Q_\h=nRT_\h\ln\dfrac{V_2}{V_1}\]
    向低温热源放出的热$Q_\c$为
    \[Q_\c=nRT_\c\ln\dfrac{V_1}{V_2}\]

\end{theorem}
\Part{热机效率}
\indent 热机从高温热源吸热$Q_\h$,仅将一部分转化为功,将另一部分热$Q_\c$传递给低温热源.%
我们可以定义\tbf{热机效率}来衡量热机将吸收的热转化为功的比例.
\begin{definition}[2C.2.4 热机效率]
    将热机对环境做的功$W$和从高温热源吸收的热$Q_\h$之比称为\tbf{热机效率},记为$\eta$.    
\end{definition}
\setcounter{equation}{0}
对于Carnot热机有
\begin{equation}
    \eta=\dfrac{-W}{Q_\h}=\dfrac{nR\left(T_\h-T_\c\right)\ln\dfrac{V_2}{V_1}}{nRT_\h\ln\dfrac{V_2}{V_1}}=\dfrac{T_\h-T_\c}{T_\h}=1-\dfrac{T_\c}{T_\h}
\end{equation}
\begin{equation}
    \eta=\dfrac{-W}{Q_\h}=\dfrac{Q_\c+Q_\h}{Q_\h}=1+\dfrac{Q_\c}{Q_\h}
\end{equation}
因此,Carnot热机的效率$\eta<1$.我们将在\tbf{Chapter 3}中证明所有工作于$T_\h$和$T_\c$的热机的效率都不可能超过Carnot热机,从而所有热机的效率都小于$1$.\vspace{4pt}\\
\Part{制冷机与热泵}
\indent 如果把Carnot热机沿$A-D-C-B-A$的途径倒开,就得到了一个制冷机.此时,环境对系统做功,系统从低温热源吸收热量$Q_\c'$,向高温热源放出热量$Q_\h'$.%
与热机相似的,我们可以定义制冷效率.
\begin{definition}[2C.2.5 制冷效率]
    将制冷机从低温热源吸收的热量$Q_\c'$与环境对系统做的功$W$之比称为\tbf{制冷效率},记为$\beta$.
\end{definition}
对于Carnot热机有
\[\beta=\dfrac{Q_\c'}{W}=\dfrac{T_\c}{T_\h-T_\c}\]
\indent 对于制冷机,我们关注的是通过做功使得从低温热源中吸收热量.%
对于热泵,虽然它的本质是和制冷机相同的,不过我们更关心它向高温热源输送的热量.同样地,我们有热泵的工作效率.
\begin{definition}[2C.2.6 热泵的工作效率]
    将热泵向高温热源放出的热量$Q_\h'$与环境对系统做的功$W$之比称为\tbf{热泵的工作效率},记为COP.
\end{definition}
对于Carnot热机有
\[COP=\dfrac{Q_\h'}{W}=\dfrac{T_\h}{T_\h-T_\c}\]
\indent 一般的商用热泵的COP在$2$到$7$之间,说明做$\SI{1}{J}$的功可以向高温热源输送$\SI{2}{J}$到$\SI{7}{J}$热能.%
这比直接用这些功转化为热量要高效得多.\vspace{12pt}\\
\Section{2C.3 热力学第一定律对实际气体的应用}
\Part{Joule-Thomson效应}
\indent 前面已经指出,Joule所做的自由膨胀实验并不精确.因此,在1853年,他与Thomson合作完成了\tbf{节流膨胀实验}.\\
\indent 节流膨胀实验在一段内部装有多孔塞(或节流阀)\footnote{Joule也许采用的是棉花或者软木塞.}的绝热管道中进行.将气体通入管道的一端,并调节该侧气体的压力恒定为$p_1$.%
随后,气体缓慢流经多孔塞,由于多孔塞的局部阻力,气体压力降低至$p_2$.%
经过一段时间达到稳定态后,入口和出口的气体温度各自稳定在$T_1$和$T_2$.这一过程被称为\tbf{节流膨胀过程}.
\begin{derivation}
    我们假定体积为$V_1$的气体经过上述节流膨胀后的体积变为$V_2$.\\
    后来的气体对这气体做等温压缩,压力为$p_1$,体积由$V_1$变化至$0$,做的功为
    \[W_1=-p_1\Delta V=p_1V_1\]
    气体从多孔塞出来后等温膨胀,对前面的气体做功,压力为$p_2$,体积由$0$变化至$V_2$,做的功为
    \[W_2=-p_2\Delta V=-p_2V_2\]
    又因为这一过程是绝热过程,于是$Q=0$.根据热力学第一定律有
    \[\Delta U=U_2-U_1=p_1V_1-p_2V_2\]
    即
    \[U_1+p_1V_1=U_2+p_2V_2\]
    即
    \[H_1=H_2\]

\end{derivation}
\begin{theorem}[2C.3.1 节流膨胀过程]
    节流膨胀前后,气体的焓不变.
\end{theorem}
实验发现实际气体经过节流膨胀后温度会发生改变,这表明实际气体的焓可能还与压力有关.Joule和Thomson为了定量描述这种效应而提出了Joule-Thomson系数.
\begin{definition}[2C.3.2 Joule-Thomson系数]
    实际气体经节流膨胀后的温度变化与压力变化的比值的微分形式为
    \[\mu_{\text{J-T}}=\pa TpH\]
    称作\tbf{Joule-Thomson系数}.它表示气体经过节流膨胀后温度随压力的变化率,是一个微分效应(即$H$一定时$T-p$曲线的斜率),因此又称为\tbf{微分节流效应}.
\end{definition}
在常温下,一般气体的$\mu_{\text{J-T}}$都是正值,这表明经过节流膨胀后温度下降.不过,H$_2$和He等气体的$\mu_{\text {J-T}}$在常温下是负值.%
不过,在较低的温度下,它们的$\mu_{\text{J-T}}$也会变成正值.$\mu_{\text{J-T}}=0$时的温度称为\tbf{转化温度}.\\
\indent 下面讨论$\mu_{\text{J-T}}$的正负情况.
\begin{derivation}
    对于定量的气体,$H$是$T,p$的函数.对$H$做全微分有
    \[\di H=\pa HTp\di T+\pa HpT\di p\]
    节流膨胀过程是等焓过程,因此$\di H=0$,于是
    \[\begin{aligned}
        \mu_{\text{J-T}}
        &= \pa TpH = -\dfrac{\pa HpT}{\pa HTp} = -\dfrac{1}{C_p}\pa{(U+pV)}pT \\
        &= \left[-\dfrac1{C_p}\pa UpT\right]+\left[-\dfrac1{C_p}\pa{(pV)}pT\right]
    \end{aligned}\]
    对于理想气体,由于$\displaystyle\pa UpT=\pa{(pV)}pT=0$,因此其$\mu_{\text{J-T}}=0$.\\
    而对于实际气体,由于其内能不仅是温度的函数,还与压力有关.对于一般的气体,等温下压力增大,必须吸收能量以克服分子间作用力,因此
    \[\pa UpT<0\]
    而后一项则可以由等温$p-V$图求出.
\end{derivation}
Joule-Thomson效应最重要的用途是给气体降温.显然,只有$\mu_{\text{J,T}}>0$的气体才能通过节流膨胀而降温.\vspace{4pt}\\
\Part{实际气体的$U$和$H$}
\indent 在下一章中,借助Maxwell关系式可以得出
\[\pa UVT=T\pa pTV-p\]
\[\pa HpT=V-T\pa VTp\]
因此,只要知道实际气体的状态方程,就可以求出等温下实际气体的$U,H$随$p,V$变化的关系.%
下面以van der Waals气体为例求$U,H$随$p,V$的变化关系.
\begin{derivation}
    \[\pa UVT=T\cdot\dfrac{nR}{V-nb}-p=\dfrac{an^2}{V^2}\]
    所以等温条件下van der Waals气体从体积为$V_1$变化至体积为$V_2$的内能变和焓变为
    \[\Delta U=an^2\left(\dfrac1{V_1}-\dfrac{1}{V_2}\right)\]
    \[\Delta H=an^2\left(\dfrac1{V_1}-\dfrac{1}{V_2}\right)+\Delta(pV)\]

\end{derivation}
按照分子动理论的观点,体积膨胀后应当克服分子之间的引力做功.因此,可以用等温膨胀时反抗分子间引力(即我们在\tbf{1C.3.1}中提到的内压力$p_\i$)所做的功衡量此过程的内能变化,即
\[p_\i=\pa UVT=\dfrac{a}{V_\m^2}\ \ \ \ \ \Delta U=p_\i\di V\]
可以发现,这样的推导和\tbf{1C.3}中的论述得到了相同的结果.这进一步表明了van der Waals方程的正确性.
\end{document}