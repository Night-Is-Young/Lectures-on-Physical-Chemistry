\documentclass{ctexart}
\usepackage{PhysicalChemistryNote}

\begin{document}\pagestyle{plain}
\noindent\tbf{\LARGE 3A 熵与热力学第二定律}\vspace{15pt}\\
\indent “既然大海是一个巨大的恒温热源,那么我们为什么不提取海水中的热量来给远洋航行的船功能呢?”\\
\indent “傻瓜,你不知道热力学第二定律吗?”\vspace{12pt}\\
\Section{3A.1 热力学第二定律}
\indent 想象一个网球在地上弹跳的过程,显然它会随着弹跳次数的增加而降低高度,最终静止在地面上.%
在这一过程中,网球的机械能转化为地面和网球的热能.%
显然,这一过程是自发进行%
的,无需外力的干涉(除了重力,不过我们可以把地球和网球作为系统).
\begin{hint}
    关于自发过程的准确的描述,也许并没有一个很清晰的答案.一般来说,我们认为自发过程是使得系统从非平衡态趋向平衡态的过程,或者认为它是热力学上的不可逆过程.%
    这两种说法应当是等价的,并且也符合我们对客观世界的认识.
\end{hint}
\indent 然而,从来没有人看到过一个在地上的网球自发的降低它的温度而从地面上弹跳起来.%
换言之,在不施加干预的情况下,热能不能自发的转化为机械能.\\
\indent 种种迹象表明,热能和其余形式的能量似乎不能随意的转化.%
因此,在观察了大量实验现象后,Kelvin和Clausius分别总结出了热力学第二定律的两种表述.
\begin{theorem}[3A.1.1 热力学第二定律]
    热力学第二定律有如下两种自然语言表述方式.\\
    \tbf{Kelvin表述}:不可能从单一热源取出热使其完全转化为功而不发生其它变化.\\
    \tbf{Clausius表述}:不可能把热从低温物体传到高温物体而不发生其它变化.
\end{theorem}
尽管表述方法有差别,实际上两种说法是等价的.我们在此用Carnot热机做简单地说明.
\begin{proof}
    如果Clausius表述不成立,考虑一个工作于高温热源$T_\h$和低温热源$T_\c$之间的Carnot热机.%
    假定Carnot热机从高温热源吸热$Q_\h$,其中一部分$Q_\c$传给低温热源,%
    另一部分$W=Q_\h-Q_\c$用于做功,然后再从低温热源传递$Q_\c$的热量至高温热源.%
    这一过程的净结果是Carnot热机从高温热源吸收了$Q_\h-Q_\c$的热量并全部用于做功,这与Kelvin表述相悖.\\
    如果Kelvin表述不成立,那么假定某热机能从低温热源$T_\c$吸热$Q$并将其全部转化为功$W$,%
    然后考虑另一个热机将功$W$全部转化为热$Q$并传递给高温热源$T_\h$(这在实际中是可以做到的).%
    这一过程的净结果是从低温热源吸热$Q$并全部传递给高温热源而不发生其它变化,这表明Clausius表述不成立.\\
    于是我们可以知道两者是等价的.
\end{proof}
热力学第二定律否定了第二类永动机\footnote{第二类永动机即违反Kelvin表述的机器,例如从海水中提取热量驱动轮船工作(这实际上需要一个与海洋规模相当且温度低于海水温度的热源).%
因此Kelvin表述也可以表达为:第二类永动机不可能造成.}%
的存在,我们不能随意地无限地从一个热源中提取热量,%
也不能将热全部转化为功而不引起其余变化.这表明热机的效率存在一定上限.下面我们来探究热机能达到的最高的工作效率.\vspace{12pt}\\
\Section{3A.2 Carnot定理}
\indent Carnot循环为提高热机效率提供了理论指导.事实上,它也成为了热机效率的上限.尽管提出这一点早于热力学第二定律的提出,%
不过正确地证明它仍然需要用到热力学第二定律.下面我们来论证之.
\begin{proof}
    设在高温热源$T_\h$和低温热源$T_\c$之间有可逆机$C$(例如Carnot热机)和另一任意的热机$I$工作.\\
    调节两个热机做的功相等,即$C$从高温热源吸热$Q_C$,做功$W$,放热$Q_C-W$到低温热源;%
    $I$从高温热源吸热$Q_I$,做功$W$,放热$Q_I-W$到低温热源.于是两者的热机效率分别为
    \[\eta_C=-\dfrac{W}{Q_C}\ \ \ \ \ \eta_I=-\dfrac{W}{Q_I}\]
    取$W$为其绝对值.\\
    现在,假设$I$的效率大于$C$,于是
    \[\eta_I=\dfrac{W}{Q_I}>\dfrac{W}{Q_C}=\eta_C\]
    即$Q_C>Q_I$.\\
    现在以$I$带动$C$倒开,使得$C$成为制冷机.这样,$I$从高温热源吸热$Q_I$后放热$Q_I-W$到低温热源,%
    然后$C$从低温热源吸热$Q_C-W$,接受$I$做的功$W$后将$Q_C$的热量传递至高温热源.\\
    由于$Q_C-Q_I>0$,于是这一过程的净结果是从低温热源吸收$Q_C-Q_I$的热量传递至高温热源,%
    而没有引起其它变化.这显然违反了Clausius表述,因此我们的假设不成立,于是
    \[\eta_I\leqslant\eta_C\]
    这就证明了我们的命题.\\
    另外,这一证明过程还说明任何可逆机的效率都是相同的.考虑将上述热机$I$换成可逆机$C'$,则有
    \[\eta_{C'}\leqslant\eta_C\]
    交换$C$与$C'$的位置,又可以得到
    \[\eta_{C}\leqslant\eta_{C'}\]
    于是
    \[\eta_{C'}=\eta_C\]

\end{proof}
\begin{theorem}[3A.2.1 Carnot定理]
    所有工作于同温热源和同温冷源的热机,其效率都不可能超过可逆机.
\end{theorem}
\Section{3A.3 熵}
\Part{熵的热力学定义}
\indent 回忆我们在\tbf{2C.2.4}中提到的Carnot热机的效率
\[\eta=1-\dfrac{T_\c}{T_\h}=1+\dfrac{Q_\c}{Q_\h}\]
\indent 这表明
\[\dfrac{Q_\c}{T_\c}+\dfrac{Q_\h}{T_\h}=0\]
\indent 再回想我们再上一节中所证明的所有可逆机的效率都相等,可逆循环过程是否都满足上面的条件呢?答案是肯定的\footnote{这里的证明只做简略的说明.}.
\begin{proof}
    考虑一个可逆循环的过程$X$,在$p-V$图上应当表现为一条闭合的曲线$L$.我们为其选定一个方向(例如逆时针方向),并记作$L^+$.\\
    在$L$上按循环的方向取两点$P,Q$,过$P,Q$两点作可逆绝热线$l_P,l_Q$.\\
    然后作等温线$l_1$分别交$l_P,l_Q$于$A,B$,并且使得气体按$P-A-B-Q$和$P-Q$的过程所作的功相等%
    (这实际上只需要等温线截$L,l_P,l_Q$所得的两个三角形面积相等).\\
    现在,假定$l_P,l_Q$再交$L$于$M,N$,类似地,作等温线$l_2$交$l_P,l_Q$于$C,D$,使得气体按$M-C-D-N$和$M-N$的过程所作的功相等.\\
    这样,$l_1,l_2,l_P,l_Q$分别是两条等温线和绝热线,于是$A-B-D-C$构成一个Carnot循环.\\
    我们用上述办法把$L$足够精细地划分,使得$L$的每一段都可以用Carnot循环来代替.\\
    对于每一个小的Carnot循环,都有
    \[\dfrac{\delta Q_{i,\h}}{T_{i,\h}}+\dfrac{\delta Q_{i,\c}}{\delta T_{i,\c}}=0\]
    其中由于我们的分割足够精细,因此$Q$可以写成无穷小量$\delta Q$.%
    我们把上面的式子对$i$求和(高温热源和低温热源的温度/热量可以归属于$L^+$这一整个循环,就不必区分),即有
    \[\sum\dfrac{Q_i}{T_i}=0\]
    由于可逆循环在$p-V$图上是性质良好的光滑图像,因此将其无限细分并进行上述替代是可行的.%
    将求和符号改写为积分,就有
    \[\oint_{L^+}\dfrac{\delta Q}{T}=0\]
    这里的$\displaystyle\oint$符号为环程积分符号,意为按照曲线$L^+$的路径进行积分\footnotemark.\\
    考虑$L$上的两点$A,B$,有两种方向相反的沿$L$从$A$到$B$的路径,不妨记为$\left(\wideparen{AB}\right)^+$和$\left(\wideparen{AB}\right)^-$,就有
    \[\oint_{L^+}\dfrac{\delta Q}{T}=\int_{\left(\wideparen{AB}\right)^+}\dfrac{\delta Q}{T}-\int_{\left(\wideparen{AB}\right)^-}\dfrac{\delta Q}{T}=0\]
    于是
    \[\int_{\left(\wideparen{AB}\right)^+}\dfrac{\delta Q}{T}=\int_{\left(\wideparen{AB}\right)^-}\dfrac{\delta Q}{T}\]
    把任意两条从$A$到$B$的可逆过程对应的路径拼接成$L$,就可知对于任意可逆过程,$\displaystyle\int_{AB}\dfrac{\delta Q}{T}$是定值.%
    于是Clausius据此定义了一个状态函数$S$,其变化值就是任意可逆过程中$\displaystyle\int_{AB}\dfrac{\delta Q}{T}$的值,即
    \[S_{B}-S_{A}=\Delta S=\int_{AB}\dfrac{\delta Q}{T}\]
    将上式两边微分可得
    \[\di S=\dfrac{\delta Q}{T}\]
    这个状态函数就是我们所说的\tbf{熵}.
\end{proof}\footnotetext{实际上这是第一型曲线积分.关于曲线积分的内容详见数学基础部分.}
\begin{definition}[3A.3.1 熵的热力学定义]
    \tbf{熵}(用符号$S$表示)是这样的一个状态函数,对给定系统的任意可逆过程$L$都有
    \[\Delta S=\int_{L}\dfrac{\delta Q}{T}\]
    其微分形式为
    \[\di S=\dfrac{\delta Q}{T}\]
    需要注意的是,上面的$\delta Q$指可逆过程的热量变化.
\end{definition}
\Part{Clausius不等式}
\indent 我们已经定义了一个全新的热力学函数以描述系统的某种性质.尽管它的中文名——熵是好理解的,由胡刚复教授取热温商(即可逆过程中热量与温度的商)之意译成.\\
\indent 然而,Clausius给它取的名字却为Entropy,其希腊语意为:一个系统不受外部干扰时往内部最稳定状态发展的特性.%
这和我们刚刚的推导似乎没有什么关系,毕竟我们一直讨论的是可逆过程,而这里似乎谈论的是自发变化(即不可逆过程)的特性.为此,我们需要知道这句话和我们定义的熵之间的联系.
\begin{derivation}
    回忆我们在证明Carnot定理时使用的两个热机$C$和$I$.不难知道
    \[\eta_I=\dfrac{Q_{I,\c}+Q_{I,\h}}{Q_{I,\h}}=1+\dfrac{Q_{I,\c}}{Q_{I,\h}}\]
    \[\eta_C=1-\dfrac{T_\h}{T_\c}\]
    现在假定$I$是不可逆机,则$\eta_I<\eta_C$,于是
    \[1+\dfrac{Q_{I,\c}}{Q_{I,\h}}<1-\dfrac{T_\c}{T_\h}\]
    即
    \[\dfrac{Q_{I,\c}}{T_\c}+\dfrac{Q_{I,\h}}{T_\h}<0\]
    考虑由热机$I$完成的从状态$A$到状态$B$的不可逆过程$L_1$,并假设$I$与多个热源$T_1,\cdots,T_n$接触,分别吸收的热量为$\li Q,n$,于是有
    \[\sum\dfrac{\delta Q_i}{T_i}<0\]
    再考虑状态$B$到状态$A$的可逆过程$L_2$.$L_1+L_2$仍是一个不可逆过程,于是有
    \[\oint_{L_1+L_2}\dfrac{\delta Q}{T}<0\]
    这里把求和符号改为积分符号,且由于$L_1+L_2$是一个循环构成,故采取环程积分.又由于
    \[\int_{L_2}\dfrac{\delta Q}{T}=\Delta S=S_A-S_B\]
    代入上式有
    \[\int_{L_1}\dfrac{\delta Q}{T}<S_A-S_B\]
    这表明系统经历不可逆过程的热温商小于系统的熵变.如果$A$和$B$足够接近,就有微分形式
    \[\dfrac{\delta Q}{T}<\di S\]
    这里的$\delta Q$是不可逆过程的热量变化.
\end{derivation}
这样就得到了热力学第二定律的数学表示形式——Clausius不等式.
\begin{theorem}[3A.3.2 Clausius不等式]
    对于任意过程都有
    \[\Delta S\geqslant\int\dfrac{\delta Q}{T}\]
    或其微分形式
    \[\di S\geqslant\dfrac{\delta Q}{T}\]
    当且仅当该过程是可逆过程时等号成立.
\end{theorem}
现在,我们知道熵变可以判断过程是否是可逆变化.这就涉及到了熵增加原理.\vspace{4pt}\\
\Part{熵增加原理}
\indent 在一个绝热系统中,$\delta Q=0$,于是根据\tbf{3A.3.2}就有
\[\di S\geqslant 0\]
\indent 这表明绝热系统中的任意变化都不会使系统的熵减小.隔离系统当然也是绝热系统,于是上述结论可以表述为以下原理.
\begin{theorem}[3A.3.3 熵增加原理]
    绝热系统(或隔离系统)中的任意过程都不会使熵减小.特别地,如果这一过程是不可逆过程,那么系统的熵将增大.
\end{theorem}
熵增加原理也可以表述为:绝热系统中的任意自发变化都朝着熵增大的方向进行.\\
\indent 我们知道,一个自发过程是系统从非平衡态趋向平衡态的过程.%
倘若系统处于平衡态,那么其中发生的过程不可能为自发过程(否则这与系统达到平衡态相悖),也不可能是非自发过程(否则,将该过程反过来即为自发过程).%
这表明处于平衡态的系统中只能发生可逆过程.\\
\indent 综合上述论述,我们可以得出以下结论.
\begin{theorem}[3A.3.4 绝热系统熵最大的状态]
    处于平衡态的绝热系统的熵最大.
\end{theorem}
这是很好理解的,因为如果系统不处于平衡态,那么就会发生某些自发过程趋近平衡态并且使熵增大;%
直至达到平衡态时,不能再发生自发或非自发变化,只能发生可逆变化,熵也就不再改变了.\vspace{12pt}\\
\Section{3A.4 熵和能量退降}
\indent 我们似乎可以从Carnot定理发现这样一件事实,即能量是有区别的,同样的热在高温热源和低温热源上表现出的做功的能力是不同的.%
我们可以对这一过程进行具体地推导.
\begin{derivation}
    考虑三个热源$A,B,C$,其中$T_A>T_B>T_C$.\\
    考虑工作于$A,C$之间的Carnot热机,其热机效率为$\eta_{AC}=1-\dfrac{T_C}{T_A}$.\\
    考虑工作于$B,C$之间的Carnot热机,其热机效率为$\eta_{BC}=1-\dfrac{T_C}{T_B}$.\\
    由于$T_A>T_B$,于是$\eta_{AC}>\eta_{BC}$.这表明同样从$A,B$吸热$Q$,所能做的功是不等的,温度更高的$A$中取出的热的做功能力就更强.我们有
    \[W_{AC}-W_{BC}=Q\left(\eta_{AC}-\eta_{BC}\right)=T_C\left(\dfrac{Q}{T_B}-\dfrac{Q}{T_A}\right)\]
    考虑从$A$向$B$直接传热的过程,这显然是一个不可逆过程(否则将违背Clausius表述),于是这一过程的熵变
    \[\Delta S=\dfrac{Q}{T_B}-\dfrac{Q}{T_A}>0\]
    这就再一次验证了$W_{AC}>W_{BC}$,即高温热源的热做功能力更强.
\end{derivation}
这就表明,系统熵的增加会导致其中一部分能量失去做功的能力.尽管能量是守恒的,但是功可以无条件地转化为热,热却不能无条件地转化为功,必须要将一部分热传递给低温热源.
这说明功的“等级”是高于热的.同样,高温热源的热做功的能力就比低温热源要强,因此高温热源的热的“等级”也是高于低温热源的.\\
\indent 因此,不同“等级”的能量的转化是不可逆的,尽管能量总体是守恒的,但是其做功的能力却会越来越低.这就是我们说的\tbf{能量退降}.
\end{document}