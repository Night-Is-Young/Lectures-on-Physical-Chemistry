\documentclass{ctexart}
\usepackage{PhysicalChemistryNote}

\begin{document}\pagestyle{plain}
\noindent\tbf{\LARGE 3B 熵的统计学意义\footnote{本节关于熵的统计学意义的解释是十分粗浅的,而在你学过统计热力学基础后会对这一内容有更加深刻的理解.}}\vspace{15pt}\\
\indent “如果下次你的家长指责你的房间太过混乱,你可以微微一笑,并指出你的衣服和书的混乱摆放是熵增的,因而是自发的.%
你在帮宇宙执行它的规律.”\vspace{12pt}\\
\Section{3B.1 熵与热力学概率的关系——Boltzmann公式}
\indent 回忆我们在\tbf{3A.1}中提到的网球,它经由一个不可逆过程将机械能(也就是功)转化为了热.%
我们在\tbf{2A.3}中提到了功与热的统计学意义,那么这一过程就是大量质点由有序运动传递的能量%
变为大量质点无需运动传递的能量.\\
\indent 我们再考虑一个经典的不可逆过程:气体在绝热容器中向真空膨胀.显然,气体不会自己坚决地待在容器的某一个角落,而是会自发地%
,均匀地布满容器.或者,更严谨地说,气体分子在宏观上均匀地处在容器中的概率最大.\\
\indent 显然,系统中的各分子处于一个无序运动的状态的概率要比它们同时做有序运动的概率大得多,%
系统中的各分子均布的概率也比它们同时处在系统中的某一部分的概率大.这似乎告诉我们熵与概率有一定的关系.为此,我们先定义热力学概率.
\begin{definition}[3B.1.1 热力学概率]
    系统处于某宏观状态时对应的微观状态数即为系统处于此宏观状态的\tbf{热力学概率}.
\end{definition}
我们做一个简单的说明.假定一个容器中的分子有两种微观状态,即在左半部分和右半部分,处于这两种微观状态的概率是等同的.%
容器中的$2N$个分子在左边有$n$个,在右边有$2N-n$个的状态数为%
$\text{C}_{2N}^{n}$\footnote{这是组合数公式,$\text{C}_n^m=\frac{m!(n-m)!}{n!}$.},%
即处于此状态的热力学概率为$\text C_{2N}^n$,当$n=N$时取到最大,即该系统的分子在左右均匀分布的热力学概率最大.\\
\indent 我们采取如下方式简单地推导熵与热力学概率的关系.
\begin{derivation}
    考虑一个体积为$V$的绝热容器,其中的气体分子数目为$N$,将其可逆膨胀至体积为$V'$.\\
    记膨胀前后气体的状态分别为$A$和$B$.\\
    膨胀后在原先的体积为$V$的部分中找到一个气体分子的概率为$\dfrac{V}{V'}$,于是所有$N$个分子处于原先的部分的概率为
    \[\left(\dfrac{V}{V'}\right)^{N}\]
    因此,$A$与$B$对应的微观状态数$W_A$和$W_B$之比为
    \[\dfrac{W_A}{W_B}=\left(\dfrac{V}{V'}\right)^{N}\]
    两边取对数就有
    \[\ln\dfrac{W_A}{W_B}=N\ln\dfrac{V}{V'}\]
    由于系统绝热,于是根据热力学第一定律有
    \[\delta Q=-\delta W=-p\di V=-\dfrac{nRT}{V}\di V\]
    由于过程可逆,于是将$\di S=\dfrac{\delta Q}{T}$代入可得
    \[\di S=-nR\dfrac{\di V}{V}\]
    两端积分可得
    \[\Delta S=S_B-S_A=nR\ln\dfrac{V'}{V}\]
    于是
    \[S_B-S_A=nR\ln\dfrac{V'}{V}=Nk_\text B\ln\dfrac{V'}{V}=k_\text B\ln\dfrac{W_B}{W_A}=k_\text B\ln W_B-k_\text B\ln W_A\]
    于是就可以定义
    \[S=k_\text B\ln W\]

\end{derivation}
\begin{theorem}[3B.1.1 Boltzmann公式]
    熵$S$与热力学概率$W$满足
    \[S=k_\text B\ln W\]
    其中$k_\text B$为Boltzmann常数.\\
    Boltzmann公式说明一切熵增的过程都是热力学概率由小变大的过程.
\end{theorem}
从统计学角度来说,微观状态数越多表明宏观状态越无序.例如,所有分子向同一方向运动显然比所有分子向各个方向运动更加有序,%
而前者对应的微观状态数也更少.因此,结合熵增加原理(\tbf{3A.3.3})可知孤立系统总是朝着混乱程度变大的方向变化.\\
\indent 于是你可以向你的家长解释你的房间自发地变得混乱的原因.一切都是熵增加原理导致的(笑).
\begin{hint}
    需要说明的是,在Boltamann公式中的热力学概率并不是在经典力学下的微观状态数.%
    由于经典力学框架下能量的分布等是连续的,而此处的微观状态数是定域子系统Boltzmann分布的状态数,它假定分子的状态只有特定的一些取值.%
    这需要你学习统计热力学基础后才能有更深刻的认识.
\end{hint}\vspace{8pt}
\Section{3B.2 晶体的对称性与熵}
\indent 你也许时常会见到晶体的对称性和熵呈现一种复杂的关系.%
我们在这里进行简单的讨论.\\
\indent 对于某些晶体,例如\ce{C2H6},在高温下为立方晶系,低温下则为单斜晶系.%
这是由于高温下,每个\ce{C2H6}分子有数种可能的取向,在晶体中统计地分布,不能分辨,%
从而使得各\ce{C2H6}在宏观上等价,因而晶体对称性高;%
而单斜晶系中的乙烷则有确定的取向.%
这样,立方晶系这一宏观状态对应的微观状态数就多,而单斜晶系这一宏观状态对应的微观状态数就少.%
于是前者尽管对称性高,熵却更大.\\
\indent 对于某些晶体,例如\ce{KrF2},在高温下为简单四方点阵,最高次轴为$4_2$螺旋轴;%
在低温下为体心四方点阵,最高次轴为$C_4$轴.这是由于低温时\ce{KeF2}作为直线型分子排列地更为紧密,%
分子能振动的范围更小,因而相比于高温时的熵更小,对称性也更高.
\end{document}