\documentclass{ctexart}
\usepackage{PhysicalChemistryNote}

\begin{document}\pagestyle{plain}
\noindent\tbf{\LARGE 3F 热力学基本方程}\vspace{15pt}\\
我们已经学过了热力学的所有基本定律,现在是时候将它们结合起来了.%
通过一些数学推导,你将获得很多非常有用的结论.\vspace{12pt}\\
\Section{3F.1 热力学基本方程}
\indent 对于只做体积功的封闭系统有
\[\delta W=-p_e\di V\]
根据热力学第二定律,系统发生可逆变化时有
\[\delta Q=T\di S\]
假定系统的体积功是在准静态过程中做的(这也符合可逆变化的要求),就有$p=p_e$.%
代入热力学第一定律的微分形式,就可以得到热力学的基本方程
\[\di U=T\di S-p\di V\]
尽管我们推出这个方程是在过程可逆的基础上而得,不过由于$U,T,S,p,V$均为状态函数,%
因而这个式子在任何情况下都成立,但只有在可逆的情况下才有$\delta W=-p\di V$和$\delta Q=T\di S$.\\
\indent 同样地,对于$H,A,G$进行相似的操作可得四个热力学基本方程.
\begin{theorem}[3F.1.1 热力学基本方程]
    对于组成不变且只做体积功的封闭系统有
    \[\di U=T\di S-p\di V\]
    \[\di H=T\di S+V\di p\]
    \[\di A=-S\di T-p\di V\]
    \[\di G=-S\di T+V\di p\]

\end{theorem}\vspace{8pt}
\Section{3F.2 Maxwell关系}
\Part{Maxwell关系}
\indent 我们首先讨论由简单的流体系统.假定系统只做体积功,那么根据热力学基本方程有
\[\di U=T\di S-p\di V\]
这个方程可以看作内能$U$以熵$S$和体积$V$为独立变量的全微分表达式.根据全微分的性质,可以得出
\[\pa TVS=-\pa pSV\]
这个关系被称为Maxwell关系,它是$U$的全微分的数学性质的体现.
\begin{hint}
    如果一个关于$x,y$的函数$z=f(x,y)$有全微分
    \[\di z=\pa zxy\di x+\pa zyx\di y\]
    那么令
    \[A(x,y)=\pa zxy\ \ \ \ \ B(x,y)=\pa zyx\]
    对$A,B$分别微分就有
    \[\pa Ayx=\dfrac{\p^2 z}{\p y\p x}=\dfrac{\p^2 z}{\p x\p y}=\pa Bxy\]
    其中中间的等号正是全微分性质的体现.于是就有
    \[\pa Ayx=\pa Bxy\]

\end{hint}
类似地,我们还可以基于\tbf{3F.1.1}中其余式子得到对应的Maxwell关系,即
\begin{theorem}[3F.2.1 Maxwell关系]
    对于组成不变且只做体积功的封闭系统有
    \[\pa TVS=-\pa pSV\ \ \ \ \ \pa TpS=\pa VSp\]
    \[\pa pTV=\pa SVT\ \ \ \ \ \pa VTp=-\pa SpT\]

\end{theorem}
Maxwell关系的重要意义在于可以把一些难以测定的量(尤其是有关熵的偏微分)转化为容易测定的量%
(例如与状态方程,热容等物理量相关的偏微分),从而起到重要作用.\vspace{4pt}\\
\Part{Maxwell关系的应用}
\indent 本节简单举一些应用Maxwell关系的例子以介绍其应用.
\begin{problem}[3F.2.2]
    将$U$写为关于$T,V$的微分方程.
\end{problem}
\begin{solution}
    有关$U$的热力学基本方程为
    \begin{equation}
        \di U=T\di S-p\di V
    \end{equation}
    熵$S$关于$T,V$的全微分
    \begin{equation}
        \di S=\pa STV\di T+\pa SVT\di V
    \end{equation}
    将(2)代入(1)有
    \begin{equation}
        \di U=T\pa STV\di T+\left[T\pa SVT-p\right]\di V
    \end{equation}
    根据(3)就可以得到定容热容
    \begin{equation}
        C_V=\pa UTV=T\pa STV
    \end{equation}
    将\tbf{3F.2.1}代入(3),即可得等温情况下热力学能$U$随体积$V$的变化关系
    \begin{equation}
        \pa UVT=\pa SVT-p=T\pa pTV-p
    \end{equation}
    特别地,对于理想气体来说有$pV=nRT$,代入(5)中有
    \begin{equation}
        \pa UVT=T\pa{\left(\frac{nRT}{V}\right)}TV-p=T\cdot\dfrac{nR}{V}-p=0
    \end{equation}
    这也印证了\tbf{2C.1}中说明的理想气体的热力学能不随体积变化而变化.\\
    对于其余实际气体,可以通过对应状态方程求算(5)而得到相应的结果.
\end{solution}
\begin{problem}[3F.2.3]
    将$H$写为关于$T,p$的微分方程.
\end{problem}
\begin{solution}\setcounter{equation}{0}
    采用与\tbf{3F.2.2}中一样的变换,可以得到
    \begin{equation}
        \di H=T\pa STp\di T+\left[T\pa SpT+V\right]\di p
    \end{equation}
    于是等温情况下焓$H$随压强$p$的变化关系
    \begin{equation}
        \pa HpT=T\pa SpT+V=V-T\pa VTp
    \end{equation}
    对于理想气体来说有$pV=nRT$,于是
    \begin{equation}
        \pa HpT=V-T\pa{\left(\frac{nRT}{p}\right)}Tp=V-T\cdot\dfrac{nR}{p}=0
    \end{equation}
    于是理想气体的焓不随体积变化而变化.
\end{solution}
\begin{problem}[3F.2.4]
    求$S$随$p$或$V$的变化关系.
\end{problem}
\begin{solution}\setcounter{equation}{0}
    由\tbf{3F.2.1}有
    \begin{equation}
        \pa SpT=-\pa VTp
    \end{equation}
    定义
    \begin{equation}
        \alpha=\dfrac1V\pa VTp
    \end{equation}
    则等温时有
    \begin{equation}
        \di S=\pa SpT\di p+\pa STp\di T=\pa SpT\di p
    \end{equation}
    代入(1)(2)并积分可得
    \begin{equation}
        \Delta S=-\int_{p_1}^{p_2}\alpha V\di p
    \end{equation}
    又由\tbf{3F.2.1}可得
    \begin{equation}
        \pa SVT=\pa pTV
    \end{equation}
    定义
    \begin{equation}
        \beta=\dfrac1p\pa pTV
    \end{equation}
    则等温时有
    \begin{equation}
        \di S=\pa SVT\di V+\pa STV\di T=\pa SVT\di V
    \end{equation}
    代入(5)(6)并积分可得
    \begin{equation}
        \Delta S=\int_{V_1}^{V_2}p\beta\di V
    \end{equation}

\end{solution}
我们在推导中用到了两个参量$\alpha$和$\beta$.它们有明确的定义和物理含义.
\begin{definition}[3F.2.5 等压热膨胀系数,等温压缩系数,等体压强系数]
    定义等压热膨胀系数
    \[\alpha=\dfrac1V\pa VTp\]
    定义等温压缩系数
    \[\kappa=-\dfrac1V\pa VpT\]
    定义等体压强系数
    \[\beta=\dfrac1p\pa pTV\]

\end{definition}
\begin{problem}[3F.2.6]
    求$C_p$与$C_v$之差.
\end{problem}
\begin{solution}
    由\tbf{3F.2.2}和\tbf{3F.2.3}可知
    \[C_V=T\pa STV\ \ \ \ \ C_p=T\pa STp\]
    将$S(T,V)$改写为$S(T,V(T,p))$,则有
    \[\pa STp=\pa STV+\pa SVT\pa VTp\]
    于是
    \[C_p-C_V=T\pa SVT\pa TVp=T\pa pTV\pa TVp\]
    对于理想气体有$pV=nRT$,于是
    \[C_p-C_V=T\pa{\left(\frac{nRT}{V}\right)}TV\pa{\left(\frac{pV}{RT}\right)}Vp
    =T\cdot\dfrac{nR}{V}\cdot\dfrac{p}{RT}=nR\]
    这与我们在\tbf{2C.1.2}中推出的结论一致.\\
    或者可以用$\alpha$和$\kappa$表示$C_p-C_V$,即
    \[C_p-C_V=\dfrac{VR\alpha^2}{\kappa}\]

\end{solution}
最后,作为本节的结末,我们来讨论Gibbs自由能$G$随温度$T$的变化关系.
\begin{derivation}
    根据热力学基本公式有
    \[\di G=-S\di T+V\di p\ \ \ \ \ \pa GTp=-S\]
    又$G=H-TS$,于是
    \[\pa GTp=\dfrac{G-H}{T}\]
    对上式两边同除$T$并移项可得
    \[\dfrac1T\pa GTp-\dfrac{G}{T^2}=-\dfrac{H}{T^2}\]
    注意到
    \[\left[\dfrac{\p}{\p T}\left(\dfrac{G}{T}\right)\right]_p=\dfrac1T\pa GTp-\dfrac{G}{T^2}\]
    于是
    \[\left[\dfrac{\p}{\p T}\left(\dfrac{G}{T}\right)\right]_p=-\dfrac{H}{T^2}\]
    保持压强恒等,对上式移项积分可得
    \[\dfrac{G}{T}=-\int\dfrac{H}{T^2}\di T+I\]
    其中$I$是积分常数.\\
    如果还知道体系的$C_p$,就有
    \[H=H_0+\int C_p\di T\]
    假定$C_p$作为$T$的函数可以写作$C_p=a+bT^2+cT^3+\cdots$,则有
    \[H=H_0+aT+\dfrac12bT^2+\dfrac13cT^3+\cdots\]
    最终代入上式可得
    \[G=H_0-aT\ln T-\dfrac12bT^2-\dfrac16cT^3-\cdots+I\]

\end{derivation}
这就是Gibbs自由能关于温度的一个重要的定理.
\begin{theorem}[3F.2.7 Gibbs$-$Helmholtz方程]
    对于组成不变且只做非体积功的系统有
    \[\left[\dfrac{\p}{\p T}\left(\dfrac{G}{T}\right)\right]_p=-\dfrac{H}{T^2}\]
    或其积分形式
    \[\dfrac{G}{T}=-\int\dfrac{H}{T^2}\di T+I\]

\end{theorem}
这在我们研究化学热力学的时候非常有用.\vspace{12pt}\\
\Section{3F.3 特性函数}
\indent 本节我们将讨论如何从一些基本的热力学性质(例如物态方程,热容等)出发%
确定一个系统的重要的热力学函数(热力学能,熵等).我们先以焓$H$和熵$S$为例.
\begin{derivation}\setcounter{equation}{0}
    假定我们有系统的如下物态方程
    \begin{equation}
        V=V(T,p)
    \end{equation}
    由\tbf{3F.2.3}(1)可得
    \begin{equation}
        \di H=C_p\di T+\left[V-T\pa VTp\right]\di p
    \end{equation}
    对上式进行第二型曲线积分就有
    \begin{equation}
        H(p,T)=H_0+\int_{\left(p_0,T_0\right)}^{(p,T)}\left\{C_p\di T+\left[V-T\pa VTp\right]\di p\right\}
    \end{equation}
    由于$H$是状态函数,因此上式实际上与积分路径无关.我们可以任意采取一种路径求算$H$的值(例如先等压变温至$T$,再等温压缩或膨胀至$p$).%
    只要规定某一基准点的焓和知道物态方程(1),就可以知道任意状态下焓的值.\\
    事实上对于(3)式,我们并不需要知道所有压强下的定压热容,只需要知道某一压强下的定压热容即可.这是由于
    \begin{equation}
        \begin{aligned}
            \pa{C_p}pT
            &= T\dfrac{\p}{\p p}\left[\pa STp\right]_T=-T\dfrac{\p}{\p T}\left[\pa SpT\right]_p \\
            &= -T\dfrac{\p}{\p T}\left[\pa VTp\right]=-T\left(\dfrac{\p^2 V}{\p T^2}\right)_p
        \end{aligned}
    \end{equation}
    其中第二步交换了求偏导的顺序,第三步代入了Maxwell关系式.\\
    于是对(4)积分就有
    \begin{equation}
        C_p(p,T)=C_p\left(p_0,T\right)-T\int_{p_0}^p\left(\dfrac{\p^2 V}{\p T^2}\right)_p\di p
    \end{equation}
    将(5)代入(3)中即可完全由物态方程和某一状态的等压热容确定$H$.\\
    将熵$S$作为$p,T$的函数,对其全微分有
    \begin{equation}
        \di S=\pa SpT\di p+\pa STp\di T
    \end{equation}
    代入Maxwell关系和热容的定义可得
    \begin{equation}
        \di S=\dfrac{C_p}{T}\di T-\pa VTp\di p
    \end{equation}
    仿照与(3)相同的步骤即可得到$S$关于$p,T$的表达式.\\
    这样,我们就由物态方程$V=V(T,p)$和某一状态$p_0,T_0$时的定压热容$C_{p_0}$得到任意状态下的焓和熵.\\
    对于其余热力学函数,过程也是类似的.
\end{derivation}
1869间,Massieu证明如果独立变量选取地适当,那么只要知道一个热力学函数就可以完全确定%
一个均匀且平衡的系统的热力学性质.这样的函数函数称为特性函数.
\begin{definition}[3F.3.1 特性函数]
    如果某个状态函数的独立变量选取适当,从而能通过此函数的偏微分求得系统的其余状态函数,%
    进而确定系统的性质,就把这样的状态函数(和与之对应的独立变量)称为\tbf{特性函数}.
\end{definition}
我们以内能$U$为例.根据热力学基本方程\tbf{3F.1.1}有
\[\di U=T\di S-p\di V\]
将$S,V$作为$U$的独立变量,于是根据全微分的性质可得
\[T=\pa USV\ \ \ \ \ p=-\pa UVS\]
然后根据$H,G,A$与$U$的关系就可以得到
\[H=U+pV=U-V\pa UVS\]
\[G=H-TS=U-V\pa UVS-S\pa USV\]
\[A=U-TS=U-S\pa USV\]
进而确定系统的性质.可以证明,如果选取其它的状态函数作为独立变量(例如$p,V$)就不能将%
所有状态函数表示为偏微分的形式,进而不能确定系统的性质.\\
\indent 对于$U,H,A,G$作为特性函数的独立变量的选取,实际上只需关注其对应的热力学基本方程%
的全微分的对应变量即可.例如,$U$的热力学基本方程是对$S,V$微分,这就是$U$对应的独立变量;%
$G$的热力学基本方程是对$T,p$微分,这就是$G$对应的独立变量.
\end{document}