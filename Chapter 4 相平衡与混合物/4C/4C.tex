\documentclass{ctexart}
\usepackage{PhysicalChemistryNote}

\begin{document}\pagestyle{plain}
\noindent\tbf{\LARGE 4C 溶液}\vspace{15pt}\\
\indent 在系统地探讨了化学势和气体混合物中各组分的化学势后,我们将把目光放在一种更加复杂的体系——溶液上.%
研究溶液,事实上包括系统中液相和气相的相平衡问题,因此这一节也将作为联系\tbf{4B}和\tbf{4D}的桥梁.%
事实上,由于液相反应在化学中也十分广泛,因此研究溶液的热力学性质也是很有必要的.\vspace{12pt}\\
\Section{4C.1 理想溶液和理想稀溶液}
\Part{理想溶液与Raoult定律}
\indent 我们从考虑多组分系统的液相与气相之间的相平衡开始.
\begin{derivation}
    假定多组分系统中有气液两相,假定组分$i$的蒸气为理想气体.\\
    首先考虑纯的物质$i$的相平衡,根据\tbf{4B.4.1},其气相的化学势为
    \[\mu_{i,\g}^\ast=\mu_{i,\g}^\ominus(T)+RT\ln\dfrac{p_i^\ast}{p^\ominus}\]
    其中$p_i^\ast$即为纯物质$i$在此温度下的饱和蒸气压.又根据\tbf{4B.3.4},相平衡时有
    \[\mu_{i,\l}^\ast=\mu_{i,\g}^\ast\]
    现在考虑多组分系统,假定组分$i$的蒸气压变为$p_i$,在液相的化学势变为$\mu_{i,\l}$,则有
    \[\mu_{i,\l}=\mu_{i,\g}=\mu_{i,\g}^\ominus(T)+RT\ln\dfrac{p_i}{p^\ominus}\]
    联立上面三个式子可得
    \[\mu_{i,\l}=\mu_{i,\l}^\ast+RT\ln\dfrac{p_i}{p_i^\ast}\]
    
\end{derivation}
以上是我们基于气相是理想气体的假设所导出的结论.
\begin{hint}
    如无额外说明,我们都假定研究的理想溶液或理想稀溶液的蒸气是理想气体.\\
    事实上,理想溶液和理想稀溶液并不要求气相是理想气体,相应地在相平衡时应当用逸度代替压力,%
    并且仍需要运用Lewis-Randall定律进行进一步的简化.\\
    由于这样会使得推导复杂化,因此我们做如上的假设,以得到更简洁的结果.
\end{hint}
显然,式中的最后一项$\dfrac{p_i}{p_i^\ast}$与溶液的组成有关.%
1887年,在对一系列性质相近的液体混合物的蒸气压研究后,Francois Raoult发表了如下重要的实验事实.
\begin{theorem}[4C.1.1 Raoult定律]
    温度一定时,溶液中组分的蒸气压与该组分的纯物质的蒸气压之比等于该组分在液相中的摩尔分数,即
    \[p_i=p_i^\ast x_i\]

\end{theorem}
我们把Raoult定律代入上述推导中,就能得到理想溶液的定义.
\begin{definition}[4C.1.2 理想溶液]
    如果溶液中的任意组分$i$都满足
    \[\mu_{i,\l}=\mu_{i,\l}^\ast+RT\ln x_i\]
    则称该溶液为\tbf{理想溶液}.换言之,任意组分都满足Raoult定律的溶液为理想溶液.
\end{definition}
需要注意的是,上述定义意味着Raoult定律的成立,而非来源于Raoult定律,你需要明确两者的因果关系.%
这一定义也比用Raoult定律作为理想溶液的定义更好,因为它事实上并不依赖于理想气体模型.\\
\indent Raoult定律的微观解释有两种:
\begin{enumerate}[label=\tbf{(\arabic*)},leftmargin=41pt]
    \item 从蒸发的微观过程考虑,理想溶液的组分之间相互作用的差异可以不计,%
        而形成液体混合物将导致对于每个组分来说,其单位体积和单位表面积内的分子数目都将小于纯物质时的分子数目,%
        因而单位时间内离开溶液的分子数目相较纯液体将减少,使得该组分在更低的蒸气压下就能达成气液平衡.
    \item 从熵的角度考虑,蒸发是熵增过程,而溶液的熵相较纯液体应当更大,于是就有更小的趋势通过蒸发使系统获取最高的熵,%
        因而各组分的蒸气压都将下降.
\end{enumerate}

\indent 尽管实际情况下大部分溶液都对Raoult定律有一定偏离,但是在溶质极稀的情况下,%
溶剂总是满足Raoult定律的(对\tbf{4C.1.1}中取$x_i\to1$即可得知),因此Raoult定律也可以视作一种极限状态下的定律%
(就和理想气体状态方程是所有气体在$p\to0$时的极限情况那样).不过,由于离子的库仑作用,强电解质的溶液%
即使在很稀的情况下也与Raoult定律有所偏离.\vspace{4pt}\\
\Part{理想稀溶液与Henry定律}
\indent 对于低浓度的实际溶液,尽管溶剂是满足Raoult定律的,但William Henry却发现溶质的蒸气压与其摩尔分数虽然也成正比,%
但比例系数并非纯溶质的饱和蒸气压.总结实验数据之后,他提出了以下定律%
\footnote{尽管我们叙述上的逻辑如此,但事实上Henry定律的提出是在1803年,比Raoult定律早84年.}.
\begin{theorem}[4C.1.3 Henry定律]
    温度一定时,气体$i$在溶液中的摩尔分数(也就是气体的溶解度)与该气体的平衡分压成正比,比例系数记为\tbf{Henry常数}$k_i$,即
    \[p_i=k_ix_i\]

\end{theorem}
以及,我们可以根据前面的两条经验定律给出理想稀溶液的定义.
\begin{definition}[4C.1.4 理想稀溶液]
    溶剂满足Raoult定律,溶质满足Henry定律的溶液称为理想稀溶液.
\end{definition}
对于溶剂和溶质的行为差异,我们给出一种定性的微观解释:对于理想稀溶液,溶剂分子附近仍然是大量的溶剂分子,%
因此对溶剂分子的作用力实际上相当于纯溶剂液体并未改变多少,于是溶剂符合Raoult定律.%
与之相反,溶质分子周围几乎全部都是溶剂分子,这使得其周围的环境与纯溶质时完全不同,%
因此遵守Henry定律.如果溶剂分子和溶质分子足够相似,那么溶质就满足Raoult定律,%
此时的Henry常数$k$就是纯溶质的饱和蒸气压.\\
\indent 在实际应用中,Henry常数常常由以下方式给出.
\begin{derivation}
    对于稀溶液,设溶剂的物质的量为$n_\text s$\footnotemark,其余溶质的物质的量为$\li n,k$.对于溶质$i$,Henry定律可以化简如下
    \[p_i=k_ix_i=k_i\dfrac{n_i}{n_\text s+\sum_{j=1}^k n_j}\xlongequal{n_\text s\gg n_j}
    k_i\dfrac{n_i}{n_\text s}=k_i\dfrac{n_iM_\text s}{m_\text s}\]
    令$k_{m,i}=k_iM_s$,于是就有
    \[p_i=k_{m,i}m_i\]
    其中$m_i$为$i$的质量摩尔浓度.实际应用常常用此种单位的$k_{m,i}$来给出Henry常数.
\end{derivation}\footnotetext{下标s代表solvent,意为溶剂.}
应用Henry定律,需要注意两相中的物质的存在形式应当相同.例如,如果考虑$\text{CO}_2$的稀溶液,%
Henry定律给出的关系应为$\text{CO}_2$分子在溶液中的摩尔分数,而非各种存在形式(例如$\text{H}_2\text{CO}_3$等等)的和.\vspace{4pt}\\
\Part{理想稀溶液的溶质的化学势}
\indent 前面我们已经知道满足Raoult定律的组分的化学势(也即理想溶液中各组分的化学势).%
可以看出,Henry定律与Raoult定律具有十分相似的形式,那么理想稀溶液的溶质的化学势是否也具有类似的表达式呢?下面我们来尝试推导.
\begin{derivation}
    对于溶质$i$而言,在气相和液相平衡时有
    \[\mu_{i,\l}=\mu_{i,\g}=\mu_{i}^\ominus(T)+RT\ln\dfrac{p_i}{p^\ominus}\]
    根据Henry定律有
    \[p_i=k_{i}x_{i}\]
    代入上式有
    \[\mu_{i,\l}=\mu_{i}^\ominus(T)+RT\ln\dfrac{k_i}{p^\ominus}+RT\ln x_i=\mu_{i,\l}^\ast+RT\ln x_i\]
    需要注意的是,这里的$\mu_{i,\l}^\ast$指的是满足Henry定律的那个纯态.对于大部分稀溶液而言,这样的纯态实际上是不存在的,%
    因此这是一个假想的情形.在之后我们会证明,对于理想溶液,这一纯态就是满足Raoult定律的纯态.
\end{derivation}
\begin{theorem}[4C.1.5 理想稀溶液的溶质的化学势]
    理想稀溶液的溶质$i$的化学势
    \[\mu_{i,\l}=\mu_{i,\l}^\ast+RT\ln x_i\]
    其中$\mu_{i,\l}^\ast$指的是满足Henry定律的纯态,尽管在客观上它可能并不存在.
\end{theorem}
\vspace{8pt}
\Section{4C.2 理想溶液的性质}
\indent 基于\tbf{4C.1.2},我们可以推导出理想溶液的诸多热力学性质.
\begin{derivation}\setcounter{equation}{0}
    由化学势与温度,压强的关系(见\tbf{4B.3.3})有
    \begin{equation}\pa{\mu_i}{p}{T,n}=V_{\m,i}\end{equation}
    \begin{equation}\pa{\mu_i}{T}{p,n}=-S_{\m,i}\end{equation}
    \begin{equation}\left[\dfrac{\p}{\p T}\left(\dfrac{\mu_i}{T}\right)\right]_{p,n}=-\dfrac{H_{\m,i}}{T^2}\end{equation}
    以及理想溶液的定义
    \begin{equation}\mu_{i,\l}=\mu_{i,\l}^\ast+RT\ln x_i\end{equation}
    将(4)代入(1)有
    \begin{equation}
        V_{\m,i}=\pa{\mu_{i,\l}}{p}{T,n}=\pa{\mu_{i,\l}^\ast}{p}{T,n}=V_{\m,i}^\ast
    \end{equation}
    其中第二个等号是因为各组分的物质的量$n$一定时$x_i$是定值.\\
    于是理想溶液中各组分的偏摩尔体积等于纯的该组分的摩尔体积.应用偏摩尔量加和公式有
    \begin{equation}
        V=\sum_{i=1}^{k}V_{\m,i}n_i=\sum_{i=1}^{k}V_{\m,i}^\ast n_i
    \end{equation}
    于是理想溶液的体积就等于混合前的各组分的体积之和,即混合过程中没有体积变化.\\
    将(4)代入(2)有
    \begin{equation}
        -S_{\m,i}=\pa{\mu_{i,\l}}{T}{p,n}=\pa{\mu_{i,\l}^\ast}{T}{p,n}+R\ln x_i=-S_{\m,i}^\ast+R\ln x_i
    \end{equation}
    于是将各组分混合形成理想溶液的混合熵
    \begin{equation}
        \Delta S=\sum_{i=1}^{k}S_{\m,i}n_i-\sum_{i=1}^{k}S_{\m,i}^\ast n_i
        =-R\sum_{i=1}^{k}n_i\ln x_i=-nR\sum_{i=1}^{k}x_i\ln x_i
    \end{equation}
    其中$n$为总物质的量.可以看出,理想溶液的混合熵与理想气体混合物的混合熵具有完全一致的形式.\\
    将(4)代入(3)有
    \begin{equation}
        H_{\m,i}=-T^2\left[\dfrac{\p}{\p T}\left(\dfrac{\mu_i}{T}\right)\right]_{p,n}
        =-T^2\left[\dfrac{\p}{\p T}\left(\dfrac{\mu_i^\ast}{T}\right)\right]_{p,n}
        =H_{\m,i}^\ast
    \end{equation}
    与(6)同理有
    \begin{equation}
        H=\sum_{i=1}^{k}H_{\m,i}n_i=\sum_{i=1}^{k}H_{\m,i}^\ast n_i
    \end{equation}
    于是理想溶液的焓就等于纯的各组分的焓的和,混合过程的焓变$\Delta H$为零,不产生热效应.\\
    将Gibbs自由能的定义与(8)和(10)相结合可得
    \begin{equation}
        \Delta G=\Delta H-T\Delta S=RT\sum_{i=1}^{k}n_i\ln x_i=nRT\sum_{i=1}^{k}x_i\ln x_i
    \end{equation}
    可见混合过程的$\Delta G>0$,因而是自发过程.
\end{derivation}
于是我们就得到了理想溶液的一些热力学性质.
\begin{theorem}[4C.2.1 理想溶液的热力学性质]
    由纯的组分混合得到理想溶液的过程满足
    \[\Delta_{\text{mix}}V=\Delta_{\text{mix}}H=0\]
    \[\Delta_{\text{mix}}S=-R\ln\sum_{i=1}^{k}n_i\ln x_i\]
    \[\Delta_{\text{mix}}G=RT\ln\sum_{i=1}^{k}n_i\ln x_i\]

\end{theorem}
事实上,理想溶液并不仅仅满足Raoult定律,它其实也满足Henry定律.
\begin{theorem}[4C.2.2 理想溶液的Henry定律与Raoult定律]
    对于理想溶液,Henry定律和Raoult定律是等价的.
\end{theorem}
\begin{proof}\setcounter{equation}{0}
    考虑理想溶液的组分$i$在气相和液相的平衡.根据\tbf{4B.4.1}和\tbf{4C.1.2}以及相平衡条件可得
    \begin{equation}
        \mu_{i,\l}^\ast(T,p)+RT\ln x_i=\mu_{i,\g}^\ominus(T)+RT\ln\dfrac{p_i}{p^\ominus}
    \end{equation}
    移项后可得
    \begin{equation}
        \dfrac{p_i}{x_i}=p^\ominus\exp\left[\dfrac{\mu_{i,\l}^\ast(T,p)-\mu_{i,\g}^\ominus(T)}{RT}\right]
    \end{equation}
    在等温等压下,(2)的右边是定值,不妨记为$k_i$,于是就有
    \begin{equation}
        p_i=k_ix_i
    \end{equation}
    这就是Henry定律.令$x_i\to1$,就可以得到$k_i=p_i^\ast$,代入(3)中有
    \begin{equation}
        p_i=p_i^\ast x_i
    \end{equation}
    这就是Raoult定律.这就表明,理想溶液组分的Henry系数与纯的该组分的饱和蒸气压一致.
\end{proof}
\vspace{8pt}
\Section{4C.3 稀溶液的依数性}
\indent 大量实验表明,非挥发双组分稀溶液的某些性质仅与溶质的摩尔分数有关,而与溶液的组成物质无关.%
这些性质被称为稀溶液的\tbf{依数性}.\vspace{4pt}\\
\Part{凝固点降低}
\indent 我们知道向水中加入盐可以显著降低其凝固点,一些除雪剂就是根据这一原理工作的.%
我们可以从理想稀溶液的定义来定量地确定溶液中溶剂的凝固点降低的程度.\\
\indent 先从简单的情况入手,假定溶质和溶剂不形成固溶体,固相是纯的溶剂.
\begin{derivation}\setcounter{equation}{0}
    设在压力为$p$时,溶液的凝固点为$T_\f$,溶剂$A$的摩尔分数为$x_A$.此时固相和液相平衡,即
    \begin{equation}
        \mu_{A,\l}\left(T_\f,p,x_A\right)=\mu_{A,\s}(T,p)
    \end{equation}
    保持压强恒定,若使得溶液浓度变化$\di x_A$,则凝固点相应地变化$\di T$.%
    与Clapeyron方程的推导过程类似地,可知此时有
    \begin{equation}
        \di\mu_{A,\l}=\di\mu_{A,\s}
    \end{equation}
    将$\mu_{A,\l}$和$\mu_{A,\s}$的全微分代入(2),并令$\di p=0$,就有
    \begin{equation}\pa{\mu_{A,\l}}T{p,x_A}\di T+\pa{\mu_{A,\l}}{x_A}{T,p}=\pa{\mu_{A,\s}}{T}{p}\di T\end{equation}
    对于理想稀溶液有
    \begin{equation}\mu_{A,\l}=\mu_{A,\l}^\ast+RT\ln x_A\end{equation}
    又根据化学势与温度的关系
    \begin{equation}\pa{\mu_{\s}}{T}{p,x_A}=-S_{\m,A}\end{equation}
    将(4)(5)代入(3)就有
    \begin{equation}
        -S_{\m,A,\l}\di T+\dfrac{RT}{x_A}\di x_\s=-S_{\m,A,\s}^\ast\di T
    \end{equation}
    又因为
    \begin{equation}
        S_{\m,A,\l}-S_{\m,A,\s}^\ast
        =\dfrac{H_{\m,A,\l}-H_{\m,A,\s}^\ast}{T}
        =\dfrac{\Delta H_{\m,A}}{T}
    \end{equation}
    式中$\Delta H_{\m,A}$是在凝固点时单位量的固态溶剂$A$熔化进入溶液所吸收的热.%
    由于溶液是稀溶液,因此其性质与纯的液态$A$接近,$\Delta H_{\m,A}$也就近似地与纯的$A$的%
    摩尔熔化焓$\Delta_{\fus}H_{\m,A}^\ast$相等.%
    将这一近似关系和(7)一起代入(6)就有
    \begin{equation}\dfrac{RT}{x_A}\di x_A=\dfrac{\Delta_{\fus}H_{\m,A}^\ast}{T}\di T\end{equation}
    对于纯的$A$,即$x_A=1$,设此时的凝固点为$T_\f^\ast$.对(8)积分可得
    \begin{equation}\int_{1}^{x_A}\dfrac{\di x_A}{x_A}=\int_{T_\f^\ast}^{T_\f}\dfrac{\Delta_{\fus}H_{\m,A}^\ast}{RT^2}\di T\end{equation}
    由于溶液很稀,那么凝固点改变不大,于是再假设这一温度范围内$\Delta_{\fus}H_{\m,A}^\ast$为定值,于是由(9)可得
    \begin{equation}
        \ln x_A=\dfrac{\Delta_{\fus}H_{\m,A}^\ast}{R}\left(\dfrac{1}{T_\f^\ast}-\dfrac{1}{T_\f}\right)
    \end{equation}
    你可以发现(10)与Clausius$-$Clapeyron方程在形式上的一致.\\
    现在我们进一步简化该式.令$\Delta T_\f=T_\f^\ast-T_\f$,又因为$T_\f$与$T_\f^\ast$接近,于是$T_\f T_\f^\ast\sim\left(T_\f^\ast\right)^2$.%
    据此由(10)可得
    \begin{equation}
        -\ln x_A=\dfrac{\Delta_{\fus}H_{\m,A}^\ast}{R}\cdot\dfrac{T_\f^\ast-T_\f}{T_\f T_\f^\ast}
        =\dfrac{\Delta_{\fus}H_{\m,A}^\ast}{R\left(T_\f^\ast\right)^2}\cdot\Delta T_\f
    \end{equation}
    设溶质的物质的量为$n_B$,并考虑函数$\ln(1+x)$在$0$处的一阶Taylor展开,于是
    \begin{equation}
        -\ln x_A=-\ln\left(1-x_B\right)\sim x_B\sim\dfrac{n_B}{n_A}
    \end{equation}
    于是(11)式可以改写为
    \begin{equation}
        \Delta T_\f=\dfrac{R\left(T_\f^\ast\right)^2}{\Delta_{\fus}H_{\m,A}^\ast}\cdot\dfrac{n_B}{n_A}
    \end{equation}
    将$\dfrac{n_B}{n_A}$继续改写如下
    \begin{equation}
        \dfrac{n_B}{n_A}=\dfrac{n_B}{m(A)}\cdot M_A=M_A m_B
    \end{equation}
    其中$m_B$为溶质的质量摩尔分数,$M_A$为溶剂分子的摩尔质量.将(14)代入(13)就有
    \begin{equation}
        \Delta T_\f=\dfrac{R\left(T_\f^\ast\right)^2}{\Delta_{\fus}H_{\m,A}^\ast}\cdot M_A\cdot m_B=k_\f m_B
    \end{equation}
    其中$k_\f=\dfrac{R\left(T_\f^\ast\right)^2}{\Delta_{\fus}H_{\m,A}^\ast}\cdot M_A$.
\end{derivation}
现在,我们得到了一个简洁的式子,以近似地确定稀溶液凝固点下降的程度.
\begin{theorem}[4C.3.1 凝固点降低]
    理想稀溶液凝固点降低的大小(即$T_\f^\ast-T_\f$)为
    \[\Delta T_\f=k_\f m_B\]
    其中$m_B$为所有溶质的质量摩尔分数,$k_\f=\dfrac{R\left(T_\f^\ast\right)^2}{\Delta_{\fus}H_{\m,A}^\ast}\cdot M_A$为\tbf{凝固点降低常数},%
    其中$T_\f^\ast$为纯溶剂的凝固点,$\Delta_{\fus}H_{\m,A}^\ast$为纯溶剂的熔化焓(取正值),$M_A$为溶剂分子的摩尔质量.
\end{theorem}
可以看出,凝固点下降的程度与溶质的质量摩尔分数成正比.\\
\indent 现在,你可以尝试推导固相和液相都是溶液(我们一般称固相的溶液为固溶体)的情形.%
过程相较前面的推导也许会复杂一些,不过你应当已经有足够的知识以解决它.
\begin{derivation}\setcounter{equation}{0}
    现在假定压力为$p$时,溶液的凝固点为$T_\f$,溶剂$A$在液相和固相的摩尔分数分别为$x_{A,\l}$和$x_{A,\s}$.于是跟前面的推导类似地有
    \begin{equation}
        \di\mu_{A,\l}=\di\mu_{A,\s}
    \end{equation}
    代入化学势的全微分和理想稀溶液中溶剂的性质(这在固相和液相中都成立),就有
    \begin{equation}
        -S_{\m,A,l}\di T+\dfrac{RT}{x_{A,\l}}\di x_{A,\l}=-S_{\m,A,\s}\di T+\dfrac{RT}{x_{A,\s}}\di x_{A,\s}
    \end{equation}
    又因为
    \begin{equation}
        S_{\m,A,\l}-S_{\m,A,\s}
        =\dfrac{H_{\m,A,\l}-H_{\m,A,\s}}{T}
        =\dfrac{\Delta_{\fus}H_{\m,A}}{T}
    \end{equation}
    其中$\Delta_{\fus}H_{\m,A}$为此情况下单位物质的量$A$从固相进入液相的焓变.%
    将(3)代入(2)即可得
    \begin{equation}
        \dfrac{\di x_{A,\l}}{x_{A,\l}}-\dfrac{\di x_{A,\s}}{x_{A,\s}}
        =\dfrac{\Delta_{\fus}H_{\m,A}}{RT^2}\di T
    \end{equation}
    对上式从纯溶剂的情形(即$x_{A,\l}=x_{A,\s}=1$,$T=T_\f^\ast$)开始积分,可得
    \begin{equation}
        \int_{1}^{x_{A,\l}}\dfrac{\di x_{A,\l}}{x_{A,\l}}-
        \int_{1}^{x_{A,\s}}\dfrac{\di x_{A,\s}}{x_{A,\s}}
        =\int_{T_\f^\ast}^{T_\f}\dfrac{\Delta_{\fus}H_{\m,A}}{RT^2}\di T
    \end{equation}
    同样地,我们做出$\Delta_{\fus}H_{\m,A}$随温度变化不大的假设,就有
    \begin{equation}
        \ln\dfrac{x_{A,\l}}{x_{A,\s}}
        =\dfrac{\Delta_{\fus}H_{\m,A}}{R}\left(\dfrac{1}{T_\f^\ast}-\dfrac{1}{T_\f}\right)
    \end{equation}
    在对$T_\f$和$T_\f^\ast$做近似后就有
    \begin{equation}
        -\ln\dfrac{x_{A,\l}}{x_{A,\s}}
        =\dfrac{\Delta_{\fus}H_{\m,A}}{R\left(T_\f^\ast\right)^2}\cdot\Delta T_\f
    \end{equation}
    这就得到了两相均是理想稀溶液时的凝固点降低公式.我们也可以从上式推出凝固点下降或升高的情况.
\end{derivation}
\begin{theorem}[4C.3.2 固相为固溶体时的凝固点变化]
    液相和固相都是稀溶液时,凝固点变化可由
    \[-\ln\dfrac{x_{A,\l}}{x_{A,\s}}
    =\dfrac{\Delta_{\fus}H_{\m,A}}{R\left(T_\f^\ast\right)^2}\cdot\Delta T_\f\]
    近似地给出.\\
    如果$x_{A,\l}>x_{A,\s}$,即液相中$A$的摩尔分数更高,则$\Delta T_\f<0$,凝固点升高.\\
    如果$x_{A,\l}<x_{A,\s}$,即固相中$A$的摩尔分数更高,则$\Delta T_\f>0$,凝固点降低.
\end{theorem}
\Part{沸点升高}
\indent Raoult定律告诉我们,理想稀溶液的溶剂的蒸气压会降低,于是达到沸点的温度就应当升高.%
事实上,气液两相的平衡和液固两相的平衡具有诸多相似之处,%
采用与上一节完全相同的方法就可以得到理想稀溶液的沸点升高的程度.
\begin{theorem}[4C.3.3 沸点升高]
    非挥发性溶质的理想稀溶液沸点升高的大小(即$T_\b-T_\b^\ast$)为
    \[\Delta T_\b=k_\f m_B\]
    其中$m_B$为所有溶质的质量摩尔分数,$k_\b=\dfrac{R\left(T_\b^\ast\right)^2}{\Delta_{\vap}H_{\m,A}^\ast}\cdot M_A$为\tbf{沸点升高常数},%
    其中$T_\b^\ast$为纯溶剂的沸点,$\Delta_{\vap}H_{\m,A}^\ast$为纯溶剂的蒸发热,$M_A$为溶剂分子的摩尔质量.
\end{theorem}
同样地,如果溶质有挥发性,那么液相是稀溶液,气相则可以假定为理想气体混合物.%
与前面对固溶体-溶液平衡的推导类似地有
\begin{theorem}[4C.3.4 溶质和溶剂均可蒸发时的沸点变化]
    液相是理想稀溶液,气相为理想气体混合物时,沸点变化可由
    \[\ln\dfrac{x_{A,\g}}{x_{A,\l}}
    =\dfrac{\Delta_{\vap}H_{\m,A}}{R\left(T_\f^\ast\right)^2}\cdot\Delta T_\b\]
    近似地给出.\\
    如果$x_{A,\g}>x_{A,\s}$,即液相中$A$的摩尔分数更高,则$\Delta T_\b<0$,沸点降低.\\
    如果$x_{A,\g}<x_{A,\s}$,即气相中$A$的摩尔分数更高,则$\Delta T_\b>0$,沸点升高.
\end{theorem}
\Part{渗透压}
\indent 我们知道理想稀溶液中溶剂的化学势为$\mu_\s=\mu_\s^\ast+RT\ln x_\s$.在恒定温度下,%
$x_\s$越大(即越接近纯溶剂),化学势越大.这表明我们将同一溶剂形成的不同浓度的稀溶液用仅能通过溶剂分子的半透膜隔开时,%
溶剂分子总是向溶质浓度高的地方移动(这也符合我们对浓度不同的溶液混合情况的直觉).\\
\indent 如果想要阻止这一移动的过程,我们需要想办法提高更浓的一侧的$\mu_\s$.%
最简单的办法就是提高对溶液的压力,以阻止另一侧的溶剂向此处扩散.
\begin{definition}[4C.2.5 渗透压]
    为了阻止溶剂从(溶质)浓度低的一侧经由半透膜扩散至浓度高的一侧而向高浓度溶液施加的最小压力即\tbf{渗透压}.
\end{definition}
同样地,利用化学势,我们也可以推出渗透压的大小.
\begin{derivation}\setcounter{equation}{0}
    我们假定半透膜两侧溶剂$A$的摩尔分数为$x_{A,\l}^{l}$和$x_{A,\l}^{r}$(上标$l$和$r$分别表示左右),于是
    \begin{equation}\mu_{A,\l}^l(T,p)=\mu_{A,\l}^\ast(T,p)+RT\ln x_{A,\l}^l\end{equation}
    \begin{equation}\mu_{A,\l}^r(T,p)=\mu_{A,\l}^\ast(T,p)+RT\ln x_{A,\l}^r\end{equation}
    假定$x_{A,\l}^l<x_{A,\l}^r$,那么应当向右边的溶液施加压力$\Pi$使得两边化学势相等,即
    \begin{equation}\mu_{A,\l}^l(T,p)=\mu_{A,\l}^r(T,p+\Pi)\end{equation}
    考虑到恒定温度下化学势随压力的变化
    \begin{equation}\pa{\mu_\s}p{T,n}=V_{\m,A}\end{equation}
    于是
    \begin{equation}
        \begin{aligned}
            \mu_{A,\l}^r(T,p+\Pi)
            &= \mu_{A,\l}^\ast(T,p+\Pi)+RT\ln x_{A,\l}^r \\
            &= \mu_{A,\l}^\ast(T,p)+\int_{p}^{p+\Pi}V_{\m,A}\di p+RT\ln x_{A,\l}^r \\
            &= \mu_{A,\l}^\ast(T,p)+V_{\m,A}\Pi+RT\ln x_{A,\l}^r
        \end{aligned}
    \end{equation}
    将(5)和(1)代入(3)有
    \begin{equation}
        \mu_{A,\l}^\ast(T,p)+RT\ln x_{A,\l}^l=\mu_{A,\l}^\ast(T,p)+V_{\m,A}\Pi+RT\ln x_{A,\l}^r
    \end{equation}
    对(6)移项化简可得
    \begin{equation}
        RT\ln\dfrac{x_{A,\l}^l}{x_{A,\l}^r}=V_{\m,A}\Pi
    \end{equation}
    考虑到两边都是稀溶液,因此$x_{A,\l}^l,x_{A,\l}^r\to1$.假定两边溶质的摩尔分数分别为$x_{B}^l,x_B^r$,%
    利用近似$\ln(1+x)\sim x(x\to0)$就有
    \begin{equation}RT\left(x_{B}^r-x_{B}^l\right)=V_{\m,A}\Pi\end{equation}
    而稀溶液中有近似
    \begin{equation}x_B=\dfrac{n_B}{n_A+n_B}\sim\dfrac{n_B}{n_A}\end{equation}
    将(9)代入(8)并移项可得
    \begin{equation}RT\left(n_B^r-n_B^l\right)=n_AV_{\m,A}\Pi=V_{A}\Pi\end{equation}
    又因为稀溶液的体积可以近似地看作溶剂的体积,并且$c_B=\dfrac{n_B}{V}$,于是
    \begin{equation}\Pi=\left(c_B^r-c_B^l\right)RT\end{equation}
    或者可以写作
    \begin{equation}
        \Pi=\Delta cRT
    \end{equation}
    特别地,如果左侧是纯溶剂,就有$\Pi=cRT$.
\end{derivation}
这样,我们就得到了一个简洁的关于渗透压和溶质浓度的(近似的)等式,即
\begin{theorem}[4C.2.6 van$'$t Hoff 公式]
    理想稀溶液的渗透压$\Pi$满足
    \[\Pi=cRT\]
    其中$c$为溶质的浓度.
\end{theorem}
渗透压常常被用于测定大分子(例如蛋白质)的摩尔质量,%
只需知道质量摩尔浓度和渗透压即可根据$M=\dfrac{m}{c}$求算摩尔质量.%
不过,由于大分子溶液对van$'$t Hoff公式有明显偏差,因此我们可以对上式virial展开\footnote{多么令人熟悉的操作.},即
\[\Pi=cRT\left(1+Ac+Bc^2+\cdots\right)\]
一般取到$Ac$项即可,通过拟合数据得到virial系数$A$,就可以求得待测分子的摩尔质量.\vspace{12pt}\\
\Section{4D Duhem-Margule公式}
\indent 我们已经知道,偏摩尔量总是满足Gibbs-Duhem公式.在溶液中,我们可以对这一公式进行延伸,%
以得到溶液中各个状态函数的关系.
\begin{derivation}\setcounter{equation}{0}
    当系统中气相和液相平衡时,对任意组分$i$在两相中都有
    \begin{equation}\mu_{i,\l}=\mu_{i,\g}^\ominus+RT\ln\dfrac{p_i}{p^\ominus}\end{equation}
    其中$p_i$是$i$在蒸气中的分压.对上式微分即可得
    \begin{equation}\di\mu_{i,\l}=RT\dfrac{\di p_i}{p_i}\end{equation}
    现在考虑液相.根据偏摩尔量的加和公式的微分形式有
    \begin{equation}\di G_\l=\sum_{i=1}^k\left(n_{i,\l}\di\mu_{i,\l}+\mu_{i,\l}\di n_{i,\l}\right)\end{equation}
    又因为
    \begin{equation}\di G_\l=-S\di T+V_\l\di p+\sum_{i=1}^k\mu_{i,\l}\di n_{i,\l}\end{equation}
    保持温度恒定,比较上面两式就有
    \begin{equation}\sum_{i=1}^{k}n_{i,\l}\di\mu_{i,\l}=V_\l\di p\end{equation}
    将(2)代入(5)就有
    \begin{equation}RT\sum_{i=1}^kn_{i,\l}\dfrac{\di p_i}{p_i}={V_\l}\di p\end{equation}
    在(6)式两边同除以液相总物质的量,移项后可得
    \begin{equation}
        \sum_{i=1}^kx_{i,\l}\dfrac{\di p_i}{p_i}=\dfrac{V_\l\di p}{RT\displaystyle\sum_{i=1}^kn_{i,\l}}
        =\dfrac{V}{\displaystyle\sum_{i=1}^kn_{i,\l}}\cdot\dfrac{p}{RT}\cdot\dfrac{\di p}{p}
        =\dfrac{V_{\m,l}}{V_{\m,g}}\dfrac{\di p}{p}
    \end{equation}
    一般情况下,气相的摩尔体积总是远大于液相,即$V_{\m,g}\gg V_{\m,l}$.于是略去上式中右边的项即可得
    \begin{equation}\sum_{i=1}^kx_{i,\l}\dfrac{\di p_i}{p_i}=0\end{equation}
    现在,对于只含有$A$和$B$的双组分系统,由上式有
    \begin{equation}x_{A,\l}\dfrac{\di p_A}{p_A}+x_{B,\l}\dfrac{\di p_B}{p_B}=0\end{equation}
    在恒定温度和压力下,分压的改变仅与液相的组成有关,即
    \begin{equation}
        x_{A,\l}\pa{\ln p_A}{x_{A,\l}}{T,p}\di x_{A,\l}+x_{B,\l}\pa{\ln p_B}{x_{B,\l}}{T,p}\di x_{B,\l}=0
    \end{equation}
    又因为$\di x_{A,\l}+\di x_{B,\l}=0$,于是可以将(10)写为
    \begin{equation}
        x_{A,\l}\pa{\ln p_A}{x_{A,\l}}{T,p}=x_{B,\l}\pa{\ln p_B}{x_{B,\l}}{T,p}
    \end{equation}
    或者可以对(11)变形得到
    \begin{equation}
        \pa{\ln p_A}{\ln x_{A,\l}}{T,p}=\pa{\ln p_B}{\ln x_{B,\l}}{T,p}
    \end{equation}
    和
    \begin{equation}
        \dfrac{x_{A,\l}}{p_A}\pa{p_A}{x_{A,\l}}{T,p}=\dfrac{x_{B,\l}}{p_B}\pa{p_B}{x_{B,\l}}{T,p}
    \end{equation}

\end{derivation}
这样,我们就得到了具有气液两相的多组分系统的各组分分压与在液相中的含量应当遵守的关系.
\begin{theorem}[4C.4.1 Duhem-Margule公式]
    对于具有气液两相的多组分系统,如果把气相视作理想气体混合物,就有
    \[\sum_{i=1}^kx_{i}\di\ln p_i=0\]
    其中$x_i$为组分$i$在液相的摩尔分数,$p_i$为组分$i$在气相的分压.\\
    特别地,对于由$A$和$B$组成的双组分系统有
    \[\dfrac{x_{A}}{p_A}\pa{p_A}{x_{A}}{T,p}=\dfrac{x_{B}}{p_B}\pa{p_B}{x_{B}}{T,p}\]
    其中各符号的意义与前面相同.并且,这式子还有其等价形式(11)与(13).
\end{theorem}
通过Duhem-Margule公式,我们也可以得到一些有用的推论.
\begin{problem}[4C.4.2]
    试通过Duhem-Margule公式证明:如果双组分溶液的组分$A$满足Raoult定律,那么另一组分$B$一定满足Henry定律.
\end{problem}
\begin{proof}
    根据Raoult定律有$p_A=p_A^\ast x_A$,于是
    \[\dfrac{x_{A}}{p_A}\pa{p_A}{x_{A}}{T,p}=\dfrac{1}{p_A^\ast}\cdot p_{A}^\ast=1\]
    根据Duhem-Margule公式,就有
    \[\pa{\ln p_B}{\ln x_{B,\l}}{T,p}=1\]
    即
    \[\di\ln p_B=\di\ln x_B\]
    对上式两边积分可得
    \[\ln p_B=\ln x_B+C\]
    其中$C$为积分常数.再变形可得
    \[p_B=k_{x,B}x_{B}\]
    这就说明$B$满足Henry定律.
\end{proof}
以上这一点说明,如果稀溶液的溶剂满足Raoult定律,那么它就是理想稀溶液.
\begin{theorem}[4C.4.3 科诺瓦洛夫规则]
    在溶液的$p-x$图(或$T-x$图)上,如果有最高(或最低)点,则在该点时气液两相的组成相等.%
    如果增加溶液的蒸气中一个组分的含量,能使溶液的总蒸气压增加,则该组分在气相中的含量要比在液相中的含量大.%
\end{theorem}
这是由俄国人亚历山大$\cdot$科诺瓦洛夫总结出的经验规则.%
上述两条规则也被称作科诺瓦洛夫第一规则和科诺瓦洛夫第二规则.%
根据Duhem-Margule公式,我们可以在理论上对其进行证明.
\begin{proof}
    现在考虑由$A$和$B$组成的双组分系统.假定气相仅由$A$和$B$组成,就有
    \[p_A=px_{A,\g}\ \ \ \ \ p_B=px_{B,\g}=p\left(1-x_{A,\g}\right)\]根据Duhem-Margule公式有
    \[x_{A,\l}\dfrac{\di\left(px_{A,\g}\right)}{px_{A,\g}}
    +\left(1-x_{A,\l}\right)\dfrac{\di\left[p\left(1-x_{A,\g}\right)\right]}{p\left(1-x_{A,\g}\right)}
    =\dfrac{V_{\m,l}}{V_{\m,g}}\dfrac{\di p}{p}\]
    将微分项分离变量有
    \[x_{A,\l}\left(\dfrac{\di p}{p}+\dfrac{\di x_{A,\g}}{x_{A,\g}}\right)+
    \left(1-x_{A,\l}\right)\left(\dfrac{\di p}{p}-\dfrac{\di x_{A,\g}}{1-x_{A,\g}}\right)
    =\dfrac{V_{\m,l}}{V_{\m,g}}\dfrac{\di p}{p}\]
    移项可得
    \[\left(\dfrac{x_{A,\l}}{x_{A,\g}}-\dfrac{1-x_{A,\l}}{1-x_{A,\g}}\right)\di x_{A,\g}
    =\left(\dfrac{V_{\m,l}}{V_{\m,g}}-1\right)\dfrac{\di p}{p}\]
    即有
    \[\pa{\ln p}{x_{A,\g}}{T}
    =\dfrac{x_{A,\g}-x_{A,\l}}{x_{A,\g}\left(1-x_{A,\g}\right)\left(\dfrac{V_{\m,l}}{V_{\m,g}}-1\right)}\]
    由于$\dfrac{V_{\m,l}}{V_{\m,g}}\ll1$,于是上式可以写为
    \[\pa{\ln p}{x_{A,\g}}{T}
    =\dfrac{x_{A,\g}-x_{A,\l}}{x_{A,\g}\left(1-x_{A,\g}\right)}\]
    因此,如果增加$x_{A,g}$时$p$也增加,说明$\pa{\ln p}{x_{A,\g}}{T}>0$,即$x_{A,\g}>x_{A,\l}$.%
    反之如果增加$x_{A,g}$时$p$减小,说明$x_{A,\g}<x_{A,\l}$.\\
    如果$p-x$图像上出现最高点,那么此时就有$\pa{\ln p}{x_{A,\g}}{T}=0$,这就要求$x_{A,\g}=x_{A,\l}$,于是两相的成分相同.
\end{proof}
\vspace{8pt}
\Section{4C.5 分配定律与萃取}
\indent 前面我们已经比较系统地讨论了理想溶液和理想稀溶液中的性质,而溶剂总是只有一种(或者更严谨地说,溶剂只有一项).%
事实上我们知道,大多数溶质在各种不同的溶剂中都有一定的溶解性,有时这些溶剂也并不互溶.%
可以想见的是,溶质$i$应当在不互溶的溶剂$\s_\alpha$和$\s_\beta$中都占有一定的比例.而实验证实了这一比例是定值.
\begin{theorem}[4C.5.1 分配定律]
    在等温等压下,如果一个溶质$i$同时溶解在两个同时存在的不互溶的液相$\alpha$和$\beta$中,那么达到平衡之后,该物质在两相中的浓度之比为定值,即
    \[\dfrac{c_{i,\alpha}}{c_{i,\beta}}=K_i\]
    常数$K_i$称为\tbf{分配系数}.
\end{theorem}
同样地,分配定律也可以通过理想稀溶液的性质来证明.
\begin{proof}
    平衡后溶质$i$在$\alpha$和$\beta$两相中的化学势相等,即
    \[\mu_{i,\alpha}=\mu_{i,\beta}\]
    代入理想稀溶液中溶质的化学势就有
    \[\mu_{i,\alpha}^\ast+RT\ln x_{i,\alpha}=\mu_{i,\beta}^\ast+RT\ln x_{i,\beta}\]
    移项整理可得
    \[\dfrac{x_{i,\alpha}}{x_{i,\beta}}=\exp\left(-\dfrac{\mu_{i,\alpha}^\ast-\mu_{i,\beta}^\ast}{RT}\right)\]
    我们在推导稀溶液的渗透压时已经用过近似$x_i\sim\dfrac{n_i}{n_{\s}}=\dfrac{n_i}{V_\s}\cdot V_{\m,\s}=c_iV_{\m,\s}$,于是上式可变形为
    \[\dfrac{c_{i,\alpha}}{c_{i,\beta}}=\dfrac{V_{\m,\beta}}{V_{\m,\alpha}}\exp\left(-\dfrac{\mu_{i,\alpha}^\ast-\mu_{i,\beta}^\ast}{RT}\right)\]
    在恒定温度与压强下,右边的式子是常数,即我们所述的分配系数$K_i$.这就证明了分配定律.
\end{proof}
你可以看出,上式右边的指数部分与我们在普通化学中的平衡常数的表达式十分类似.%
我们将在\tbf{Chapter 5}中详细讨论化学平衡与分配平衡(乃至所有类型的相平衡)之间的联系.\\
\indent 需要注意的是,两相中的溶质的存在形式应当相同.如果溶质在某一相中会发生缔合或解离等等一系列改变形态的变化,%
那么这些变化后的形体都不应当计入分配平衡的等式中.这与我们讨论Henry定律的应用条件是一致的.\\
\indent 分配定律在化学中的一个重要的应用就是萃取.利用溶质在不同溶剂中的溶解度的差异,可以将溶质从一相转移到另一相,%
从而实现物质的分离.
\begin{definition}[4C.5.2 萃取]
    \tbf{萃取}是利用系统中组分在溶剂中有不同的溶解度来分离混合物的单元操作.
\end{definition}
利用分配定律,我们可以推算萃取操作的效率,以指导这一步骤的高效进行.
\begin{problem}[4C.5.3]
    设在体积为$V_\beta$的溶剂为$\beta$的溶液中含有溶质$i$的浓度为$c_{0}$.%
    用体积为$V_\alpha$的溶剂$\alpha$对其萃取.假定分配系数为$K=\dfrac{c_{\alpha}}{c_{\beta}}$,试计算萃取效率$\eta$(即被$\alpha$带走的$i$的占总体的比例).%
    假定两相的体积不发生变化.
\end{problem}
\begin{solution}
    平衡时有
    \[\left\{\begin{array}{l}
        c_0V_\beta=c_{\alpha}V_{\alpha}+c_{\beta}V_{\beta} \\
        \dfrac{c_\alpha}{c_\beta}=K
    \end{array}\right.\]
    解得
    \[\left\{\begin{array}{l}
        c_\beta=\dfrac{V_\beta}{KV_\alpha+V_\beta}c_0 \\
        c_\alpha=\dfrac{KV_\beta}{KV_\alpha+V_\beta}c_0
    \end{array}\right.\]
    于是萃取效率
    \[\eta=\dfrac{c_\alpha V_\alpha}{c_0V_\beta}=\dfrac{KV_\alpha}{KV_\alpha+V_\beta}\]
    由此可见,分配系数$K$越大,萃取剂体积$V_\alpha$越大,萃取效率$\eta$越高.
\end{solution}
实际情况中我们并不会只萃取一次,而是尽量用少量萃取剂萃取多次以达到更好的效果.%
现在我们来着手证明之.
\begin{theorem}[4C.5.4 多次萃取]
    用同样的总体积为$V$的萃取剂平均地萃取,多次萃取的总效率更高.
\end{theorem}
\begin{proof}
    我们在\tbf{4C.5.3}的解答过程中考虑留存于$\beta$相中的溶质$i$的浓度$c_\beta$,有
    \[c_\beta=\dfrac{V_{\beta}}{KV_\alpha+V_\beta}c_0\]
    因此进行$m$次萃取后的留存浓度
    \[c_{\beta,k}=c_0\prod_{i=1}^{m}\dfrac{V_\beta}{KV_\alpha+V_\beta}
    =\dfrac{c_0}{\displaystyle\prod_{i=1}^m\left(1+\dfrac{K}{V_\beta}\cdot V_\alpha\right)}\]
    又因为进行平均萃取,则每次用去的萃取剂的体积$V_{\alpha}=\dfrac{V}{m}$.于是
    \[\eta=1-\dfrac{c_{\beta,k}}{c_0}=1-\left(1+\dfrac{KV}{V_\beta}\cdot m\right)^{-m}\]
    令
    \[f(m)=\left(1+\dfrac{KV}{V_\beta}\cdot m\right)^{-m}\]
    我们有
    \[f'(m)=-\dfrac{KV}{V_\beta}\left(1+\dfrac{KV}{V_\beta}\cdot m\right)^{-m}\ln\left(1+\dfrac{KV}{V_\beta}\cdot m\right)\]
    当$m>1$时$f'(m)<0$,这表明$f(m)$随$m$的增大而减小,于是$\eta=1-f(m)$随$m$的增大而增大.\\
    这就证明了萃取次数越多,萃取效率越高.
\end{proof}
需要注意的是,尽管我们在理论上验证了萃取次数越多越好,实际上因为重复萃取几次后浓度便可降得很低,%
又考虑到操作的损耗,因此一般萃取次数在三到四次为宜.\vspace{12pt}\\
\Section{4C.6 非理想溶液的性质}
\Part{活度与活度系数}
\indent 在研究实际气体的性质时,我们曾经引入了逸度这一概念以计算实际气体的化学势,这里的逸度可以视作校准后的压力.%
同样地,在面对偏离Raoult定律的非理想溶液时,我们也可以定义一个新的变量来校准各组分的浓度.\\
\indent 考虑Raoult定律$\dfrac{p_i}{p_i^\ast}=x_i$.我们对摩尔分数$x_i$做校准,令
\[\dfrac{p_i}{p_i^\ast}=\gamma_{x,i}x_i\]
这样,组分$i$的化学势就可以改写为
\[\mu_i=\mu_i^\ast(T,p)+RT\ln\left(\gamma_{x,i}x_i\right)\]
为了方便考虑,我们不妨把$\gamma_{x,i}x_i$定义为一个新的变量以描述非理想溶液中各组分的化学势.
\begin{definition}[4C.6.1 活度与活度系数]
    非理想溶液中组分$i$的用摩尔分数表示的\tbf{活度}$a_i$是满足
    \[\mu_i=\mu_{i}^\ast(T,p)+RT\ln a_{x,i}\]
    的变量.用摩尔分数表示的\tbf{活度系数}$\gamma_{x,i}$定义为$\gamma_{x,i}=\dfrac{a_{x,i}}{x_i}$.\\
    对于纯物质的情况,有
    \[\lim_{x_i\to1}\gamma_{x,i}=1\]
    这时纯物质的活度就与其浓度相同.
\end{definition}
需要注意的是,尽管我们仍然用Raoult定律引入活度的概念,但它的定义仍应基于化学势%
\footnote{这和我们讨论理想溶液的热力学定义是同理的.}.从上面的定义可以看出,%
$\gamma_{x,i}$实际上也可以看作组分$i$对Raoult定律的偏离系数.\\
\indent 可以看出,上述表达式和用逸度表示实际气体的化学势的式子具有相同的形式.%
这再一次印证了两种体系具有某种程度上的相似性.\\
\indent 可以想见的是,对于双组分系统,两个组分的活度应当存在一定的联系.
\begin{derivation}
    由Gibbs-Duhem公式可知,在双组分系统的液相中有
    \[x_A\di\mu_A+x_B\di\mu_B=0\]
    又因为
    \[\di\mu=\di\left(\mu^\ast+RT\ln\gamma x\right)=RT\left(\dfrac{\di x}{x}+\dfrac{\di\gamma}{\gamma}\right)\]
    代入上式有
    \[x_A\left(\dfrac{\di x_A}{x_A}+\dfrac{\di\gamma_A}{\gamma_A}\right)+
    x_B\left(\dfrac{\di x_B}{x_B}+\dfrac{\di\gamma_B}{\gamma_B}\right)=0\]
    又因为$\di x_A+\di x_B=0$,于是
    \[\dfrac{x_A}{\gamma_A}\di\gamma_A+\dfrac{x_B}{\gamma_B}\di\gamma_B=0\]
    根据上式移项积分可得$\gamma$的值.
\end{derivation}
\begin{theorem}[4C.6.2 双组分系统中活度的关系]
    在由$A$和$B$构成的双液系中,两者的(摩尔分数表示的)活度系数满足
    \[\dfrac{x_A}{\gamma_A}\di\gamma_A+\dfrac{x_B}{\gamma_B}\di\gamma_B=0\]

\end{theorem}
\indent 对于理想稀溶液,由于其溶质多是固体或气体,很难找到实际存在的纯液相的溶质作为标准态.%
此时,我们更常用物质的量浓度表示的活度系数$\gamma_{c,i}$,而它衡量的实际上是溶质对Henry定律的偏离程度.%
在研究电解质溶液的性质时,我们会更加详细地讨论活度及其应用.\vspace{4pt}\\
\Part{渗透因子}
\indent 尽管看起来,我们已经找到了一个实用的参数——活度来描述非理想溶液,%
并且活度对非理想稀溶液的溶质也能很好地描述.然而,用活度表示此时溶剂对理想情况的偏差时,%
却会遇到活度与浓度差异很小的问题.例如,室温下水的$\text{KCl}$溶液(水的摩尔分数$x_\s=0.9328$)%
中水的活度$a_\s=0.9364$,活度系数$\gamma_\s=1.004$,这几乎与理想溶液相似,因此偏差难以得到更加精确的衡量.%
为此,我们定义一个新的变量描述此时溶剂对理想情况的偏离程度.
\begin{definition}[4C.6.3 渗透因子]
    定义非理想稀溶液中溶剂的\tbf{渗透因子}$\varphi$为满足
    \[\mu_\s=\mu_{\s}^\ast+\varphi RT\ln x_\s\]
    的参数,并且满足
    \[\lim_{x_\s\to1}\varphi=1\]
    不难看出$\phi$与$\gamma_\s$满足
    \[\phi=1+\dfrac{\ln\gamma_\s}{\ln x_\s}\]
    的关系.
\end{definition}
对于上述$\text{KCl}$溶液,水的渗透因子$\varphi=0.944$,这比用$\gamma$描述水对理想情况的偏离程度要显著得多.\vspace{4pt}\\
\Part{超额函数}
\indent 以上我们总体讨论了理想溶液和理想稀溶液中各组分的性质.如果要衡量整个溶液对理想情况的偏离,%
可以通过观察由各个组分混合成溶液前后的热力学函数的变化,于是就引出了下面的定义.
\begin{definition}[4C.6.4 超额函数]
    实际溶液的热力学性质可以用\tbf{超额函数}$Z^\text E$衡量,定义为实际观察到的混合过程的热力学函数$Z$的变化值与%
    形成理想溶液时$Z$的变化值,即
    \[Z^\text E=\Delta_{\text{mix}}Z^{\text{real}}-\Delta_{\text{mix}}Z^{\text{ideal}}\]

\end{definition}
我们在\tbf{4C.2.1}已经讨论过混合得到理想溶液这一过程的各热力学函数的变化.类似地,我们也可以通过活度求得超额函数.
\begin{derivation}\setcounter{equation}{0}
    我们首先讨论超额Gibbs自由能$G^\text E$.利用活度和偏摩尔量的加和公式可得
    \begin{equation}
        \begin{aligned}
            \Delta_{\text{mix}}G^{\text{real}}
            &= G_{\text{after}}-G_{\text{before}} = \sum_{i=1}^{k}\mu_in_i-\sum_{i=1}^{k}\mu_i^\ast n_i \\
            &= RT\sum_{i=1}^{k}n_i\ln a_i = RT\sum_{i=1}^{k}n_i\left(\ln x_i+\ln\gamma_i\right)
        \end{aligned}
    \end{equation}
    又因为理想溶液混合时
    \begin{equation}\Delta_{\text{mix}}G^{\text{ideal}}=RT\sum_{i=1}^{k}n_i\ln x_i\end{equation}
    于是(1)减去(2)可以得到
    \begin{equation}G^\text E=RT\sum_{i=1}^{k}n_i\ln\gamma_i\end{equation}
    我们已经知道
    \begin{equation}\pa GpT=V\end{equation}
    \begin{equation}\pa GTp=S\end{equation}
    \begin{equation}
        \left[\dfrac{\p}{\p T}\left(\dfrac{G}{T}\right)\right]_p=-\dfrac{H}{T^2}
    \end{equation}
    理想溶液混合的体积变化$\Delta_{\text{mix}}V^{\text{ideal}}=0$,于是超额体积
    \begin{equation}
        \begin{aligned}
            V^\text E
            &= \Delta_{\text{mix}}V^{\text{real}}-\Delta_{\text{mix}}V^{\text{ideal}}=\Delta_{\text{mix}}V^{\text{real}} \\
            &= \pa{G^\text E}pT=RT\sum_{i=1}^{k}n_i\pa{\ln\gamma_i}{p}T
        \end{aligned}
    \end{equation}
    理想溶液混合的焓变$\Delta_{\text{mix}}H^{\text{ideal}}=0$,于是超额焓
    \begin{equation}
        \begin{aligned}
            H^\text E
            &= \Delta_{\text{mix}}H^{\text{real}}-\Delta_{\text{mix}}H^{\text{ideal}}=\Delta_{\text{mix}}H^{\text{real}} \\
            &= -T^2\left[\dfrac{\p}{\p T}\left(\dfrac{G^\text E}{T}\right)\right]=-RT^2\sum_{i=1}^{k}\pa{\ln\gamma_i}Tp
        \end{aligned}
    \end{equation}
    同理,超额熵
    \begin{equation}
        \begin{aligned}
            S^{\text E}
            &= \Delta_{\text{mix}}S^{\text{real}}-\Delta_{\text{mix}}S^{\text{ideal}} \\
            &= -\pa{G^\text E}Tp \\
            &= -R\sum_{i=1}^{k}n_i\ln\gamma_i-RT\sum_{i=1}^{k}n_i\pa{\ln\gamma_i}Tp
        \end{aligned}
    \end{equation}
    由(3)(8)和(9)可以得到
    \[G^\text E=H^\text E-TS^\text E\]
    
\end{derivation}
这样我们就得到了用活度描述几个主要的超额函数的方法.
\begin{theorem}[4C.6.5 混合得到非理想溶液的热力学函数变化]
    混合得到非理想溶液的过程满足
    \[G^\text E=RT\sum_{i=1}^{k}n_i\ln\gamma_i\]
    \[V^\text E=RT\sum_{i=1}^{k}n_i\pa{\ln\gamma_i}{p}T\]
    \[H^\text E=-RT^2\sum_{i=1}^{k}\pa{\ln\gamma_i}Tp\]
    \[S^\text E=-R\sum_{i=1}^{k}n_i\ln\gamma_i-RT\sum_{i=1}^{k}n_i\pa{\ln\gamma_i}Tp\]
    并且有
    \[G^\text E=H^\text E-TS^\text E\]

\end{theorem}
对于某些溶液,也许其不理想性的来源是比较单一的,例如可能主要来源于混合焓或者主要来源于混合熵%
(这正好对应引起超额Gibbs自由能$G^\text E$的两项).我们尝试通过一定的近似寻找这样的溶液中组分的活度和温度的关系.
\begin{derivation}
    考虑
    \[G^\text E=H^\text E-TS^\text E\]
    如果$H^\text E\gg-TS^\text E$,说明溶液的非理想性主要由混合热引起.此时(近似地)有$S^\text E=0$.于是
    \[S^\text E=\pa{G^\text E}{T}{p}=0\]
    考虑各组分的超额化学势$\mu_i^\text E$,就有
    \[\pa{\mu_i^\text E}{T}{p}=\dfrac{\p}{\p T}\pa{G^\text E}{n_i}{T,p,n'}
    =\left(\dfrac{\p^2 G^\text E}{\p T\p n_i}\right)_p=\dfrac{\p}{\p n_i}\pa{G^\text E}{T}{p}=0\]
    于是我们知道定压下$\mu_i^\text E$不随温度变化.又因为
    \[\mu_i^\text E=RT\ln\gamma_i\]
    于是
    \[\ln\gamma_i\propto\dfrac{1}{T}\]
    如果$H^\text E\ll-TS^\text E$,说明溶液的非理想性主要由混合熵引起.此时(近似地)有$H^\text E=0$.于是
    \[H^\text E=-T^2\left[\dfrac{\p}{\p T}\left(\dfrac{G^\text E}{T}\right)\right]=0\]
    于是
    \[0=\dfrac{\p}{\p n_i}\left[\dfrac{\p}{\p T}\left(\dfrac{G^\text E}{T}\right)\right]_p
    =\dfrac{\p}{\p T}\left(\dfrac{1}{T}\dfrac{\p G^\text E}{\p n_i}\right)_p=\dfrac{\p}{\p T}\left(\dfrac{\mu_i^\text E}{T}\right)_p=0\]
    于是
    \[\pa{\ln\gamma_i}{T}{p}=0\]
    即此时活度不随温度变化.
\end{derivation}
我们讨论了超额Gibbs自由能(在某种程度上衡量了液体对理想情况的偏离程度)的主要来源,而上述两种情形各有对应的定义.
\begin{definition}[4C.6.6 正规溶液和无热溶液]
    满足$H^\text E\gg-TS^\text E$的溶液称为\tbf{正规溶液}.正规溶液的各组分的活度满足
    \[\ln\gamma_i\propto\dfrac{1}{T}\]
    满足$H^\text E\ll-TS^\text E$的溶液称为\tbf{无热溶液}.无热溶液的各组分的活度不随温度变化而变化.
\end{definition}
\end{document}