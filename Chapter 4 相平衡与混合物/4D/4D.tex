\documentclass{ctexart}
\usepackage{PhysicalChemistryNote}

\begin{document}\pagestyle{plain}
\noindent\tbf{\LARGE 4D 多组分系统的相变与相图}\vspace{15pt}\\
\indent 在经历了漫长的对溶液的讨论后,我们终于来到了多组分系统与相图这一节.%
尽管我们没有对多组分系统相变的规律进行总结,不过我们在\tbf{4C}已经不计其数地使用了多组分系统相平衡的条件.%
因此,本节事实上没有多少新的知识,更多的是总结与归纳,以及引入一些新的方法描述多组分系统各相的平衡情况.\vspace{12pt}\\
\Section{4D.1 多组分系统的相律}
\indent 我们已经在\tbf{4A.1.7}中不加证明地给出了单组分系统的相律.对于更一般的多组分系统,%
我们也需要给定一定数量的状态函数(例如温度$T$,压力$p$等等)来确定系统的状态.%
可以想见,需要给定的条件数应当与系统中的组分数和相数都有关系.下面我们来确定其具体关系式.
\begin{derivation}
    考虑某个平衡的系统,组分数为$S$,相数为$\varPhi$,并且假定各组分在每个相中都存在,且不发生化学反应.\\
    采取摩尔分数$x_i$表示每个相的组成.因为有$\displaystyle\sum_{i=1}^{k}x_i=1$的限制存在,%
    于是每个相都需要$S-1$个独立的变量以描述其组成.由于相数为$\varPhi$,再加上相平衡时各相的温度$T$和压力$p$,%
    就需要$\varPhi(S-1)+2$个变量描述该系统.\\
    然而,由于我们需要保证各组分在各相中的化学势$\mu$相等,于是就有
    \[\mu_{i,\alpha}=\mu_{i,\beta}=\cdots=\mu_{i,\phi}(i=1,2,\cdots,S)\]
    其中$\alpha,\beta,\cdots,\phi$为系统的所有$\varPhi$个相.\\
    由于化学势是温度$T$,压力$p$和摩尔分数$x_i$的函数,因此对于每个$i$,只需任意地确定某一相$\lambda$中的$x_{i,\lambda}$,%
    就可以通过上述等量关系求得所有相中的$x_i$.因此,对于每个组分$i$,上述等式意味着$\varPhi-1$个需要满足的等量关系,一共就有$S(\varPhi-1)$个需要满足的等量关系.\\
    每个等量关系都可以使我们描述系统的独立变量减少$1$个.根据自由度的定义,就有
    \[f=\left(\varPhi(S-1)+2\right)-\left(S(\varPhi-1)\right)=S-\varPhi+2\]
    稍作整理,就可得
    \[\varPhi+f=S+2\]

\end{derivation}
\begin{theorem}[4D.1.1 多组分系统的相律]
    对于一个不发生化学反应且各组分在各相中均存在的系统,其自由度$f$满足
    \[\varPhi+f=S+2\]
    其中$S$为系统的组分数,$\varPhi$为系统的相数.
\end{theorem}
\end{document}