\documentclass{ctexart}
\usepackage{PhysicalChemistryNote}

\begin{document}\pagestyle{plain}
\setcounter{footnote}{0}
\begin{center}
    \tbf{\Huge Chapter 4 相平衡与混合物}
\end{center}\vspace{15pt}

\indent 若你曾凝视一滴墨坠入清水,看它舒展成丝缕的烟霭,最终消融为一盏静谧的碧色;%
若你曾触碰冬日玻璃上的霜花,目睹冰晶与雾气在晨光中悄然谈判边界;%
若你煮咖啡时,牛奶的漩涡与棕褐的液体缠绵成大理石纹路——那么你早已在生活的褶皱里,窥见过相平衡的隐喻.\\
\indent 这是一场关于“共存”的哲学实验.%
在这里,乙醇与水不再争论谁更热情,而是牵起氢键的手,在微观世界里跳起优雅的圆舞曲;%
盐粒坠入大海时,结晶的执念被潮汐温柔瓦解,化作离子间无声的契约.%
相律如同一位沉默的指挥家,用$F=C-P+2$的韵律,为冰,水,蒸汽的三重奏谱写节拍.\\
\indent 我们将穿越由杠杆原理搭建的彩虹桥,在苯与水的分层中听见不溶的叹息,却又在乙醇与乙醚的拥抱里领悟混溶的浪漫.%
分馏塔像一首循环的十四行诗,用沸点的平仄将石油吟唱成汽油,柴油与沥青的韵脚;%
而Henry定律的常数,则是层叠在夏日汽水里的海潮,轻轻一摇便迸发出二氧化碳的浪沫.\\
\indent 当你在实验室摇晃分液漏斗,看两相液体在重力中缓慢诀别,%
那分明是宇宙在演示最古老的分离术——它让玫瑰提取液沉淀于醚,让叶绿素悬浮成翡翠的云.%
而相图里交错的曲线,原是物质写给温度与压力的情书:%
共晶点如一次宿命的邂逅,临界点则是液态与气态放弃界限的私奔.\\
\indent 翻开这一章吧,带上你的烧杯作茶盏,温度计当笔杆.%
我们将从咖啡杯里的扩散方程,一直航行到星际尘埃云的相变史诗.%
当科学的逻辑戴上诗意的冠冕,连蒸馏瓶折射的光,都成了实验室窗边未完成的十四行诗.
\end{document}