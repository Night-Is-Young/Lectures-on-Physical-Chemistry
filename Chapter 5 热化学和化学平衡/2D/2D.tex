\documentclass{ctexart}
\usepackage{PhysicalChemistryNote}

\begin{document}\pagestyle{plain}
\noindent\tbf{\LARGE 2D 热化学$^\ast$}\vspace{15pt}\\
\indent 终于,在经历数十页几乎纯物理的描述后,我们走向了与化学变化有关系的一部分——热化学.%
化学反应常常伴有吸热或者放热现象,对这些热效应进行精密测定并做详尽的讨论,即\tbf{热化学}着重研究的方向.\vspace{12pt}\\
\Section{2D.1 热化学的基础概念}
\Part{反应进度}
\indent 为了方便讨论化学反应进行的程度,我们需要用一个统一的量对“反应进行了多少”进行衡量.%
这一衡量方式应当与参与反应的物质的计量比无关,以具有更好的普适性.因此,我们定义了\tbf{反应进度}.
\begin{definition}[2D.1.1 反应进度]
    考虑反应
    \[a\text{A}+b\text{B}\longrightarrow c\text{C}+d\text{D}\]
    用$\nu_{\text X}$表示物质X的计量系数.对于反应物,$\nu$取负值;对于产物,$\nu$取正值.定义反应进度$\xi$为
    \[n_\text X(\xi)\xlongequal{def}n_\text X(0)+\nu_\text B\xi\]
    其中$n_\text X(0),n_\text X(\xi)$分别代表起始时和反应进度为$\xi$时X的物质的量.因此,$\xi$的单位是mol.
\end{definition}
对$\xi$的定义式微分可得
\[\di\xi=\dfrac{\di n_\text X}{\nu_\text X}\]
\indent 引入反应进度最大的优点在于你可以用任意一种反应物或者产物的物质的量的变化衡量反应进行的程度.例如,对于我们在\tbf{2D.1.1}提到的反应有
\[\di\xi=-\dfrac{\di n_\text A}{a}=-\dfrac{\di n_\text B}{b}=\dfrac{\di n_\text C}{c}=\dfrac{\di n_\text D}{d}\]
\indent 反应进度必须与具体的化学反应方程式相对应,否则我们无法得知各物质的计量数,也就不能定义反应进度.\vspace{4pt}\\
\Part{等压热效应和等容热效应}
\indent 系统发生化学反应后可能与环境有热的交换.因此,我们定义热效应.
\begin{definition}[2D.1.2 热效应]
    系统发生化学变化后,其温度回到反应前始态的温度,系统放出或吸收的热量称为该反应的\tbf{热效应}.
\end{definition}
需要注意的是,系统的温度一定要回复反应前始态的温度.热效应可以通过各种量热计进行\footnote{在普通化学书中有详细介绍.}.%
热效应的正负仍采取热力学惯例,系统吸热为正值,放热为负值.\\
\indent 通常如果不特别注明,反应热都是指等压热效应.然而,常用的量热计都是测定的热效应都是等容的.%
因此,有必要知道等压热效应$Q_V$和等容热效应$Q_p$的差值.
\begin{derivation}
    
\end{derivation}
\end{document}