\documentclass{ctexart}
\usepackage{PhysicalChemistryNote}

\begin{document}\pagestyle{plain}
\noindent\tbf{\LARGE 5B 化学平衡}\vspace{15pt}\\
\indent 正如我们在\tbf{3E.2.2}中讲的那样,等温等压下系统自发变化的方向总是向着%
Gibbs自由能$G$减小的方向进行,直到系统平衡为止.这一原理自然也可以用于发生化学反应的系统,%
我们需要通过$G$的变化来研究系统达到平衡的条件.\vspace{12pt}\\
\Section{5B.1 平衡常数与标准摩尔Gibbs自由能变}
\Part{化学反应的Gibbs自由能变}
\indent 显然的,达到化学平衡的体系必然是混合物,因此我们需要用化学势代替纯物质的Gibbs自由能以对体系进行研究.%
我们不妨从最简单的平衡反应\ce{$A$ <=> $B$}开始研究.
\begin{derivation}
    在等温等压下,含有上述$A$和$B$物质的系统的Gibbs自由能的微分
    \[\di G=\mu_{A}\di n_A+\mu_B\di n_B\]
    依照反应进度的定义,如果反应进行了$\di\xi$,那么$\di n_A=-\di\xi,\di n_{B}=\di\xi$.这样就有
    \[\di G=\left(\mu_B-\mu_A\right)\di\xi\]
    亦即
    \[\pa{G}{\xi}{T,p}=\left(\mu_B-\mu_A\right)\]
    回顾化学势的定义,$\mu_B-\mu_A$不仅代表$B$与$A$在这样的系统组成下的化学势的差值,也等于在大量(该组成的)系统中反应进行$1\mol$上述反应的Gibbs自由能变.%
    这就是摩尔Gibbs自由能变的定义.
\end{derivation}
\begin{definition}[5B.1.1 反应摩尔Gibbs自由能变]
    发生化学反应的系统的\tbf{反应摩尔Gibbs自由能变}$\Delta_\r G_\m$定义为
    \[\Delta_\r G_\m=\pa G\xi{T,p}=\sum_{i=1}^{k}\nu_i\mu_i\]
    在系统的组成一定时,$\Delta_\r G_\m$既可以定义为单位物质的量的产物与反应物的化学势之差,也可以定义为系统的Gibbs自由能$G$对反应进度$\xi$的变化率.%
    这两种定义是等价的.
\end{definition}
我们仍然以上述反应为例讨论系统的Gibbs自由能何时达到最小值.
\begin{derivation}
    等温等压下,系统的Gibbs自由能$G$只与$\xi$有关,可以作图如下.
    \begin{tightcenter}
        \documentclass{standalone}
\usepackage{PhysicalChemistryNote}
\begin{document}
\begin{tikzpicture}
    \draw[-] (0,0) -- (6,0);
    \draw[-] (0,0) -- (0,4);
    \draw[-] (6,0) -- (6,4);
    \draw[-] (0,4) -- (6,4);
    \node[below] at (0,0) {$A$};
    \node[below] at (6,0) {$B$};
    \node[below] at (3,0) {$\xi$};
    \draw[domain=0:6] plot[smooth](\x,{1+(\x-4.25)^2/8});
    \node[left] at (0,3.2578) {$\mu_A^\ast$};
    \node[right] at (6,1.3828) {$\mu_B^\ast$};
    \draw[dashed] (5.5,1) -- (3,1) node[left]{\small{$\pa{G}{\xi}{T,p}=0$}};
    \fill (4.25,1) circle (1.5pt);
\end{tikzpicture}
\end{document}
    \end{tightcenter}
    当$\pa G\xi{T,p}<0$时,$G$随$\xi$的增大而减小,系统自发地向$\xi$增大的方向变化,这时有$\mu_B<\mu_A$.\\
    当$\pa G\xi{T,p}<0$时,$G$随$\xi$的增大而增大,系统自发地向$\xi$减小的方向变化,这时有$\mu_B>\mu_A$.\\
    当$\pa G\xi{T,p}=0$时,$G$达到最小值,即系统达到平衡,这时有$\mu_B=\mu_A$.
\end{derivation}
这和我们在相平衡用到的判据是一致的,只不过把达成平衡的两相改成达成平衡的两种物质.\vspace{4pt}\\
\Part{理想气体混合物的平衡}
\indent 如果我们知道体系中各种物质的化学势的表达式,就可以得出平衡的条件.%
现在我们从最简单的情况,假定$A$和$B$都是理想气体开始.
\begin{derivation}\setcounter{equation}{0}
    我们已经知道理想气体混合物的化学势
    \begin{equation}
        \mu_{i}=\mu_i^\ominus(T)+RT\ln\dfrac{p_i}{p^\ominus}
    \end{equation}
    我们需要根据$\mu_B-\mu_A$的值判断反应是否平衡(或者自发地向哪个方向移动).将(1)代入,则有
    \begin{equation}
        \begin{aligned}
            \mu_B-\mu_A
            &= \left(\mu_B^\ominus(T)+RT\ln\dfrac{p_B}{p^\ominus}\right)-\left(\mu_A^\ominus(T)+RT\ln\dfrac{p_A}{p^\ominus}\right) \\
            &= \left(\mu_B^\ominus(T)-\mu_A^\ominus(T)\right)+RT\ln\dfrac{p_B}{p_A}
        \end{aligned}
    \end{equation}

    我们知道$\mu_i^\ominus(T)$实际上就是纯的$i$在压力为$p^\ominus$时的摩尔Gibbs自由能,于是可以将(2)改写为
    \begin{equation}
        \Delta_\r G_\m=\left(G_{\m,B}^\ominus(T)-G_{\m,A}^\ominus(T)\right)+RT\ln\dfrac{p_B}{p_A}
    \end{equation}
    现在,我们把$G_{\m,B}^\ominus(T)-G_{\m,A}^\ominus(T)$记为$\Delta_\r G_\m^\ominus$,这就是我们接下来要定义的\tbf{标准反应Gibbs自由能变};%
    把分压之比$\dfrac{p_B}{p_A}$记为$Q$,这就是我们接下来将要定义的\tbf{反应商}.当$\Delta_\r G_\m=0$时,反应达到平衡,%
    又因为$G_{\m,A}^\ominus(T)$和$G_{\m,B}^\ominus(T)$在等温条件下都是常数,因此平衡时的$Q$满足
    \[Q=\exp\left(-\dfrac{\Delta_\r G_\m}{RT}\right)\]
    在等温条件下也是常数,这就是我们接下来要定义的\tbf{平衡常数}.
\end{derivation}
\Part{标准反应Gibbs自由能变}
\indent 在前面的推导中首先出现了一个量,即纯的$i$在压力为$p^\ominus$时的摩尔Gibbs自由能$G_{\m,i}^\ominus(T)$.%
与焓相同,我们无法确定$G_{\m,i}^\ominus$的绝对数值.但因为我们只需知道$G_{\m,i}^\ominus$的差值,%
因此我们可以采用与定义标准焓变相似的办法定义一个类似的量,即标准反应Gibbs自由能变.%
\begin{definition}[5B.1.3 标准反应Gibbs自由能变]
    由标准态的反应物完全转变为标准态的产物这一过程的Gibbs自由能的变化即为\tbf{标准反应Gibbs自由能变}.%
    如果令$\xi=1\mol$,则这一过程的焓变即为\tbf{标准摩尔(反应)Gibbs自由能变},记为$\Delta_\r G_\m^\ominus$.
\end{definition}
这样,对于反应\ce{A <=> B}就有
\[\Delta_\r G_{\m}^\ominus=G_{\m,A}^\ominus(T)-G_{\m,B}^\ominus(T)\]
\indent 可以看出,标准反应Gibbs自由能变(\tbf{5B.1.3})与反应Gibbs自由能变(\tbf{5B.1.1})的定义完全不同.%
前者强调完全转化(因此也是假想的情况),是处于标准态的所有产物与反应物的Gibbs自由能之差(或者更准确地说是它们的化学势之差);%
后者则是一个微分量,与系统的组成相关,并且当$\Delta_\r G=0$时系统就达到平衡.\\
\indent 由于Gibbs自由能也是状态函数,因此Hess定律对它仍然适用.于是为了方便计算所有反应的$\Delta_\r G_{\m}^\ominus$,%
我们可以定义各物质的标准生成Gibbs自由能.
\begin{definition}[5B.1.4 标准生成Gibbs自由能]
    在标准压力下和给定的温度$T$下,由稳定单质生成物质\ce{X}的标准摩尔Gibbs自由能变即\ce{X}的\tbf{标准摩尔生成Gibbs自由能},%
    用符号$\Delta_\f G_\m^\ominus(\text{X,物态,}T)$表示.
\end{definition}
这样,就可以通过标准生成Gibbs自由能来推算各个反应的标准反应Gibbs自由能变.\vspace{4pt}\\
\Part{反应商与平衡常数}
\indent 前面我们已经知道了理想气体混合物的化学平衡存在平衡常数.%
那么,是否所有化学反应都如此呢?答案是肯定的.
\begin{derivation}\setcounter{equation}{0}
    我们考虑一个具有$k$种物质的系统,并假定它们由一个化学反应联系.\\
    在等温等压下,系统的Gibbs自由能的微分为
    \begin{equation}
        \di G=\sum_{i=1}^{k}\mu_i\di n_i=\sum_{i=1}^{k}\mu_i\left(\nu_i\di\xi\right)=\left(\sum_{i=1}^{k}\nu_i\mu_i\right)\di\xi
    \end{equation}
    于是
    \begin{equation}
        \Delta_\r G_\m=\pa G\xi{T,p}=\sum_{i=1}^{k}\nu_i\mu_i
    \end{equation}
    根据\tbf{4B.4.1},理想气体混合物中各组分的化学势
    \begin{equation}
        \mu_i=\mu_i^\ominus+RT\ln\dfrac{p_i}{p^\ominus}
    \end{equation}
    根据\tbf{4B.4.2},实际气体混合物中各组分的化学势
    \begin{equation}
        \mu_i=\mu_i^\ominus+RT\ln\dfrac{f_i}{p^\ominus}
    \end{equation}
    根据\tbf{4C.1.2}和\tbf{4C.1.5},理想溶液和理想稀溶液中各组分的化学势
    \begin{equation}
        \mu_i=\mu_i^\ominus+RT\ln x_i
    \end{equation}
    需要注意的是$\mu_i^\ominus$可能因为选取的标准态满足Henry定律还是满足Raoult定律而有所不同.\\
    根据\tbf{4C.6.1},非理想溶液中各组分的化学势
    \begin{equation}
        \mu_i=\mu_i^\ominus+RT\ln a_i
    \end{equation}
    观察上述所有化学势的计算公式,我们都可以把它们写作
    \begin{equation}
        \mu_i=\mu_i^\ominus+RT\ln a_i
    \end{equation}
    其中$a_i$即为组分$i$(广义的)活度.将(7)代入(2)就有
    \begin{equation}
        \begin{aligned}
            \Delta_\r G_\m
            &= \sum_{i=1}^k\nu_i\mu_i = \sum_{i=1}^k\nu_i\mu_i^\ominus+RT\sum_{i=1}^k\left(\nu_i\ln a_i\right) \\
            &= \Delta_\r G_\m^\ominus+RT\ln\left(\prod_{i=1}^k a_i^{\nu_i}\right)
        \end{aligned}
    \end{equation}
    前面一项即为标准反应Gibbs自由能变的定义.现在,定义后面括号内的一项为\tbf{反应商}$Q$.%
    由于反应物的计量系数为负数,因此它们的活度将自然地出现在分母上,而产物的活度将出现在分子上.于是可以将(8)改写为
    \begin{equation}
        \Delta_\r G_\m=\Delta_\r G_\m^\ominus+RT\ln Q
    \end{equation}
    当$\Delta_\r G_\m=0$时反应达到平衡,记此时的反应商$Q$为\tbf{标准平衡常数}$K^\ominus$,于是平衡时就有
    \begin{equation}
        \prod_{i=1}^k a_i^{\nu_i}=K^\ominus=\exp\left(-\dfrac{\Delta_\r G_\m^\ominus}{RT}\right)
    \end{equation}
    亦即
    \begin{equation}
        \Delta_\r G_\m^\ominus=-RT\ln K^\ominus
    \end{equation}
    一般来说,出现在(7)式中的活度$a_i$和标准态时的$\mu_i^\ominus$将按照如下方式进行替换.
    \begin{enumerate}[label=\tbf{\alph*.}]
        \item 对于气相物质,活度$a_i=\dfrac{f_i}{p^\ominus}$,其中$f_i$为组分$i$的逸度.\\
            一般而言,我们总是近似地\tbf{假定气相为理想气体混合物},这样就有$a_i=\dfrac{p_i}{p^\ominus}$.\\
            选取的标准态即为纯的气体$i$在指定温度$T$和标准压力$p^\ominus$下的状态.
        \item 对于溶液(不管是液相还是固相),我们一般讨论稀溶液的溶质,并有$a_i=\dfrac{a_{i,c}}{c^\ominus}$.\\
            如果\tbf{假定溶液是理想的},就有$a_i=\dfrac{c_i}{c^\ominus}$,其中$c_i$为$i$的物质的量浓度,$c^\ominus=1\mol\cdot\text L^{-1}$.\\
            选取的标准态即为物质的量活度$a_{i,c}=1\mol\cdot\text L^{-1}$(非理想情况)或浓度$c_i=1\mol\cdot\text L^{-1}$(理想情况).
        \item 对于纯的液相或固相,化学势不变,其活度就恒定为$1$,标准态即为纯的液相或固相.\\
            一般的稀溶液的溶剂(例如水溶液中的水)$x_\s$几乎恒定为$1$,根据Raoult定律
            \[\mu_\s=\mu_\s^\ast+RT\ln x_\s\]
            其化学势变化很小且几乎与纯态一致,%
            因此这时也可以将溶剂视为纯液相,活度恒定为$1$,标准态即为纯溶剂.
    \end{enumerate}
    事实上,为了方便考虑,在无特殊说明的情况下总是假定我们的反应体系是理想混合物(气相为理想气体混合物,溶液相为理想溶液或理想稀溶液).\\
    这样,所有的活度的量纲\footnotemark 为$1$,因而平衡常数$K^\ominus$的量纲也为$1$.
\end{derivation}\footnotetext{通俗地说,物理量的量纲就是它的单位.}
现在我们可以说,任意的反应都存在反应商和平衡常数.
\begin{definition}[5B.1.5 反应商与平衡常数]
    对于一个化学反应,其\tbf{反应商}$Q$定义为
    \[Q=\prod_{i=1}^k a_i^{\nu_i}\]
    其中$\nu_i$为物质$i$的化学计量数,$a_i$为选定标准态后的活度.\\
    \tbf{标准平衡常数}$K^\ominus$(亦称为\tbf{热力学平衡常数})定义为反应平衡(即$\Delta_\r G_\m=0$)时的反应商,满足
    \[\Delta_\r G_\m^\ominus=-RT\ln K^\ominus\]

\end{definition}
需要注意,如果选定的标准态不同,那么活度和标准反应Gibbs自由能都会发生变化.\\
\indent 上述定义中的最后一个式子是非常重要的,它将处于标准态的各物质与平衡时的体系联系在了一起,%
成为沟通理想和现实的桥梁.从今往后,只要知道反应在某一温度下的$\Delta_\r G_\m^\ominus$,我们就能知道该温度下%
的平衡常数,进而方便地计算平衡时系统的组成.\vspace{4pt}\\
\Part{由反应焓变和熵变求算反应Gibbs自由能变}
\indent Gibbs自由能$G$是难以测定的状态函数,因而反应的$\Delta_\r G_\m^\ominus$也是难以直接测定的.%
因此,我们需要考虑以容易测定的量来计算$\Delta_\r G_\m^\ominus$.
\begin{derivation}
    在恒定温度$T$下,首先有
    \[\Delta_\r G_\m^\ominus=\sum_{i=1}^{k}\nu_i\mu_i^\ominus=\sum_{i=1}^{k}\nu_i G_{\m,i}^\ominus\]
    又根据Gibbs自由能的定义
    \[G_{\m,i}^\ominus=H_{\m,i}^\ominus-TS_{\m,i}^\ominus\]
    其中$H_{\m,i}^\ominus$和$S_{\m,i}^\ominus$分别为纯的$i$在标准态下的摩尔焓和摩尔熵.\\
    采取我们在\tbf{3D.3.3}中已经给出的物质的规定熵作为其标准摩尔熵,代入上式就有
    于是我们就有
    \[\Delta_\r G_{\m}^\ominus
    =\sum_{i=1}^{k}\nu_i G_{\m,i}^\ominus
    =\sum_{i=1}^k\nu_i\left(H_{\m,i}^\ominus-TS_{\m,i}^\ominus\right)
    =\Delta_\r H_{\m}^\ominus-T\Delta_\r S_{\m}^\ominus\]
    其中$\Delta_\r H_{\m}^\ominus$和$\Delta_\r S_{\m}^\ominus$分别为反应的标准摩尔焓变和标准摩尔熵变.
\end{derivation}
\begin{theorem}[5B.1.6 由$\Delta_\r H_{\m}^\ominus$和$\Delta_\r S_{\m}^\ominus$求算$\Delta_\r G_{\m}^\ominus$]
    我们有
    \[\Delta_\r G_{\m}^\ominus=\Delta_\r H_{\m}^\ominus-T\Delta_\r S_{\m}^\ominus\]
    这样就可以由反应的标准摩尔焓变和熵变求算标准摩尔Gibbs自由能变.需要注意的是,所使用的数据要求温度为$T$.%
\end{theorem}
我们将在下一节讨论温度对$\Delta_\r G_\m^\ominus$的影响.\vspace{12pt}\\
\Section{5B.2 条件对化学平衡的影响}
\Part{Le Chatelier原理\footnote{音译为“勒夏特列”,这就是我们所熟知的勒夏特列原理.}}
\indent 除了Hess定律之外,我们在普通化学中学到的另一个重要的经验定律就是Le Chatelier原理.%
\begin{theorem}[5B.2.1 Le Chatelier原理]
    化学平衡是动态平衡,如果改变影响平衡的一个因素,平衡就向能够减弱(或消除)这种改变的方向移动,以抗衡该改变.
\end{theorem}
需要注意的是,如果反应体系的多个条件同时发生了改变,Le Chatelier原理可能失效.\\
\indent 如果你学习过楞次定律,就会发现它们实际上都是负反馈调节的经典的例子%
\footnote{笔者曾经读过一篇名为《勒夏特列原理与老庄思想》的文章,其中提到了《道德经》中“反者,道之动”与该原理的相似性.}.%
我们将在接下来的几小节中用严谨的推导说明这一原理的正确性,并定量地衡量条件变化对化学平衡的影响.\vspace{4pt}\\
\Part{温度对化学平衡的影响——van't Hoff方程}
\indent 温度对化学平衡的影响是很常见的.恒压条件下装有\ce{N2O4}的容器,随着温度升高其颜色会变浅;%
抽成真空的容器放入\ce{CaCO3},容器内气体的压力随着温度升高而增大,等等.究其原因,是因为各物质的标准摩尔Gibbs自由能会随温度变化而变化.%
现在我们来具体推导温度对平衡的影响.
\begin{derivation}\setcounter{equation}{0}
    根据\tbf{3F.2.7}(或\tbf{4B.3.3}),纯物质$i$的标准摩尔Gibbs自由能满足
    \begin{equation}
        \left[\dfrac{\p}{\p T}\left(\dfrac{G_{\m,i}^\ominus}{T}\right)\right]_{p}=-\dfrac{H_{\m,i}^\ominus}{T^2}
    \end{equation}
    又因为
    \begin{equation}
        \Delta_\r G_\m^\ominus=\sum_{i=1}^{k}\nu_i G_{\m,i}^\ominus
    \end{equation}
    将(2)代入(1)中有
    \begin{equation}
        \begin{aligned}
            \left[\dfrac{\p}{\p T}\left(\dfrac{\Delta_\r G_\m^\ominus}{T}\right)\right]_{p}
            &= \left[\dfrac{\p}{\p T}\left(\dfrac{1}{T}\sum_{i=1}^{k}\nu_i G_{\m,i}^\ominus\right)\right]_{p} = \sum_{i=1}^{k}\nu_i\left[\dfrac{\p}{\p T}\left(\dfrac{G_{\m,i}^\ominus}{T}\right)\right]_{p} \\
            &= \sum_{i=1}^{k}\nu_i\left(-\dfrac{H_{\m,i}^\ominus}{T^2}\right) = -\dfrac{\Delta_\r H_{\m}^\ominus}{T^2}
        \end{aligned}
    \end{equation}
    又因为标准平衡常数$K^\ominus$满足
    \begin{equation}
        \Delta_\r G_\m^\ominus=-RT\ln K^\ominus
    \end{equation}
    将(4)代入(3)中有
    \begin{equation}
        \pa{\ln K^\ominus}{T}{p}=\dfrac{\Delta_{\r}H_{\m}^\ominus}{RT^2}
    \end{equation}
    这就是van't Hoff方程.\\
    将此式在等压条件下移项积分就有
    \begin{equation}
        \ln\dfrac{K^\ominus\left(T_2\right)}{K^\ominus\left(T_1\right)}
        =\int_{T_1}^{T_2}\dfrac{\Delta_{\r}H_{\m}^\ominus}{RT^2}\di T
    \end{equation}
    或其不定积分形式
    \begin{equation}
        \ln K^\ominus(T)=\int\dfrac{\Delta_\r H_\m^\ominus}{RT^2}+I
    \end{equation}
    积分常数可以由给定的数据求得(例如给出的$298.15$ K下的各标准数据).\\
    如果假定反应的标准摩尔焓变$\Delta_{\r}H_{\m}^\ominus$随温度变化不大,就有
    \begin{equation}
        \ln\dfrac{K^\ominus\left(T_2\right)}{K^\ominus\left(T_1\right)}
        =\dfrac{\Delta_\r H_\m^\ominus}{R}\left(\dfrac{1}{T_1}-\dfrac{1}{T_2}\right)
    \end{equation}
    你可以发现此式与Clausius-Clapeyron方程是极其相似的.事实上.如果你愿意的话,可以把所有相平衡都当作化学平衡处理,%
    这在物理上是等价的.\\
    如果已知反应的标准等压热容变化$\Delta C_{p,\m}$,根据\tbf{5A.3.1}Kirchhoff定律
    \begin{equation}
        \Delta_\r H_\m^\ominus(T)=\Delta_\r H_\m^\ominus\left(T_0\right)+\int_{T_0}^{T}\Delta C_{p,\m}\di T
    \end{equation}就有
    不妨对$\Delta C_{p,\m}$做Taylor展开(一般展开两到三项即可满足精度需求),就有
    \begin{equation}
        \Delta C_{p,\m}=a+bT+cT^2+\cdots
    \end{equation}
    然后将(10)代入(9)并写成不定积分形式,就有
    \begin{equation}
        \Delta_\r H_\m^\ominus(T)=\Delta H_0+\int\Delta C_{p,\m}\di T=\Delta H_0+aT+\dfrac12bT^2+\dfrac13cT^3+\cdots
    \end{equation}
    理论上,(11)式中的积分常数$\Delta H_0$为该反应在$0$ K时的标准焓变,不过在温度接近近绝对零度时(10)的Taylor展开将失效,因此最好只把它当作一个常数,不去深究其物理意义.%
    这一常数亦可以通过给定温度下的标准焓变得出.\\
    由此,我们将(11)代入(7)可得
    \begin{equation}
        \begin{aligned}
            \ln K^\ominus(T)
            &= \int\dfrac{\Delta_\r H_\m^\ominus}{RT^2}+I \\
            &= \int\dfrac{1}{RT^2}\left(\Delta H_0+aT+\dfrac12bT^2+\dfrac13cT^3+\cdots\right)\di T+I \\
            &= \dfrac{1}{R}\left(-\dfrac{\Delta H_0}{T}+a\ln T+\dfrac12 bT+\dfrac16cT^2+\cdots\right)+I
        \end{aligned}
    \end{equation}
    这样,只需要知道某一给定温度$T_0$下反应的标准平衡常数$K^\ominus\left(T_0\right)$和标准焓变%
    $\Delta_\r H_\m^\ominus\left(T_0\right)$就可以根据(7)式求出积分常数$I$,%
    再由$\Delta C_{p}$对$T$的级数展开式和$\Delta_\r H_\m^\ominus\left(T_0\right)$%
    就可以根据(11)求得另一积分常数$\Delta H_0$,进而根据(12)式求出任意温度$T$下反应的标准平衡常数.
\end{derivation}
\begin{theorem}[5B.2.2 van't Hoff方程]
    等压条件下有
    \[\pa{\ln K^\ominus}{T}{p}=\dfrac{\Delta_{\r}H_{\m}^\ominus}{RT^2}\]
    van't Hoff方程的积分形式和各种近似计算的方式已经在前面的推导中给出.
\end{theorem}
如果将$\Delta_\r H_\m^\ominus$视作常数,就可以得出以下经验定律.
\begin{theorem}[5B.2.3 温度对平衡的影响]
    如果一个反应是放热反应(即$\Delta_\r H_\m^\ominus<0$),那么温度升高会使得$K^\ominus$减小,平衡逆向移动.\\
    如果一个反应是吸热反应(即$\Delta_\r H_\m^\ominus>0$),那么温度升高会使得$K^\ominus$增大,平衡正向移动.
\end{theorem}
这里的逆向或正向移动对应$\di\xi<0$或$\di\xi>0$.这是符合Le Chatelier原理的,即温度升高会使得平衡向吸热的方向移动,%
以期阻止温度升高带来的影响.\vspace{4pt}\\
\Part{压力对化学平衡的影响}
\indent 尽管标准平衡常数$K^\ominus$与压力$p$无关,但改变系统的压力$p$仍然可能会导致平衡移动.这主要取决于施加压力的方式.%
具体而言,如果各组分的分压发生变化,那么平衡就有可能发生改变.\\
\indent 如果是向系统中充入惰性气体并保持气体体积不变,并且假设系统是理想气体混合物,那么%
各参与反应的组分的分压并不改变,因而此时仍有$K^\ominus=Q$,即充入惰性气体不会使平衡发生移动.
\begin{theorem}[5B.2.4 充入惰性气体对平衡的影响I]
    如果系统是理想气体混合物,那么恒容时充入惰性气体对平衡没有影响.
\end{theorem}
在研究普通的气相的化学平衡中,由于我们总是假定各组分为理想气体,因而\tbf{5B.2.4}一般而言是成立的.\\
\indent 如果我们以压缩系统的方式增大系统的压力,就会导致各组分的分压也随之增大.%
仍然假设系统是理想气体混合物,我们现在来简单地讨论此时平衡移动的方向.
\begin{derivation}\setcounter{equation}{0}
    设系统的总压力为$p$,组分$i$的摩尔分数$x_i=\dfrac{p_i}{p}$.\\
    于是有
    \begin{equation}
        K^\ominus=\prod_{i=1}^{k}\left(\dfrac{p_i}{p^\ominus}\right)^{\nu_i}
        =\prod_{i=1}^{k}x_i^{\nu_i}\cdot\left(\dfrac{p}{p^\ominus}\right)^{\sum_{i=1}^{k}\nu_i}
    \end{equation}
    令$K_x=\displaystyle\prod_{i=1}^{k}x_i^{\nu_i}$,就可以用$K_x$衡量系统中反应物和产物的相对含量.$K_x$越高,产物的含量越高;反之,$K_x$越低,反应物的含量越高.%
    将$K_x$代入(1)式就有
    \begin{equation}
        \ln K^\ominus=\ln K_x+\dfrac{p}{p^\ominus}\sum_{i=1}^{k}\nu_i
    \end{equation}
    在恒定温度下,$K^\ominus$是常数.将(2)式对$p$求偏微分可得
    \begin{equation}
        \pa{\ln K_x}{p}{T}=-\dfrac{1}{p}\sum_{i=1}^{k}\nu_i
    \end{equation}
    考虑到各组分都是理想气体,因此
    \begin{equation}
        \Delta_{\r}V_\m=\sum_{i=1}^{k}\nu_i V_{\m,i}=\dfrac{RT}{p}\sum_{i=1}^{k}\nu_i
    \end{equation}
    代入(3)式可得
    \begin{equation}
        \pa{\ln K_x}{p}{T}=-\dfrac{\Delta_\r V_{\m}}{RT}
    \end{equation}
    从(5)可以看出,%
    如果$\Delta_\r V_\m<0$,则$K_x$随$p$的增大而增大,反之则$K_x$随$p$的增大而减小.\\
    总结地说,压力增大会使得平衡向体积减小的方向移动.
\end{derivation}
\begin{theorem}[5B.2.5 压缩系统对平衡的影响]
    以压缩系统的方式增大压强,会使得平衡向体积减小的方向移动.
\end{theorem}
这也是符合Le Chatelier原理的,反应向体积减小的方向进行是为了减小压缩系统带来的体积减小的效应.%
特别地,如果我们在恒压条件下向系统充入惰性气体,那么各组分的分压也会等比例地减小,等效于使系统膨胀.这时有
\begin{theorem}[5B.2.6 充入惰性气体对平衡的影响II]
    如果系统是理想气体混合物,那么恒压时充入惰性气体会使得平衡向体积增大的方向移动.
\end{theorem}
我们以一个简单的例子让你更直观地感受压力对反应的影响.
\begin{problem}[5B.P.1]
    假定初始状态下系统中充入一定量的\ce{N2O4}.考虑反应
    \begin{tightcenter}
        \ce{N2O4 <=> 2NO2}
    \end{tightcenter}
    在温度为$T$时的标准平衡常数为$K^\ominus$.求\ce{N2O4}的解离度$\alpha$和系统总压$p$之间的关系.
\end{problem}
在解答这道题目之前,我们首先需要介绍解离度的概念.%
正如它的名字所说的那样,解离度衡量了某一物质(一般是分解反应的反应物)在平衡时分解的比例.
\begin{definition}[5B.2.7 解离度]
    在一定条件下物质达到解离平衡时,已解离的分子数与原有的分子数之比即为该物质在此条件下的\tbf{解离度}.
\end{definition}
一般来说,我们要求系统在起始时只有反应物.如果反应物和产物都存在,并且产物又不成比例,就很难界定解离度了.\\
\indent 现在我们来求解\tbf{5B.P.1}.
\begin{solution}
    假定初始时$n_{\ce{N2O4}}=n$.根据解离度的定义有
    \[n_{\ce{N2O4}}=\left(1-\alpha\right)n\ \ \ \ \ n_{\ce{NO2}}=2\alpha n\]
    于是\ce{NO2}和\ce{N2O4}的摩尔分数分别为
    \[x_{\ce{NO2}}=\dfrac{n_{\ce{NO2}}}{n_{\ce{N2O4}}+n_{\ce{NO2}}}=\dfrac{2\alpha}{1+\alpha}\ \ \ \ \ x_{\ce{N2O4}}=\dfrac{1-\alpha}{1+\alpha}\]
    于是
    \[K^\ominus=\dfrac{\left(\dfrac{p_{\ce{NO2}}}{p^\ominus}\right)^2}{\left(\dfrac{p_{\ce{N2O4}}}{p^\ominus}\right)}
    =\dfrac{p}{p^\ominus}\cdot\dfrac{x_{\ce{NO2}}^2}{x_{\ce{N2O4}}}
    =\dfrac{p}{p^\ominus}\cdot\dfrac{4\alpha^2}{1-\alpha^2}\]
    于是
    \[\alpha=\sqrt{\dfrac{1}{1+4\dfrac{p}{Kp^\ominus}}}\]
    这表明解离度随着压强的增大而减小.这也符合\tbf{5B.2.5}的结论.
\end{solution}
\vspace{8pt}
\Section{5B.3 可逆反应与不可逆反应}
\indent 或许你已经注意到,我们前面并没有界定一个反应可逆还是不可逆.%
这或许是因为反应可逆与否与化学反应动力学也有着相当的联系.因此,我们在这里仅对“可逆”与否这一概念%
做简单的讨论.
\begin{definition}[5B.3.1 热力学中的可逆反应]
    在一定条件下,正方向和逆方向都可以明显地进行的反应称为\tbf{可逆反应}.
\end{definition}
\indent 准确来说,如果在同一温度和压强下,我们既能在某一组成的反应体系中观察到$\xi$随时间明显地增大(即反应正向进行),%
又能在另一组成的反应体系中观察到$\xi$随时间明显地减小(即反应逆向进行),就认为反应是可逆反应.例如,%
在室温常压下向容器充入一定量\ce{N2O4},其颜色会随时间明显变淡,表明此时反应可以(显著地)向正方向进行;%
同样向容器中充入一定量\ce{NO2},其颜色又会迅速地变深,表明此时反应也可以(显著地)向逆方向进行.%
利用热力学与化学平衡的知识,我们可以知道在平衡时体系中总是有大量的\ce{N2O4}和\ce{NO2}存在.\\
\indent 因此,化学热力学要求的“可逆反应”是一个\tbf{相对的}概念,只要平衡时产物和反应物都有相当的量,%
就认为反应是可逆反应,因为我们总是能在系统只有反应物时观察到反应明显地向正向进行,%
也可以在系统只有产物时观察到反应明显地向逆方向进行.\\
\indent 基于这样的要求,我们也经常用标准平衡常数$K^\ominus$(也经常用其它各种平衡常数,这将在\tbf{5C}中介绍)衡量反应可逆与否.
\begin{theorem}[5B.3.2 用平衡常数衡量反应的可逆性]
    \tbf{一般而言},如果反应的平衡常数$K$不大于$10^5$,也不小于$10^{-5}$,我们就认为反应是可逆的.
\end{theorem}
这确实是一个非常粗糙而主观的判断方式.如果你把计量系数倍增或减半,平衡常数就会平方或开平方,%
因此同一个反应事实上有着可以任意改变的平衡常数.不过大部分时候,我们遇到的反应的计量系数都相当正常,%
不会太大也不会太小,因此\tbf{5B.3.2}的判断方式也还算有可取之处.\\
\indent 相对的,我们也可以给出不可逆反应的定义.
\begin{definition}[5B.3.3 (热力学中的)不可逆反应]
    在一定条件下,逆反应能进行的程度相对正反应相当小(以至于可以忽略不计)的反应称为\tbf{不可逆反应}.
\end{definition}
我们一般认为逆反应进行程度很小的反应为不可逆反应.如果一个反应与上述定义正好相反,那么这个反应也许称为不可发生的反应比较合适.%
不过交换它的反应物与产物就是一个不可逆反应,因此上面两种情况实际上是相同的.\\
\indent 在绝大多数情况下,不可逆反应和可逆反应一样,都是相对的概念.我们考虑下面的反应:
\begin{tightcenter}
    \ce{CH3C(OH)=CH2 <=> CH3COCH3}\ \ \ \ \ $K^\ominus=4.0\times 10^{5}$
\end{tightcenter}
这是你熟悉的丙酮的烯醇式向酮式的转化.它的平衡常数相当大,因此无论系统的组成如何,我们都很难观察到逆反应的显著发生,%
但这并不意味着产物丙酮完全无法转化为反应物丙烯醇.无论如何,体系中总是存在(尽管很少)反应物,这是不可改变的.\\
\indent 有时,你也会遇到平衡常数非常非常大的情形(例如$K>10^{100}$),以至于体系中反应物的分子数目可能小于$1$.%
不过,由于这样稀少的数目已经远远偏离了我们讨论的热力学,因此讨论此时反应物的量也许没有什么意义(甚至可能在达到理论浓度之前,普通的热力学规律就对它失效了).%
你只需要知道它很少很少即可.\\
\indent 然而,上面的“绝大多数情况”也是存在反例的.我们考虑下面的反应:
\begin{tightcenter}
    \ce{CaCO3(s) <=> CaO(s) + CO2(g)}\ \ \ \ \ $K^\ominus(800\tccentigrade)=1.16$
\end{tightcenter}
反应商$Q=\dfrac{p_{\ce{CO2}}}{p^\ominus}$.%
当系统(假定不含有其它气体)的压力小于$1.16p^{\ominus}$时,简单的计算表明体系中将不再含有\ce{CaCO3}.%
这里的不再含有是绝对意义上的,即热力学要求系统中不能存在\ce{CaCO3}.%
因此,在这样的压力条件下,反应是严格意义上不可逆的.
\begin{hint}
    造成这一现象的原因主要是我们一般讨论的体系都是混合物体系(无论气相还是液相),混合过程本身就会带来Gibbs自由能的减小.%
    例如,我们假定系统内没有产物$i$,这样$\mu_i=\mu_i^\ominus+RT\ln x_i$总是无穷小的,因此系统在此时总是会自发的向生成产物的方向变化.%
    而对于上面的纯相反应,没有物质的混合(\ce{CaCO3}和\ce{CaO}分属两个不同的固相),各组分的化学势在等温等压下是定值,不会因为组成的改变而改变,%
    故$p_{\ce{CO2}}<K^\ominus p^\ominus$时系统Gibbs自由能最低的情形就是不存在\ce{CaCO3}的情形\footnotemark.\\
    这一现象的产生可以与纯物质的相平衡做类比.当压力等于蒸气压时,液相和气相的组成比例可以任意改变,这时系统就处于相平衡;%
    而当压力大于蒸气压时,系统将只有液相;压力小于蒸气压时,系统将只有气相.这也是由于各相组成不变导致的.\\
    在有纯液相或纯固相参与的化学反应中,这样的现象是比较容易出现的.
\end{hint}\footnotetext{因此,对于这样的例外情形,笔者更倾向于认为这是热力学上的不可逆过程,而与我们一般讨论的不可逆反应有所区别%
(我们也许普遍地认为化学平衡应当发生在混合物体系中).%
这一过程的逆反应不是相对正反应可以忽略,而是在热力学上根本就不能发生.}
本节自此之后的部分,你可以在学习\tbf{Chapter 7}之后再来深入理解.\\
\indent 在化学动力学中,根据微观可逆性原理(你可以粗浅的理解为反应对时间反演\footnote{通俗的说就是时间倒流}的对称性),%
一个基本的正反应(即基元反应)必然有其对应的逆反应,而化学反应本身就由一系列基元反应构成,因此事实上所有反应在微观上都是可逆的.%
这样,我们可以给出可逆反应与不可逆反应的动力学定义.
\begin{definition}[5B.3.5 动力学中的可逆反应与不可逆反应]
    理论上所有化学反应都是可逆反应(更形象的说是\tbf{对峙反应}),但有时因为逆反应速率很慢而可以忽略,就认为这样的反应是\tbf{不可逆反应}.
\end{definition}
因此,在化学动力学中可逆反应是绝对的概念,不可逆反应则是相对的概念.\\
\indent 再次回到上面的\ce{CaCO3}热分解的反应上.即使在热力学上不可能存在\ce{CaCO3},%
在动力学上仍有\ce{CaO}和\ce{CO2}化合生成\ce{CaCO3},但此时\ce{CaCO3}的分解速率与生成速率相同,因此宏观上系统中没有\ce{CaCO3}的生成.\\
\indent 总结地说,热力学研究的是宏观的系统,是以平衡态统计物理学为基础的;%
动力学研究的是微观的系统,是以非平衡态统计物理学为基础的,因此两者看待反应的视角也有所不同.
\end{document}