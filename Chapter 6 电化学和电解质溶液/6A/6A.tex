\documentclass{ctexart}
\usepackage{PhysicalChemistryNote}

\begin{document}\pagestyle{plain}
\noindent\tbf{\LARGE 6A 氧化还原反应与可逆电池}\vspace{15pt}\\
\indent 不同于我们在\tbf{Chapter 5}中主要讨论的反应,在现实情况中,反应体系做非膨胀功(或者对体系做非膨胀功)也十分常见,%
这一非膨胀功即电功.为此,我们先需要了解氧化还原反应的基本概念.\vspace{12pt}\\
\Section{6A.1 氧化还原反应}
\Part{氧化还原反应}
\indent 氧化还原反应的概念始于我们对物质中各元素的化合价的考量.%
例如,一个很简单的氧化还原反应,锌溶于盐酸的反应,其离子反应方程式如下:
\begin{tightcenter}
    \ce{Zn(s) + 2H^+(aq) -> Zn^2+(aq) + H2(g)}
\end{tightcenter}
这一反应中\ce{Zn}失去$2$个电子,\ce{H^+}得到$1$个电子,产物中\ce{Zn}为$+2$价,\ce{H}为$0$价.%
更一般地,即使反应物或产物不是离子,而是共价化合物,我们也总是可以通过各元素氧化态的变化判断电子得失,%
因为一般而言氧化态的高低与对电子的亲和力相关.
\begin{hint}
    IUPAC规定的化合价(valence)和氧化态(oxidation state)是不同的概念.化合价的定义如下:\\
    \textit{The maximum number of univalent atoms (originally hydrogen or chlorine atoms) that may combine with an atom of the element under consideration, or with a fragment, or for which an atom of this element can be substituted.}\\
    翻译一下即为:“可以与所考虑元素的一个原子或基团相结合,或者可被该元素的一个原子取代的一价原子(最初指氢或氯原子)的最大数目.”因此,这一概念事实上指原子(或基团)最多能成的单键数目.\\
    氧化态的定义如下:\\
    \textit{The charge of this atom after ionic approximation of its heteronuclear bonds.}\\
    翻译一下即为:“对原子所成的异核键进行离子近似后该原子所带的电荷.”这才是我们一般而言提到的价态的概念.%
    在很多时候,化合价和氧化态都指的是后面这种情形,因此需要多加留意.
\end{hint}
\indent 在这里,我们给出氧化还原反应的正式定义.你应当在普通化学中已经了解过相关的概念.
\begin{definition}[6A.1.1 氧化还原反应]
    \tbf{氧化还原反应}是一种化学反应,在这种反应中,物质中的一个(或几个)元素的氧化态发生变化.%
    \tbf{氧化}是指失去电子或氧化态的增加,而\tbf{还原}则是指获得电子或氧化态的降低.\\
    得到电子的物质被称作\tbf{氧化剂},对应产物为\tbf{还原产物}.%
    失去电子的物质被称作\tbf{还原剂},对应产物为\tbf{氧化产物}.
\end{definition}
\Part{半反应与氧化还原电对}
既然任何一个氧化还原反应中都有氧化和还原两个部分.%
被氧化的物质失去电子,被还原的物质得到电子,这启示我们可以将它拆成两个反应(即\tbf{半反应}).
例如,上述氧化还原反应可以看成是两个半反应的总和结果:
\begin{tightcenter}
    \ce{Zn(s) - 2e^- -> Zn^2+(aq)}\\
    \ce{2H^+(aq) + 2e^- -> H2(g)}
\end{tightcenter}
被氧化的每个\ce{Zn}原子失去两个电子形成\ce{Zn^2+},被还原的每个\ce{H^+}得到一个电子并生成\ce{H2}.%
可以看到,两个反应相加之后,\ce{e^-}就被消去而不出现在总反应方程式中.
\begin{definition}[6A.1.2 半反应]
    任何氧化还原反应都可以被看作两个\tbf{半反应}的加和,%
    一个是氧化剂被还原的\tbf{还原半反应},另一个是还原剂被氧化的\tbf{氧化半反应}.
\end{definition}
出于统一性的考虑,我们也经常把氧化还原反应视作两个还原半反应的差.例如,上面的反应也可以看作
\begin{tightcenter}
    \ce{Zn^2+(aq) + 2e^- -> Zn(s)}\\
    \ce{2H^+(aq) + 2e^- -> H2(g)}
\end{tightcenter}
两个还原半反应的差.这样的写法也方便我们进行后面的讨论.
\begin{definition}[6A.1.3 氧化还原对]
    半反应中的还原物质\ce{Red}和氧化物质\ce{Ox}\footnotemark 形成\tbf{氧化还原对},通常记作\ce{Ox/Red},相应的还原半反应记作
    \begin{tightcenter}
        \ce{Ox + $v$e^- -> Red}
    \end{tightcenter}

\end{definition}\footnotetext{Red代表Reduction,意为“还原”;Ox代表Oxidation,意为“氧化”.}
所有的氧化还原反应都可以写作上面的形式.对于一些非氧化还原反应,也可以写作上面的形式,例如\ce{AgCl}的溶解:
\begin{tightcenter}
    \ce{AgCl(s) -> Ag^+(aq) + Cl^-(aq)}
\end{tightcenter}
就可以写作如下两个还原半反应之差:
\begin{tightcenter}
    \ce{AgCl(s) + e^- -> Ag(s) + Cl^-(aq)}\\
    \ce{Ag^+(aq) + e^- -> Ag(s)}
\end{tightcenter}
\vspace{8pt}
\Section{6A.2 原电池}
\Part{原电池}
\indent 我们在\tbf{3E.2.2}中指出封闭系统能做的最大非体积功等于系统Gibbs自由能减少的值.%
在那之后,我们给出了一个反应作为例子:
\begin{tightcenter}
    \ce{Zn(s) + Cu^2+(aq) -> Zn^2+(aq) + Cu(s)}
\end{tightcenter}
这一反应直接在烧杯中进行(即锌片与\ce{CuSO4}溶液发生置换反应)会放出热,这是化学能转化为热能的表现.%
如果我们再找一片铜片,并用导线将锌片和铜片相连,就会发现铜片的厚度不断增加,而锌片的厚度不断减少,%
这与直接发生置换时\ce{Cu}单质在锌片表面析出的现象有所不同.\\
\indent 上述实验现象指出,半反应
\begin{tightcenter}
    \ce{Zn(s) - 2e^- -> Zn^2+(aq)}\\
    \ce{Cu^2+(aq) + 2e^- -> Cu(s)}
\end{tightcenter}
分别发生在不同的位置.氧化半反应发生在锌片上,这一反应产生的电子经过导线传递到铜片上,%
在铜片上又发生了还原半反应.当然,如果不在导线上接用电器,最终化学能仍将转变为电能.%
如果你此时向这一回路中接入一个小灯泡,就会发现它被点亮了.通过让半反应在不同的地方发生并用导线连接,%
我们得到了一个电池.
\begin{definition}[6A.2.1 化学电池]
    \tbf{化学电池}由离子导体(可以是电解质溶液,电解质固体等)以及与离子导体接触的两个\tbf{电极}组成.%
    如果电池内部能自发地发生化学反应而产生电流,那么这样的化学电池称为\tbf{原电池};%
    如果电池需要外部电流驱动其发生非自发反应,那么这样的化学电池称为\tbf{电解池}.
\end{definition}
显然,上述电池属于原电池.\\
\indent 对于每一个氧化还原反应,都可以设计相应的电池使得两个半反应发生在电池的两个电极上.%
发生氧化半反应的电极给出电子,是原电池的负极;发生还原半反应的电极得到电子,是原电池的负极.%
我们对这一概念做推广.
\begin{definition}[6A.2.2 阳极与阴极]
    在化学电池中,发生氧化半反应的电极称为\tbf{阳极},发生还原半反应的电极称为\tbf{阴极}.
\end{definition}
对于原电池而言,其阳极为负极,阴极为正极\footnote{这与中文的语义有些冲突.}.%
这一概念主要是在讨论电解池时用到的.\\
\indent 我们可以对上述电池稍加改进.在我们的设计中,锌片直接与\ce{CuSO4}溶液接触,%
因此即使用导线将其与铜片相连,仍然不能完全防止\ce{Zn}直接与溶液中的\ce{Cu^2+}反应.%
为此,我们可以把锌片放在\ce{ZnSO4}溶液中,把铜片放在\ce{CuSO4}溶液中,然后将两种溶液用多孔陶瓷板(或半透膜)隔开,%
亦可以用充满\ce{KCl}溶液的琼脂(即我们所说的\tbf{盐桥})进行连接.%
这样,\ce{Cu^2+}就不会进入与\ce{Zn}相接的电解液,也就避免了上面的情形的发生.
\end{document}