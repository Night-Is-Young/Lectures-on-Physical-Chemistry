\documentclass{ctexart}
\usepackage{PhysicalChemistryNote}

\begin{document}\pagestyle{plain}
\noindent\tbf{\LARGE 6A 氧化还原反应与可逆电池}\vspace{15pt}\\
\indent 我们在\tbf{Chapter 5}讨论的反应体系都只做体积功,因此可以用$\Delta_\r G$的大小直接判断反应进行的方向.%
在现实情况中,反应体系做非膨胀功(或者对体系做非膨胀功)也十分常见,在化学电池中发生的反应就是如此,这一非膨胀功即电功.%
为此,我们先需要了解氧化还原反应的基本概念.\vspace{12pt}\\
\Section{6A.1 氧化还原反应}
\Part{氧化还原反应}
\indent 氧化还原反应的概念始于我们对物质中各元素的化合价的考量.%
例如,一个很简单的氧化还原反应如下:
\begin{tightcenter}
    \ce{2H2 + O2 -> 2H2O}
\end{tightcenter}
在这一反应中,\ce{H2}失去$2$个电子,\ce{O2}得到$4$个电子,产物\ce{H2O}中\ce{H}为$+1$价,\ce{O}为$-2$价.%
一般而言,我们总是可以通过化合价(或氧化态)判断电子得失,因为它反映了电负性大小,进而与对电子的亲和力相关.\\
\indent 我们给出氧化还原反应的定义.
\begin{definition}[6A.1.1 氧化还原反应]
    \tbf{氧化还原反应}是一种化学反应,在这种反应中,物质中的一个(或几个)元素的氧化态发生变化.%
    \tbf{氧化}是指失去电子或氧化态的增加,而\tbf{还原}则是指获得电子或氧化态的降低.
\end{definition}
既然任何一个氧化还原反应中都有氧化和还原两个部分,那么它们就可以看作是两个半反应之和.
例如,上述氧化还原反应可以看成是两个半反应的总和结果:
\end{document}