\documentclass{ctexart}
\usepackage{PhysicalChemistryNote}

\begin{document}\pagestyle{plain}
\noindent\tbf{\LARGE 6C 电化学势与电极电势}\vspace{15pt}\\
\indent 氧化还原反应中的半反应和电池的电极让我们可以单独地对一个氧化还原电对进行研究.%
正如物质有其氧化性强弱之分,电极的电势也有高低之分.在研究电极的电势之前,%
我们需要引入电化学势这一概念对电极作更深刻的了解.\vspace{12pt}\\
\Section{6C.1 电化学势}
\Part{电化学势}
\indent 一直以来,我们研究的系统中所有组分均为电中性的.而在电池中,大部分粒子都是带电粒子,%
这意味着它们的转移与变化不仅与它在某一处化学势$\mu$有关,也与该处的电势$\phi$有关.%
因此,我们需要综合考虑上述两种情形,并由此定义\tbf{电化学势}.
\begin{definition}[6C.1.1 电化学势]
    带电粒子$i$在$\alpha$处的\tbf{电化学势}$\overline{\mu}_i^\alpha$,定义为$i$在此处的摩尔电势能与化学势之和,即
    \[\overline{\mu}_i^\alpha=\mu_i^\alpha+z_iF\phi^\alpha\]
    其中$\mu_i^\alpha$为$i$在$\alpha$处的化学势,$z_i$为$i$的电荷数,$F$为Faraday常数,$\phi^\alpha$为$\alpha$处的电势.
\end{definition}
我们现在来说明这一定义(即简单相加)的合理性.
\begin{derivation}
    我们考虑将物质的量为$\di n_i$的带电粒子$i$从系统的$\alpha$处移动到系统的$\beta$处.%
    系统的Gibbs自由能变化为
    \[\di G=\di G^\alpha+\di G^\beta=-\mu_i^\alpha\di n_i+\mu_i^\beta\di n_i=\left(\mu_i^\beta-\mu_i^\alpha\right)\di n_i\]
    这一过程做的电功(注意功的符号,正电荷从电势低处移向电势高处时系统做负功)为
    \[\delta W_e=-\Delta\phi\di Q=\left(\phi^\alpha-\phi^\beta\right)z_ie\NA\di n_i=\left(\phi^\alpha-\phi^\beta\right)z_iF\di n_i\]
    根据\tbf{3E.2.2},如果$\di G<\delta W_e$,那么这一转移就是自发的,即
    \[\mu_i^\beta+z_iF\phi^\beta<\mu_i^\alpha+z_iF\phi^\alpha\]
    因此,我们把$\mu_i^\alpha+z_iF\phi^\alpha$定义为$i$在$\alpha$处的电化学势,%
    并由此可得等温等压下带电粒子$i$总是向电化学势小的地方迁移.因此,在研究含有带电粒子的系统时,%
    我们可以用电化学势代替原来的化学势对系统的热力学性质进行研究.
\end{derivation}
由此,我们可以得出与\tbf{4B.3.4}相似的结论.
\begin{definition}[6C.1.2 含带电粒子的多组分体系自发变化的方向]
    等温等压下的多组分封闭体系中,任意组分总是从电化学势大的地方转移至电化学势小的地方,%
    直到该组分在系统各处的电化学势相等.
\end{definition}
我们现在再次考虑两种金属接触时产生接触电势的现象.尽管我们已经知道,%
两种金属的界面两侧确实存在电势差,但电压表测得的数据却是$0$.对于这一点,我们给出以下结论.
\begin{theorem}[6C.1.3 电压表的实际测定值]
    电压表实际上测定的是两点之间电子的电化学势之差.
\end{theorem}
我们以\ce{Ti}和\ce{Au}接触为例.电子在两种金属中的化学势,%
可以由其逸出功\footnote{源于Albert Einstein对光电效应的解释,即电子从固体内部移到固体外部所必需的最小能量.%
固体的逸出功与其Fermi能级密切相关.}衡量.%
\ce{Ti}的逸出功为$4.33\text{ eV}$,即电子在\ce{Ti}中的化学势为$-418\kJm$,%
而\ce{Au}的逸出功为$5.10\text{ eV}$,电子在\ce{Au}中的化学势为$-492\kJm$.%
显然,电子在\ce{Au}中的化学势更低.%
当两者接触时,电子发生转移,直至各处的$\overline{\mu}_\e$相同.%
此时尽管$\phi^{\ce{Ti}}<\phi^{\ce{Au}}$,电压表却无法给出这一电势的具体值,%
而只能给出电子的电化学势之差.\\
\indent 由此,我们可以更明确地定义电极电势的概念.\vspace{4pt}\\
\Part{电极电势}
\indent 类比电子在两种不同的金属形成的界面附近的分布,%
我们可以考虑电极和电解质溶液之间各种带电粒子的电化学势.
\begin{derivation}
    我们先考虑氧化还原电极.假定电极半反应为\ce{Ox(aq) + $\nu$e^- <=> Red(aq)}.\\
    由于电荷总是守恒的,因此半反应前后粒子的电势能不变,因此电化学势仅与化学势有关.%
    于是在发生上述半反应时,我们可以将化学平衡的条件应用于此,%
    即平衡时反应两边的物质的电化学势相同.这就有
    \[\]
\end{derivation}
\end{document}