\documentclass{ctexart}
\usepackage{PhysicalChemistryNote}

\begin{document}\pagestyle{plain}
\noindent
\Section{6D.3 标准电极电势与Nernst方程的应用}
\Part{由电极电势计算电池电动势并判断反应方向}
\indent 由$\varphi^\ominus_{\ce{Ox/Red}}$可以计算反应的标准电动势,%
进而得出反应的平衡常数,并判断反应的进行方向.
\begin{exercise}[E.6D.1]
    已知$T=298.15\K$下有
    \[\varphi^\ominus_{\ce{MnO4^-/Mn^2+}}=+1.51\text{ V}\ \ \ \ \ 
    \varphi^\ominus_{\ce{CO2/H2C2O4}}=-0.49\text{ V}\]
    \begin{enumerate}[topsep=0pt,parsep=0pt,itemsep=0pt,partopsep=0pt,label=\tbf{\arabic*},leftmargin=*]
        \item 求\ce{CO2/H2C2O4}作为阳极,\ce{MnO4^-/Mn^2+}作为阴极组成原电池的标准电池电动势.
        \item 求反应\ce{2MnO4^- + 5H2C2O4 + 6H^+ <=> 2Mn^2+ + 10CO2 + 8H2O}的标准平衡常数$K^\ominus$.
    \end{enumerate}   
\end{exercise}
为了解决这一问题,我们先需要明确电极反应中\ce{H^+}的状态.事实上,\tbf{E.6D.1}中的两个反应均是在酸性条件下进行的,%
为了符合\tbf{6D.2.3}中所有物质处于标准态的定义,这里的\ce{H^+}也应当处于标准态,尽管它在氧化还原电对中并没有出现.%
一般而言,如果可以判断半反应发生的酸碱条件,则半反应中对应的\ce{H^+}或\ce{OH^-}也处于标准态.%
在其它情况下,也会特别注明给出的标准电极电势是酸性条件下还是碱性条件下的.\\
\indent 我们现在先来解决第一个问题,即求电池的标准电动势.
\begin{solution}
    根据\tbf{6D.2.3},$\varphi^\ominus_{\ce{MnO4^-/Mn^2+}}=+1.51\text{ V}$意味着\ce{MnO4^-/Mn^2+}与SHE构成的电池的标准电动势为$E_1^\ominus=1.51\text{ V}$.%
    根据\tbf{6C.2.1}Nernst方程,反应
    \begin{tightcenter}
        \ce{2MnO4^- + 5H2 + 6H^+ -> 2Mn^2+ + 8H2O}
    \end{tightcenter}
    的$\Delta_\r G_{\m,1}^\ominus=-\nu_1 FE_1^\ominus=-\nu_1F\varphi_{\text{right}}^\ominus$.%
    下标right表示这一电极作为阴极(在电池表示中写在右边),$\nu_1$为这一反应转移的电子的计量数.\\
    同样地,$\varphi^\ominus_{\ce{CO2/H2C2O4}}=-0.49\text{ V}$意味着反应
    \begin{tightcenter}
        \ce{2CO2 + H2 -> H2C2O4}
    \end{tightcenter}
    的$\Delta_\r G_{\m,2}^\ominus=-\nu_2 FE_2^\ominus=-\nu_2F\varphi_{\text{left}}^\ominus$.\\
    因此对于题设的反应,根据Hess定律可得
    \[\begin{aligned}
        \Delta_\r G_\m^\ominus
        &= \dfrac{\nu}{\nu_1}\Delta_\r G_{\m,1}^\ominus-\dfrac{\nu}{\nu_2}\Delta_\r G_{\m,2}^\ominus \\
        &= \nu F\left(\varphi_{\text{left}}^\ominus-\varphi_{\text{right}}^\ominus\right)
    \end{aligned}\]
    再根据Nernst方程,这一电池的标准电动势
    \[E^\ominus=-\dfrac{\Delta_\r G_\m^\ominus}{\nu F}=\varphi_{\text{right}}^\ominus-\varphi_{\text{left}}^\ominus\]
    代入题中的数据可得
    \[E^\ominus=2.00\text{ V}\]

\end{solution}
上面的推导过程对任意两个电极组成的电池都是适用的.由此,我们可以得出下面的结论.
\begin{theorem}[6D.3.1 任意电池的电动势]
    任意电池的标准电池电动势等于其阴极与阳极的标准电极电势之差,即
    \[E^\ominus=\varphi_{\text{right}}^\ominus-\varphi_{\text{left}}^\ominus\]
    推而广之,任意电池的电动势等于其阴极和阳极的电极电势之差,即
    \[E=\varphi_{\text{right}}-\varphi_{\text{left}}\]

\end{theorem}
后面的推广也是易证的,只需在上面的推导过程中把$\varphi^\ominus$换成对应组成下的$\varphi$,%
$\Delta_\r G_\m^\ominus$换成对应组成下的$\Delta_\r G_\m$即可.不论是否是标准态,Hess定律和Nernst方程都成立.\\
\indent 知道标准电池电动势后,计算对应反应的热力学数据就十分容易了.
\begin{solution}
    根据\tbf{6C.2.1}有
    \[K^\ominus=\exp\left(\dfrac{\nu FE^\ominus}{RT}\right)=1.2\times10^{338}\]
    反应的平衡常数十分大,这意味着这个反应进行得很彻底.
\end{solution}
\begin{hint}
    如果在普通的计算器上直接计算上式,则会由于数据太大而报错.%
    因此,我们需要将科学计数法的底数和指数分开计算.\\
    我们假定最终的答案为$a\times10^{b}$,其中$1\leqslant a<10$,$b$是整数.于是
    \[\exp\left(\dfrac{\nu FE^\ominus}{RT}\right)=a\cdot10^b\]
    两边对$10$取对数即可得
    \[\dfrac{\nu FE^\ominus}{RT}\cdot\lg\e=b+\lg a\]
    因此,$b$就等于左边的式子向下取整,$\lg a$就等于左边的式子与$b$之差.\\
    简单的计算表明此时左边的式子等于$338.07$,因此
    \[a=10^{0.07}=1.2\ \ \ \ \ b=338\]
    于是就可以得出最终的答案.
\end{hint}
上述这一反应实际上就是\ce{MnO4^-}氧化\ce{H2C2O4}的反应,而前者对应的电极电势高于后者,%
因此反应正向进行的程度非常大——这也意味着\ce{MnO4^-}(在酸性条件下)的氧化性强于\ce{CO2}%
(对应的还原产物为\ce{H2C2O4}).%
因此,电极电势也可以作为氧化性的量度.
\begin{theorem}[6D.3.2 电极电势与氧化性]
    物质作为氧化剂对应的半反应的电极电势越高,该物质的氧化性越强.\\
    物质作为还原剂对应的半反应的电极电势越低,该物质的还原性越强.
\end{theorem}
这一结论可以容易地由\tbf{6D.3.1}和\tbf{6C.2.1}推知.$\varphi$越高,%
一种物质作为阴极(即作为氧化剂)与其它物质组成电池的电动势$E^\ominus$越大,$\Delta_\r G_\m^\ominus$就越大,%
表明这一氧化反应容易发生,即该物质的氧化性强.还原性亦是同理.\\
\indent 上面的结论启示我们可以通过列表来方便地判断各物质氧化性或还原性的大小.%
这就是各种化学书附录中常见的\tbf{标准电极电势表}(一般分为酸性和碱性两张表).%
查阅此表,不仅能获知各种电极的标准电极电势,还能根据它们的顺序方便地比较各物质氧化性和还原性.\vspace{4pt}\\
\Part{pH的测定}
\indent 理论上,想要测定某溶液的pH,只需将标准氢电极中的溶液换成待测溶液后测定其电极电势,%
就可以根据Nernst方程求出此时的pH.然而,如前所述,由于标准氢电极对条件的要求较高,
因此实际情况中更多地采用玻璃电极测定.\\
\indent 玻璃电极的主要部分是一个玻璃泡,泡的下半部是对\ce{H^+}有选择透过的玻璃薄膜,泡内装有一定浓度的\ce{HCl}溶液,其中插入一支Ag-AgCl电极(称为\tbf{内参比电极}),这样就构成了玻璃电极.%
将玻璃电极浸入待测溶液中,测定其电动势$\varphi$,溶液的pH与$\varphi$满足
\[\varphi=\varphi^\ominus+\dfrac{RT}{F}\ln\dfrac{1}{a_{\ce{H^+}}}=\varphi^\ominus-\dfrac{RT\ln10}{F}\cdot\text{pH}\]
其中$\varphi^\ominus$为玻璃电极浸入$a_{\ce{H^+}}=1$的\ce{H^+}溶液时的电极电势.这取决于玻璃电极的具体组成.%
实际操作中,常配制已知pH的缓冲溶液,将玻璃电极浸入后按上式校准,从而简化操作,不必再测量$\varphi^\ominus$.\\
\indent 由玻璃电极和外参比电极(例如甘汞电极等),以及特制的测定电势的装置构成的测定pH的仪器就是\tbf{pH计}.%
你也许在以后的实验中会经常地用到这样的仪器,它兼顾了一定的准确性和便捷性.\\
\begin{derivation}
    玻璃电极中内参比电极的电位是恒定的,与待测溶液的pH无关.它之所以能测定溶液pH,是由于两侧\ce{H^+}活度不同而造成了玻璃膜上产生膜电势.膜的结构大致如下.
    \vspace{-5pt}\begin{table}[H]\centering
        \begin{tabular}{ccccccccccc}
            内参比电极 &$\vert$ &标准溶液   &$\vert$ &内溶胀层 &$\vert$    &干玻璃 &$\vert$    &外溶胀层    &$\vert$    &待测溶液\\
            &&$a_{\ce{H^+},\text{in}}$ &&$a_{\ce{H^+},\text{in}}'$&&&&$a_{\ce{H^+},\text{out}}'$&&$a_{\ce{H^+},\text{out}}$
        \end{tabular}
    \end{table}\vspace{-15pt}
    溶胀层是由于溶液中的\ce{H^+}与玻璃中的阳离子(如\ce{Na^+})交换而形成的.达成平衡后,%
    溶胀层中的金属离子基本全部被\ce{H^+}替换,因此$a_{\ce{H^+},\text{in}}'=a_{\ce{H^+},\text{out}}'$.我们考虑内外溶液与各自接触的溶胀层的接界电势,有
    \[\varphi_{\text{in}}=\phi+\dfrac{RT}{F}\ln\dfrac{a_{\ce{H^+},\text{in}}}{a_{\ce{H^+},\text{in}}'}\]
    \[\varphi_{\text{out}}=\phi+\dfrac{RT}{F}\ln\dfrac{a_{\ce{H^+},\text{out}}}{a_{\ce{H^+},\text{out}}'}\]
    其中$\phi$是与玻璃膜性质有关的常数.对于同一玻璃膜,两侧的$\phi$应当相同.这样,膜电势
    \[\varphi_{\text{mem}}=\varphi_{\text{out}}-\varphi_{\text{in}}=\dfrac{RT}{F}\ln\dfrac{a_{\ce{H^+},\text{in}}}{a_{\ce{H^+},\text{out}}}\]
    总的电极电势即为
    \[\varphi=\varphi_{\text{ref}}+\varphi_{\text{mem}}=\varphi^\ominus+\dfrac{RT}{F}\ln\dfrac{1}{a_{\ce{H^+},\text{out}}}\]
    其中,由于标准溶液的$a_{\ce{H^+}}$为恒定值,因此内参比电极的电极电势$\varphi_{\text{ref}}$和后面的$\dfrac{RT}{F}\ln a_{\ce{H^+},\text{in}}$均为定值,%
    将其合并即可得到$\varphi^\ominus$.
\end{derivation}
\Part{Latimer图和Frost图}
\indent 一种元素可能具有多个氧化态,这些不同氧化态的物质相互转化时对应的的电极电势也有所不同.%
\tbf{Latimer图}把同种元素不同氧化态间的标准电极电势按照氧化数由高到低的顺序排成图,可以方便地分析元素处于各氧化态时的氧化还原性质.\\
\indent 我们以\ce{Mn}在酸性条件下的Latimer图为例介绍其用法.
\begin{figure}[H]
    \centering\documentclass{standalone}
\usepackage{PhysicalChemistryNote}
\begin{document}
\begin{tikzpicture}[scale=0.8]
    \node[scale=0.8] at (0,0) {\ce{MnO4-}};
    \node[scale=0.8] at (3,0) {\ce{MnO4^2-}};
    \node[scale=0.8] at (6,0) {\ce{MnO2}};
    \node[scale=0.8] at (9,0) {\ce{Mn^3+}};
    \node[scale=0.8] at (12,0) {\ce{Mn^2+}};
    \node[scale=0.8] at (15,0) {\ce{Mn}};
    \draw[-] (0,0.3)--(0,1)--(12,1);
    \draw[-] (0,-0.3)--(0,-1)--(5.8,-1);
    \draw[-] (6.2,-0.3)--(6.2,-1)--(12,-1);
    \draw[-latex] (12,1)--(12,0.3);
    \draw[-latex] (5.8,-1)--(5.8,-0.3);
    \draw[-latex] (12,-1)--(12,-0.3);
    \draw[-latex] (0.65,0)--(2.25,0);
    \draw[-latex] (3.65,0)--(5.25,0);
    \draw[-latex] (6.65,0)--(8.25,0);
    \draw[-latex] (9.65,0)--(11.25,0);
    \draw[-latex] (12.65,0)--(14.25,0);
    \node[scale=0.8] at (6,1.3) {$1.507\text{ V}$};
    \node[scale=0.8] at (3,-0.7) {$1.696\text{ V}$};
    \node[scale=0.8] at (9,-0.7) {$1.224\text{ V}$};
    \node[scale=0.8] at (1.5,0.3) {$0.558\text{ V}$};
    \node[scale=0.8] at (4.5,0.3) {$2.265\text{ V}$};
    \node[scale=0.8] at (7.5,0.3) {$0.907\text{ V}$};
    \node[scale=0.8] at (10.5,0.3) {$1.541\text{ V}$};
    \node[scale=0.8] at (13.5,0.3) {$-1.185\text{ V}$};
\end{tikzpicture}
\end{document}
\end{figure}
\begin{theorem}[6D.3.3 Latimer图的画法]
    绘制Latimer图按照以下规则.
    \begin{enumerate}[topsep=0pt,parsep=0pt,itemsep=0pt,partopsep=0pt,leftmargin=*,label=\tbf{\arabic*.}]
        \item 各个物质按照目标元素的氧化数从高到低依次从左向右排列.
        \item 箭头上的电位表示含有该箭头两端物质对应半反应的标准电极电势,如上图中\ce{MnO4^-}指向\ce{Mn^2+}的箭头上的$1.507\text{ V}$代表半反应
            \begin{tightcenter}
                \ce{MnO4^- + 8H^+ + 5e^- -> Mn^2+ + 4H2O}
            \end{tightcenter}
            的标准电极电势为$1.507\text{ V}$.
        \item Latimer图有酸式和碱式的区别,分别对应酸性条件下和碱性条件下的标准电极电势.%
            除了电极电势会因pH改变而改变外,对应氧化态的物种也会因酸碱性的改变而改变.%
            例如,碱性条件下\ce{Mn}的Latimer图中\ce{Mn^{II}}对应的物种是\ce{Mn(OH)2}而非\ce{Mn^2+}.
    \end{enumerate}
\end{theorem}
我们现在来简要介绍,Latimer图除了查阅标准电极电势以外的其它功能.
\begin{theorem}[6D.3.4 Latimer图的功能]
    Latimer图有以下功能.
    \begin{enumerate}[topsep=0pt,parsep=0pt,itemsep=0pt,partopsep=0pt,leftmargin=*,label=\tbf{\arabic*.}]
        \item \tbf{直观判断氧化性强弱}.电极电势越低,低价态的物质越容易被氧化;电极电势越高,高价态的物质越容易被还原.%
            这与\tbf{6D.3.2}的结论一致.
        \item \tbf{从相邻电对的电极电势求另一电对的电极电势}.我们以\ce{MnO4-/MnO2}的电极电势为例,考虑反应
            \begin{tightcenter}
                \tbf{1.} \ce{MnO4^- + 5/2H2 + 3H^+ -> Mn^2+ + 4H2O}\\
                \tbf{2.} \ce{MnO4^- + 1/2H2 -> MnO4^2- + H^+}\\
                \tbf{3.} \ce{MnO4^2- + 2H2 + 4H^+ -> Mn^2+ + 4H2O}
            \end{tightcenter}
            可以看到$\tbf{1}=\tbf{2}+\tbf{3}$,即$\Delta_\r G_{\m,1}^\ominus=\Delta_\r G_{\m,2}^\ominus+\Delta_\r G_{\m,1}^\ominus$.于是有
            \[\varphi_1^\ominus=-\dfrac{\Delta_\r G_{\m,1}^\ominus}{\nu_1F}
            =-\dfrac{\Delta_\r G_{\m,2}^\ominus+\Delta_\r G_{\m,3}^\ominus}{\nu_1F}
            =\dfrac{\nu_2\varphi_2^\ominus+\nu_3\varphi_3^\ominus}{\nu_2+\nu_3}\]
            代入数据即可求得.一般地,对于指向同样起始物质和终点物质的箭头(一个或数个),都有$\displaystyle\sum\nu_i\varphi_i^\ominus$为定值.%
            其中$\nu_i$为各个箭头对应反应转移电子的计量数(需要注意目标元素守恒),$\varphi_i^\ominus$则为对应的标准电极电势.
        \item \tbf{判断歧化/归中反应的可能性}.对于一个物种\ce{X},记它左边的电势为$\varphi_{\text{left}}^\ominus$,$\varphi_{\text{right}}^\ominus$.%
            如果$\varphi_{\text{left}}^\ominus>\varphi_{\text{right}}^\ominus$,那么左右两边的物质会发生归中反应生成\ce{X};%
            如果$\varphi_{\text{left}}^\ominus<\varphi_{\text{right}}^\ominus$,那么\ce{X}会发生歧化反应生成两边的物质.\\
            以上图为例,不难发现在此条件下\ce{MnO4^2-}会歧化生成\ce{MnO2}和\ce{MnO4^-},而\ce{Mn^3+}和\ce{Mn}则会归中生成\ce{Mn^2+}.
    \end{enumerate}
\end{theorem}
为了更直观地判断各氧化态物质的稳定性,我们也可以采用\tbf{Frost图}.%
我们仍以\ce{Mn}在酸性条件下的Frost图为例介绍其用法.
\begin{figure}[H]
    \centering\documentclass{standalone}
\usepackage{PhysicalChemistryNote}
\begin{document}
\begin{tikzpicture}
    \draw[step=1,help lines,dashed] (0,-3) grid (7,6);
    \draw[thick,-latex] (0,-3)--(7.25,-3);
    \draw[thick,-latex] (0,-3)--(0,6);
    \draw[-] (0,0)--(2,-2.37)--(3,-0.829)--(4,0.078)--(6,4.608)--(7,5.166);
    \draw[fill=white,draw=blue] (0,0) circle(0.05) node[above right]{\ce{Mn}};
    \draw[fill=white,draw=blue] (2,-2.37) circle(0.05) node[below]{\ce{Mn^2+}};
    \draw[fill=white,draw=blue] (3,-0.829) circle(0.05) node[below right]{\ce{Mn^3+}};
    \draw[fill=white,draw=blue] (4,0.078) circle(0.05) node[below right]{\ce{MnO2}};
    \draw[fill=white,draw=blue] (6,4.608) circle(0.05) node[left]{\ce{MnO4^2-}};
    \draw[fill=white,draw=blue] (7,5.166) circle(0.05) node[above]{\ce{MnO4^-}};
    \node[below] at (3.5,-3) {Oxidation State};
    \node[above] at (0,6) {$\nu\varphi^\ominus$};
\end{tikzpicture}
\end{document}
\end{figure}
\begin{theorem}[6D.3.5 Frost图的画法]
    绘制Frost图按照以下规则.
    \begin{enumerate}[topsep=0pt,parsep=0pt,itemsep=0pt,partopsep=0pt,leftmargin=*,label=\tbf{\arabic*.}]
        \item 稳定单质(0价状态)处于横纵轴的原点处.
        \item 横轴为氧化数,从左至右增大(这与Latimer图正好相反).
        \item 纵轴为氧化数$\nu$与该物质和0价单质的电极电势$\varphi^\ominus$的乘积.
    \end{enumerate}
\end{theorem}
Frost图事实上可以看作Latimer图的图形版.我们现在简要介绍其功能.
\begin{theorem}[6D.3.6 Frost图的功能]
    Frost图有以下功能.
    \begin{enumerate}[topsep=0pt,parsep=0pt,itemsep=0pt,partopsep=0pt,leftmargin=*,label=\tbf{\arabic*.}]
        \item \tbf{直观判断氧化性强弱}.标准电极电势越高,连接两个物质的直线的斜率越大;标准电极电势越小,连接两个物质的直线的斜率(可以为负值)越小.\\
            这一点可以由Frost图的定义和简单的Nernst方程的应用而得出.
        \item \tbf{判断歧化/归中反应的可能性}.图线上凸处的物质易发生歧化,下凹处的物质的两边易发生归中而得到该物质.\\
            这一点可以由\tbf{1.}和\tbf{6D.3.4}中的\tbf{3.}共同得出.
    \end{enumerate}
\end{theorem}
\Part{$\varphi-\text{pH}$图(Pourbaix图)}
\indent 我们在前面已经提到,大多数半反应的电极电势与pH密切相关.如果把这些反应的电极电势与pH绘制成图,%
就可以方便地判断一定的pH条件下各种电极反应进行的顺序.\\
\indent 我们以\ce{Fe-H2O}体系为例,假定体系中除了\ce{H^+}(以及与此相关的\ce{OH^-})外的所有物质均处于标准态.首先是\ce{H2O}本身对应的两个电极反应,即
\begin{tightcenter}
    \ce{2H^+ + 2e^- -> H2}\\
    \ce{O2 + 4H^+ + 4e^- -> 2H2O}
\end{tightcenter}
考虑电极反应的Nernst方程有
\[\varphi_{\ce{H^+/H2}}=\varphi_{\ce{H^+/H2}}^\ominus+\dfrac{RT}{2F}\ln a_{\ce{H^+}}^2=\left(-0.05916\text{ pH}\right)\text{V}\]
\[\varphi_{\ce{O2/H2O}}=\varphi_{\ce{O2/H2O}}^\ominus+\dfrac{RT}{4F}\ln a_{\ce{H^+}}^4=\left(1.229-0.05916\text{ pH}\right)\text{V}\]
这一关系即使在碱性条件下也是成立的.你可以自行验证之.\\
\indent 现在再来考虑\ce{Fe},\ce{Fe^2+}和\ce{Fe^3+}之间的转化.我们有
\begin{tightcenter}
    \ce{Fe^2+ + 2e^- -> Fe}\ \ \ \ \ $\varphi_{\ce{Fe^2+/Fe}}^\ominus=-0.447\text{ V}$\\
    \ce{Fe^3+ + e^- -> Fe^2+}\ \ \ \ \ $\varphi_{\ce{Fe^3+/Fe^2+}}^\ominus=0.771\text{ V}$
\end{tightcenter}
这两个电极电势都是与pH无关的.在图中,它们呈现为水平的直线.\\
\indent 在碱性条件(或酸性没那么强的条件)下,\ce{Fe^2+}和\ce{Fe^3+}分别会转化成各自对应的氢氧化物.考虑到
\begin{tightcenter}
    \ce{Fe(OH)2 <=> Fe^2+ + 2OH^-}\ \ \ \ \ $K_{\text{sp},\ce{Fe(OH)2}}^\ominus=4.87\times10^{-17}$\\
    \ce{Fe(OH)3 <=> Fe^3+ + 3OH^-}\ \ \ \ \ $K_{\text{sp},\ce{Fe(OH)2}}^\ominus=2.79\times10^{-39}$
\end{tightcenter}
这两个反应不是氧化还原反应,因此在图中呈现为竖直的直线.由于各物种(除了\ce{OH^-})均处于标准态,因此上述反应分别对应的pH为$5.844$和$1.148$.\\
\indent 再来考虑\ce{Fe},\ce{Fe(OH)2}和\ce{Fe(OH)3}之间的转化.有了前面的经验,你可以很容易地写出
\begin{tightcenter}
    \ce{Fe(OH)2 + 2H^+ + 2e^- -> Fe + 2H2O}\ \ \ \ \ $\varphi_{\ce{Fe(OH)2/Fe}}^\ominus=\left(0.0464-0.05916\text{ pH}\right)\text{V}$\\
    \ce{Fe(OH)3 + H^+ + e^- -> Fe(OH)2 + H2O}\ \ \ \ \ $\varphi_{\ce{Fe(OH)3/Fe(OH)2}}^\ominus=\left(0.284-0.0592\text{ pH}\right)\text{ V}$
\end{tightcenter}
尽管这些反应大多数时候在碱性条件下进行,但我们仍把反应物写为\ce{H^+}方便讨论.\\
以及,由于\ce{Fe(OH)2}和\ce{Fe(OH)3}存在的pH范围不同,于是还有\ce{Fe^2+}与\ce{Fe(OH)3}之间的转化,即
\begin{tightcenter}
    \ce{Fe(OH)3 + 3H^+ + e^- -> Fe^2+ + 3H2O}\ \ \ \ \ $\varphi_{\ce{Fe(OH)3/Fe^2+}}^\ominus=\left(1.271-0.1775\text{ pH}\right)\text{V}$\\
\end{tightcenter}
于是就可以画出\ce{Fe-H2O}体系的$\varphi-\text{pH}$图,如下所示.
\begin{figure}[H]
    \centering\documentclass{standalone}
\usepackage{PhysicalChemistryNote}
\begin{document}
\begin{tikzpicture}[scale=0.5]
    \draw[thick] (-2,-10)--(14,-10)--(14,14)--(-2,14)--(-2,-10);

    \draw[thick,dashed] (-2,1.1832)--(14,-8.2824);
    \draw[thick,dashed] (-2,13.4732)--(14,4.0076);

    \draw[thick,red] (-2,7.71)--(2.82,7.71);
    \draw[thick,red] (-2,-4.47)--(8.34,-4.47);

    \draw[thick,blue] (2.82,7.71)--(2.82,14);
    \draw[thick,blue] (8.34,-4.47)--(8.34,-2.0939);

    \draw[thick,violet] (2.82,7.71)--(8.34,-2.0939);
    \draw[thick,violet] (8.34,-2.0939)--(14,-5.4424);
    \draw[thick,violet] (8.34,-4.47)--(14,-7.8184);

    \node at (0.41,10.8) {\ce{Fe^3+}};
    \node at (2,3) {\ce{Fe^2+}};
    \node at (10,3) {\ce{Fe(OH)3}};
    \node at (11.4,-5) {\ce{Fe(OH)2}};
    \node at (6,-7.5) {\ce{Fe}};
    \node at (3.5,-1.2) {\ce{H2O}};
    \node at (2.5,-2.2) {\ce{H2}};
    \node at (8,6.6) {\ce{H2O}};
    \node at (9,7.6) {\ce{O2}};


    \draw (0,-9.8)--(0,-10) node[below]{$0$};
    \draw (2,-9.8)--(2,-10) node[below]{$2$};
    \draw (4,-9.8)--(4,-10) node[below]{$4$};
    \draw (6,-9.8)--(6,-10) node[below]{$6$};
    \draw (8,-9.8)--(8,-10) node[below]{$8$};
    \draw (10,-9.8)--(10,-10) node[below]{$10$};
    \draw (12,-9.8)--(12,-10) node[below]{$12$};
    \node[below] at (14,-10) {$14$};
    \draw (-1.8,0)--(-2,0) node[left]{$0$};
    \draw (-1.8,5)--(-2,5) node[left]{$0.5$};
    \draw (-1.8,-5)--(-2,-5) node[left]{$-0.5$};
    \draw (-1.8,10)--(-2,10) node[left]{$1.0$};
    \draw (-1.8,-10)--(-2,-10) node[left]{$1.0$};
    \node[rotate=90] at (-4,2) {$\varphi/\text{V}$};
    \node at (6,-12) {pH};
\end{tikzpicture}
\end{document}
\end{figure}
如果给定\ce{Fe^2+},\ce{Fe^3+}等的活度,那么图线可能发生平移.\\
\indent 我们现在来简要介绍一下$\varphi-\text{pH}$图的功能.
\begin{theorem}[6D.3.7 $\varphi-\text{pH}$图的功能]
    $\varphi-\text{pH}$图主要用于判断某pH条件下的氧化还原反应发生的可能性:%
    如果图线$a$在一定pH条件下高于图线$b$,那么$a$上方对应的氧化态物质\ce{Ox(a)}会与$b$下方对应的还原态物质\ce{Red(b)}发生反应.\\
    例如,\ce{Fe^3+/Fe^2+}的线高于\ce{Fe^2+/Fe}的线,说明标准态下\ce{Fe^3+}可以氧化\ce{Fe}生成\ce{Fe^2+}.\\
    又如,\ce{Fe(OH)3 /Fe^2+}的线低于\ce{O2/H2O}的线,说明标准态下\ce{O2}可以氧化\ce{Fe^2+}为\ce{Fe(OH)3}.
\end{theorem}
随着体系不同,$\varphi-pH$图也可能发生形式上的改变,但核心思想是相同的.
\end{document}