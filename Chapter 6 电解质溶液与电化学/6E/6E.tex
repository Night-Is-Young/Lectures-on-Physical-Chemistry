\documentclass{ctexart}
\usepackage{PhysicalChemistryNote}

\begin{document}\pagestyle{plain}
\noindent\tbf{\LARGE 6E 电解与极化作用}\vspace{15pt}\\
\indent 我们在前面主要讨论了原电池,而并没有涉及化学电池中另一重要的类别——电解池.%
理论上,只需给电池外加大于其电动势的电压,就能使其变为电解池,而实际操作中往往要外加比理论值大得多的电压.%
这是由于电极的计划作用所致.本节,我们就来详细讨论电解池以及极化作用的原理.\vspace{12pt}\\
\Section{6E.1 分解电压与极化作用}
\Part{分解电压}
\indent 我们以\ce{Pt}电极电解\ce{HCl}水溶液为例.调节施加的电压$U$,测定对应的电流$I$,得到电解时的$U-I$曲线,如下图所示.
\begin{figure}[H]
    \centering\documentclass{standalone}
\usepackage{PhysicalChemistryNote}
\begin{document}
\begin{tikzpicture}
    \draw[-latex] (0,0)--(5,0) node[right] {$U$};
    \draw[-latex] (0,0)--(0,5) node[left] {$I$};
    \draw[thick,blue,domain=0:2.5] plot[smooth] (\x,{\x^2/4});
    \draw[thick,blue,domain=2.5:5] plot[smooth] (\x,{1.25*\x-1.25*1.25});
    \draw[dashed,domain=1.25:2.5] plot[smooth] (\x,{1.25*\x-1.25*1.25});
    \node[below] at (1.25,0) {$E_{\text{b},\max}$};
\end{tikzpicture}
\end{document}
\end{figure}
开始施加外电压时,尚没有\ce{H2}与\ce{Cl2}生成.继续增大外电压,在电极上开始有\ce{H2}与\ce{Cl2}生成,%
并形成与外加电压方向相反的原电池,从而形成\tbf{反电动势}.
\begin{definition}[6E.1.1 反电动势]
    电解时,电解产物附着在电极上产生的与外加电压方向相反的电势差称为\tbf{反电动势}.
\end{definition}

\end{document}