\documentclass{standalone}
\usepackage{PhysicalChemistryNote}
\begin{document}
\begin{tikzpicture}[scale=0.5]
    \draw[thick] (-2,-10)--(14,-10)--(14,14)--(-2,14)--(-2,-10);

    \draw[thick,dashed] (-2,1.1832)--(14,-8.2824);
    \draw[thick,dashed] (-2,13.4732)--(14,4.0076);

    \draw[thick,red] (-2,7.71)--(2.82,7.71);
    \draw[thick,red] (-2,-4.47)--(8.34,-4.47);

    \draw[thick,blue] (2.82,7.71)--(2.82,14);
    \draw[thick,blue] (8.34,-4.47)--(8.34,-2.0939);

    \draw[thick,violet] (2.82,7.71)--(8.34,-2.0939);
    \draw[thick,violet] (8.34,-2.0939)--(14,-5.4424);
    \draw[thick,violet] (8.34,-4.47)--(14,-7.8184);

    \node at (0.41,10.8) {\ce{Fe^3+}};
    \node at (2,3) {\ce{Fe^2+}};
    \node at (10,3) {\ce{Fe(OH)3}};
    \node at (11.4,-5) {\ce{Fe(OH)2}};
    \node at (6,-7.5) {\ce{Fe}};
    \node at (3.5,-1.2) {\ce{H2O}};
    \node at (2.5,-2.2) {\ce{H2}};
    \node at (8,6.6) {\ce{H2O}};
    \node at (9,7.6) {\ce{O2}};


    \draw (0,-9.8)--(0,-10) node[below]{$0$};
    \draw (2,-9.8)--(2,-10) node[below]{$2$};
    \draw (4,-9.8)--(4,-10) node[below]{$4$};
    \draw (6,-9.8)--(6,-10) node[below]{$6$};
    \draw (8,-9.8)--(8,-10) node[below]{$8$};
    \draw (10,-9.8)--(10,-10) node[below]{$10$};
    \draw (12,-9.8)--(12,-10) node[below]{$12$};
    \node[below] at (14,-10) {$14$};
    \draw (-1.8,0)--(-2,0) node[left]{$0$};
    \draw (-1.8,5)--(-2,5) node[left]{$0.5$};
    \draw (-1.8,-5)--(-2,-5) node[left]{$-0.5$};
    \draw (-1.8,10)--(-2,10) node[left]{$1.0$};
    \draw (-1.8,-10)--(-2,-10) node[left]{$1.0$};
    \node[rotate=90] at (-4,2) {$\varphi/\text{V}$};
    \node at (6,-12) {pH};
\end{tikzpicture}
\end{document}