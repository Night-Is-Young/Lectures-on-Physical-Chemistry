\documentclass{ctexart}
\usepackage{PhysicalChemistryNote}

\begin{document}\pagestyle{plain}
\noindent\tbf{\LARGE 7A 化学反应的速率与速率方程}\vspace{15pt}\\
\indent 通常,化学反应的进行总是需要一定时间.有些反应总是进行得很快(例如炸药的爆炸),%
而有些反应的速度却相当让人着急(比如无催化剂下\ce{N2}与\ce{H2}反应生成\ce{NH3}).%
于是,我们希望用一种普适的量描述化学反应的速率,并且想办法求出速率与各个反应物的浓度的关联.\vspace{12pt}\\
\Section{7A.1 化学反应的速率}
\indent 我们先从速率如何定义入手学习描述反应速率的方法.
\begin{derivation}
    朴素地来看,如果单位时间内产生的产物(或消耗的反应物)越多,那么反应的速率应当越快.%
    不过,由于物质的量$n$是广度量,因此系统的规模会有所影响.%
    因此,我们采用一定时间内某组分浓度的变化衡量反应速率.\\
    于是组分$i$的反应速率$v_i$就满足
    \[v_i=\dfrac{\di[i]}{\di t}=\dfrac{1}{V}\dfrac{\di n_i}{\di t}\]
    由于反应中各物质的计量数可能并不相同,因此用上面的方法得出的速率也各有差别.%
    回忆我们在热力学中消除这一影响的方法(见\tbf{5A.1.1}),我们可以用反应进度$\xi$代替$n_i$.这样就有
    \[v_i=\dfrac{1}{V}\dfrac{\di n_i}{\di t}=\nu_i\cdot\dfrac{1}{V}\dfrac{\di\xi}{\di t}\]
    于是我们可以得到一个统一的速率$v=\dfrac{1}{V}\dfrac{\di\xi}{\di t}$.结合计量数$\nu_i$,就可以得出任意组分的反应速率$v_i$.
\end{derivation}
\begin{definition}[7A.1.1 化学反应的速率]
    (体积不变的均相系统中的)\tbf{化学反应的速率}$v$定义为反应进度$\xi$对时间$t$的变化率与系统体积$V$的比值,即
    \[v\xlongequal{\text{def}}\dfrac{1}{V}\dfrac{\di\xi}{\di t}\]
    对于非均相系统,常常采用界面浓度$\sigma_i=\dfrac{n_i}{A}$代替上述推导中的浓度以得出速率.
\end{definition}
从上面的推导中很容易得出速率与各物质产生/消耗的速率的关系.
\begin{theorem}[7A.1.2 反应速率与各物质生成/消耗速率的关系]
    反应物$i$的\tbf{消耗速率}(注意这是一个正值)满足
    \[-\dfrac{\di[i]}{\di t}=-\nu_iv\]
    产物$j$的\tbf{生成速率}满足
    \[\dfrac{\di[j]}{\di t}=\nu_jv\]

\end{theorem}
\vspace{8pt}
\Section{7A.2 速率方程与速率常数}
\Part{速率方程与速率常数}
\indent 在大部分时候,反应速率都与反应物的浓度密切相关.%
研究表明,反应速率常常与反应物的浓度的幂次方成正比.例如对于反应%
\ce{A + B -> C},反应的速率可能与\ce{A}和\ce{B}的浓度分别成正比,即
\[v=k[\ce{A}][\ce{B}]\]
这样的将反应速率用各物质的浓度与比例系数联系起来的方程即为反应的速率方程.
\begin{definition}[7A.2.1 速率方程与速率常数]
    化学反应的\tbf{速率方程}是用参与反应的各物质的浓度表示反应速率的方程,即
    \[v=f\left([\ce{A}],[\ce{B}],\cdots\right)\]
    其中$f$为某种函数关系.对于均相气相反应,也常常用分压代替浓度.\\
    如果反应速率与各反应物的浓度的幂次方成正比,即
    \[v=k[\ce{A}]^{a}[\ce{B}]^b\cdots\]
    那么比例系数$k$就称为反应的\tbf{速率常数}.更一般的,出现在速率方程中的各种与浓度无关的常数都可以称作速率常数.
\end{definition}
反应的速率方程是由实验测定的,通常与反应方程式没有明显的联系.例如,实验测定反应
\begin{tightcenter}
    \ce{H2 + Br2 -> 2HBr}
\end{tightcenter}
的速率方程为
\[v=\dfrac{k_a\con{H2}\con{Br2}^{\frac32}}{\con{Br2}+k_b\con{HBr}}\]
式中$k_a$和$k_b$为两个不同的速率常数.显然,这一速率方程并不能直接的由反应方程式推出.%
我们将在之后推导这一反应的速率方程.\\
\indent 速率常数$k$的单位总是转换浓度的乘积为反应速率的单位.%
例如对于反应\ce{A + B -> C},假定反应的速率方程为$r=k\con{A}\con{B}$,并且用$\text{mol}\cdot\text{L}^{-1}$作为浓度单位,那么%
$k$的单位应当为$\text{L}\cdot\text{mol}^{-1}\cdot\text{s}^{-1}$.这是因为
\[\underbrace{\text{L}\cdot\text{mol}^{-1}\cdot\text{s}^{-1}}_k
\times\underbrace{\text{mol}\cdot\text{L}^{-1}}_{\con{A}}
\times\underbrace{\text{mol}\cdot\text{L}^{-1}}_{\con{B}}
=\underbrace{\text{mol}\cdot\text{L}^{-1}\cdot\text{s}^{-1}}_v\]
对于其他反应,亦可以根据类似的量纲分析的方法确定速率常数$k$的单位.\\
\indent 速率方程有很多实际的应用.我们在\tbf{7B}会提到,确定了反应的速率方程和初始时各物质的浓度,就可以知道任意时刻任意物质的浓度.%
在\tbf{7E}中也会提到,速率方程也可以作为判断反应机理是否合理的重要证据.\vspace{4pt}\\
\Part{反应级数}
\indent 我们在前面已经提到很多反应的速率方程都具有$v=k\con{A}^a\con{B}^b\cdots$的形式.%
事实上,研究这些式子中浓度的幂次$a,b,\cdots$是十分重要的.我们对其给出\tbf{反应级数}的定义.
\begin{definition}[7A.2.2 反应级数]
    如果反应速率对某物质(可以是反应物,也可以是产物)的浓度的幂次成正比,这一幂次就称为反应对该物质的\tbf{级数}.\\
    如果反应速率对所有物质浓度的幂次都成正比(即具有$v=k\con{A}^a\con{B}^b\cdots$的形式),%
    那么各物质的级数之和称为该反应的\tbf{总级数}.
\end{definition}
例如,如果某一反应的速率方程为$v=k\con{A}\con{B}^2$,那么反应对\ce{A}的级数为$1$,对\ce{B}的级数为$2$,总反应级数为$3$.%
又如果某一反应的速率方程为$v=k\con{A}^{\frac32}\con{B}$,那么反应对\ce{A}的级数为$\dfrac32$,对\ce{B}的级数为$1$,总反应级数为$\dfrac{5}{2}$.\\
\indent 对于某些复杂的例子,例如我们上面提到的\ce{H2}与\ce{Br2}的反应,这一反应对\ce{H2}的级数为$1$,但对于\ce{Br2}和\ce{HBr}的级数却不明确.\vspace{4pt}\\
\Part{速率方程的确定}
\indent 由于速率方程很可能与多种物质的浓度都有关系,因此我们最好采取控制变量的方式反别研究可能影响速率的物质.%
速率方程的确定可以用\tbf{隔离法}进行.对于某种反应物,如果其它反应物大大过量,在反应时其浓度就可以视作常数,从而得知速率与某反应物浓度的关系.%
依次隔离(即使得其余物质大大过量)各个反应物与产物,就可以获知速率方程的大致形式.\\
\indent 我们仍然以反应\ce{A + B -> C}为例.如果\ce{B}大大过量,那么反应的全过程中它的浓度变化都可以忽略,进而近似地视作常数.%
如果反应的真实速率方程为$v=k\con{A}^2\con{B}$,那么做此近似后就有
\[v=k_{\ce{A},\text{eff}}\con{A}^2\ \ \ \ \ k_{\ce{A},\text{eff}}=k\con{B}_0\]
其中$\con{B}_0$为\ce{B}的起始浓度.这样,反应的速率方程就由三级变为二级.我们给这样的反应做出相应的定义.
\begin{definition}[7A.2.3 准级数反应与有效速率常数]
    如果反应中某反应物的浓度特别大时,或者在反应过程中几乎不发生变化时(如催化剂),该反应物的浓度就可以作为常数处理,%
    该反应物浓度与原速率常数混合为新的\tbf{有效速率常数}$k_{\text{eff}}$,反应的总级数就会改变,这种情形称为\tbf{准级数反应}.
\end{definition}
例如上面的反应在\ce{B}大过量时就成为准二级反应.同样地,如果\ce{A}大过量,那么反应就变为准一级反应,其速率方程为
\[v=k_{\ce{B},\text{eff}}\con{B}\ \ \ \ \ k_{\ce{B},\text{eff}}=k\con{A}_0^2\]
准级反应的速率方程相对于原来的速率方程更为简单,因此更容易分析和鉴定.\\
\indent 现在,在排除其它物质的影响后,我们需要确定反应对某种物质的级数.%
我们假定隔离\ce{A}后反应的速率方程变为
\[v=k_{\ce{A},\text{eff}}\con{A}^a\]
一种简便的方法是研究反应起始的速率与起始浓度的关系%
(前者可以通过测定产物生成的速率得到,后者可以通过投料量得到).%
于是就有
\[v_0=k_{\ce{A},\text{eff}}\con{A}^a_0\]
研究指数关系的好办法是取对数,于是对上式取对数就有
\[\ln v_0=\ln k_{\ce{A},\text{eff}}+a\ln\con{A}_0\]
这恰好是$y=ax+b$的形式.因此,如果知道不同的$\con{A}_0$和对应的$v_0$,就可以采取线性回归的方式求出$a$与$\ln k_{\ce{A},\text{eff}}$.%
这种方法也被称为\tbf{对数图解法}.\\
\indent 尽管上述方法对于一般的速率方程十分有效,但仍有一些缺陷.例如,如果反应物也出现在速率方程中,就很难通过%
初始速率确定它对反应的影响,因为随着反应的进行,它的浓度也会使得反应的速率发生改变%
(仍然可以参考\ce{H2}与\ce{Br2}反应的速率方程,其中$\con{HBr}$出现在分母).%
对于这种情况,我们需要进行更多情况下速率的测定,从而得出速率方程%
\footnote{一般而言,你只需要掌握简单的对数图解法即可(即对于那些多项式形式的速率方程),而对这种复杂的反应则并不要求你从繁杂的数据中得出速率方程的形式.}.\\
\indent 我们在\tbf{7B}中也会提到另一种拟定反应级数的方式.
\end{document}