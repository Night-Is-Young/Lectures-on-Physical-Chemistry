\documentclass{ctexart}
\usepackage{PhysicalChemistryNote}

\begin{document}\pagestyle{plain}
\noindent\tbf{\LARGE 7B 积分速率方程}\vspace{15pt}\\
\indent 如我们前面所述,速率方程事实上是一个关于物质浓度的微分方程,因而可以通过数学方法求解各物质浓度与时间的关系.%
在大多数情形下,这些微分方程都有精确的解析解\footnote{即有明确函数表达式的解.}.%
我们将在本节讨论常见速率方程及其解,并由此介绍其应用.\vspace{12pt}\\
\Section{7B.1 简单整数级反应}
\Part{零级反应}
\indent 我们从最简单的零级反应入手.零级反应的积分速率方程可以推导如下.
\begin{derivation}
    考虑反应\ce{A -> P},其速率方程为
    \[v=-\dfrac{\di\con{A}}{\di t}=k\]
    这是一个再简单不过的微分方程,我们移项可得
    \[\di\con{A}=-k\di t\]
    考虑起始时间为$0$,\ce{A}的起始浓度为$\con{A}_0$,对上式两边积分可得
    \[\con{A}-\con{A}_0=-kt\]
    即
    \[\con{A}=\con{A}_0-kt\]
    这表明$\con{A}$与时间$t$成一次函数关系.我们在下图中给出了\ce{A}的浓度随时间变化的图像.
    \begin{tightcenter}
        \documentclass{standalone}
\usepackage{PhysicalChemistryNote}
\begin{document}
\begin{tikzpicture}
    \draw[->] (0,0)--(6.5,0) node[right] {$t$};
    \draw[->] (0,0)--(0,4.5) node[above] {$\con{A}$};
    \draw[-] (0,4)--(6,0);
    \draw[-] (0,4)--(2,0);
    \fill (0,4) circle (1.5pt) node[left] {$\con{A}_0$};
    \fill (2,0) circle (1.5pt) node[below] {$t_1$};
    \fill (6,0) circle (1.5pt) node[below] {$t_2$};
    \node[right] at (1,2) {$k_1$};
    \node[right] at (3.1,2) {$k_2$};
\end{tikzpicture}
\end{document}
    \end{tightcenter}
    在\ce{A}反应完全后,反应便不再进行,保持$\con{A}=0$.
\end{derivation}
\end{document}