\documentclass{ctexart}
\usepackage{PhysicalChemistryNote}

\begin{document}\pagestyle{plain}
\noindent\tbf{\LARGE 7B 积分速率方程}\vspace{15pt}\\
\indent 如我们前面所述,速率方程事实上是一个关于物质浓度的微分方程,因而可以通过数学方法求解各物质浓度与时间的关系,称为\tbf{积分速率方程}.%
在大多数情形下,这些微分方程都有精确的解析解\footnote{即有明确函数表达式的解.}.%
我们将在本节讨论常见速率方程及其解,并由此介绍其应用.\vspace{12pt}\\
\Section{7B.1 简单整数级反应的积分速率方程}
\Part{零级反应}
\indent 我们从最简单的零级反应入手.零级反应的积分速率方程的推导如下.
\begin{derivation}
    考虑零级反应\ce{A -> P},其速率方程为
    \[v=-\dfrac{\di\con{A}}{\di t}=k\]
    这是一个再简单不过的微分方程,我们移项可得
    \[\di\con{A}=-k\di t\]
    考虑起始时间为$0$,\ce{A}的起始浓度为$\con{A}_0$,对上式两边积分可得
    \[\con{A}-\con{A}_0=-kt\]
    即
    \[\con{A}=\con{A}_0-kt\]
    这表明$\con{A}$与时间$t$成一次函数关系.我们在下图中给出了\ce{A}的浓度随时间变化的图像.
    \begin{tightcenter}
        \documentclass{standalone}
\usepackage{PhysicalChemistryNote}
\begin{document}
\begin{tikzpicture}
    \draw[->] (0,0)--(6.5,0) node[right] {$t$};
    \draw[->] (0,0)--(0,4.5) node[above] {$\con{A}$};
    \draw[-] (0,4)--(6,0);
    \draw[-] (0,4)--(2,0);
    \fill (0,4) circle (1.5pt) node[left] {$\con{A}_0$};
    \fill (2,0) circle (1.5pt) node[below] {$t_1$};
    \fill (6,0) circle (1.5pt) node[below] {$t_2$};
    \node[right] at (1,2) {$k_1$};
    \node[right] at (3.1,2) {$k_2$};
\end{tikzpicture}
\end{document}
    \end{tightcenter}
    在\ce{A}反应完全后,反应便不再进行,保持$\con{A}=0$.
\end{derivation}
\begin{theorem}[7B.1.1 零级反应的积分速率方程]
    对于零级反应\ce{A -> P},其积分速率方程为
    \[\con{A}=\con{A}_0-kt\]
    其中$k$为速率常数,$\con{A}_0$为\ce{A}的起始浓度.
\end{theorem}
\Part{一级反应}
\indent 一级反应的积分速率方程的推导如下.
\begin{derivation}
    考虑一级反应\ce{A -> P},其速率方程为
    \[v=-\dfrac{\di\con{A}}{\di t}=k\con{A}\]
    同样移项可得
    \[\dfrac{\di\con{A}}{\con{A}}=-k\di t\]
    考虑起始时间为$0$,\ce{A}的起始浓度为$\con{A}_0$,对上式两边积分可得
    \[\int_{\con{A}_0}^{\con{A}}\dfrac{1}{\con{A}}\di\con{A}=-\int_0^{t}k\di t\]
    注意到$\displaystyle\int \dfrac{1}{x}\dx=\ln x$,于是上式即
    \[\ln\con{A}-\ln\con{A}_0=-kt\]
    即
    \[\ln\dfrac{\con{A}}{\con{A}_0}=-kt\ \ \ \ \ \con{A}=\con{A}_0\e^{-kt}\]
    如果将$\ln\con A$对$t$作图,将得到一条斜率为$-k$,截距为$\ln\con{A}_0$的直线.\\
    从上面的表达式可以看出,$\con{A}$的浓度随时间呈现负指数衰减.我们也在下图中给出了\ce{A}的浓度随时间变化的图像.
    \begin{tightcenter}
        \input{picture/first-order-reaction.tex}
    \end{tightcenter}
    可以看到,\ce{A}被消耗的速率随着时间而减缓.
\end{derivation}
\begin{theorem}[7B.1.2 一级反应的积分速率方程]
    对于一级反应\ce{A -> P},其积分速率方程为
    \[\con{A}=\con{A}_0\e^{-kt}\]
    其中$k$为速率常数,$\con{A}_0$为\ce{A}的起始浓度.
\end{theorem}
一级反应的速率的一个重要的指示是\tbf{半衰期}.
\begin{definition}[7B.1.3 半衰期]
    对于某个反应,某反应物浓度降低至原来的一半所用的时间为反应的\tbf{半衰期},记作$t_{1/2}$或$t_{\frac12}$.
\end{definition}
一级反应的半衰期是特殊的.
\begin{derivation}
    在\tbf{7B.1.2}的推导中有
    \[\ln\dfrac{\con{A}}{\con{A}_0}=-kt\]
    令$\dfrac{\con{A}}{\con{A}_0}=\dfrac12$,就可以得到
    \[kt_{1/2}=\ln2\]
    即$t_{1/2}=\dfrac{\ln2}{k}$.这表明一级反应的半衰期与反应物的起始浓度无关,仅与速率常数有关.
\end{derivation}
\begin{theorem}[7B.1.4 一级反应的半衰期]
    一级反应的半衰期$t_{1/2}=\dfrac{\ln 2}{k}$,是仅与速率常数有关的值.%
    因此,在相同的条件下,反应物浓度每降低一半所用的时间是定值.
\end{theorem}
我们将在后面看到,除了一级反应外,其余反应的半衰期都是与起始浓度有关的值.%
因此,有固定的半衰期这一点可以作为判断反应是否是一级的依据.\vspace{4pt}\\
\Part{二级反应}
\indent 相对而言,二级反应就比较复杂了.我们先讨论由一种反应物生成产物的二级反应.
\begin{derivation}
    考虑二级反应\ce{A -> P},其速率方程为
    \[v=-\dfrac{\di\con{A}}{\di t}=k\con{A}^2\]
    同样移项可得
    \[\dfrac{\di\con{A}}{\con{A}^2}=-k\di t\]
    考虑起始时间为$0$,\ce{A}的起始浓度为$\con{A}_0$,对上式两边积分可得
    \[\int_{\con{A}_0}^{\con{A}}\dfrac{1}{\con{A}^2}\di\con{A}=-\int_0^{t}k\di t\]
    注意到$\displaystyle\int \dfrac{1}{x^2}\dx=-\dfrac{1}{x}$,于是上式即
    \[\dfrac{1}{\con{A}_0}-\dfrac{1}{\con{A}}=-kt\]
    即
    \[\con{A}=\dfrac{\con{A}_0}{1+kt\con{A}_0}\]
    如果将$\dfrac{1}{\con{A}}$对$t$作图,将得到一条斜率为$k$,截距为$\dfrac{1}{\con{A}_0}$的直线.\\
    从上面的表达式可以看出,$\con{A}$的浓度随时间呈现倒数衰减.我们也在下图中给出了\ce{A}的浓度随时间变化的图像.
    \begin{tightcenter}
        \documentclass{standalone}
\usepackage{PhysicalChemistryNote}
\begin{document}
\begin{tikzpicture}
    \draw[->] (0,0)--(5.5,0) node[right] {$t$};
    \draw[->] (0,0)--(0,3.5) node[above] {$\dfrac{1}{\con{A}}$};
    \draw[-] (0,0.5)--(21/4,3);
    \draw[-] (0,0.5)--(1.75,3);
    \fill (0,0.5) circle (1.5pt) node[left] {$\dfrac{1}{\con{A}_0}$};
    \node[right] at (0.8,1.5) {$k_1$};
    \node[right] at (2.5,1.5) {$k_2$};
\end{tikzpicture}
\begin{tikzpicture}
    \draw[->] (0,0)--(5,0) node[right] {$t$};
    \draw[->] (0,0)--(0,3.5) node[above] {$\con{A}$};
    \draw[domain=0:5] plot[smooth](\x,{3/(1+1.5*\x)});
    \draw[domain=0:5] plot[smooth](\x,{3/(1+\x/2)});
    \fill (0,3) circle (1.5pt) node[left] {$\con{A}_0$};
    \node[right] at (0.8,1.5) {$k_1$};
    \node[right] at (2.5,1.5) {$k_2$};
\end{tikzpicture}
\end{document}
    \end{tightcenter}
    可以看到,\ce{A}被消耗的速率也随着时间而减缓,并且减缓的速度相对一级反应更大.\\
    如果令$\con{A}=\dfrac12\con{A}_0$,可得
    \[t_{1/2}=\dfrac{1}{k\con{A}_0}\]
    可见二级反应的半衰期与反应物浓度成反比例关系.
\end{derivation}
\begin{theorem}[7B.1.5 二级反应的积分速率方程I]
    对于二级反应\ce{A -> P},其积分速率方程为
    \[\dfrac{1}{\con{A}}=\dfrac{1}{\con{A}_0}+kt\ \ \text{或}\ \ \con{A}=\dfrac{\con{A}_0}{1+k\con{A}_0t}\]
    其中$k$为速率常数,$\con{A}_0$为\ce{A}的起始浓度.其半衰期为
    \[t_{1/2}=\dfrac{1}{k\con{A}}\]

\end{theorem}
现在让我们考虑更复杂的情形,即有两种反应物的二级反应.
\begin{derivation}
    考虑二级反应\ce{A + B -> P},其速率方程为
    \[v=k\con{A}\con{B}\]
    为了统一变量,不妨令$\dfrac{\xi}{V}=x$,这样根据计量数就有
    \[\con{A}=\con{A}_0-x\ \ \ \ \ \con{B}=\con{B_0}-x\]
    于是速率方程即
    \[\dfrac{\di x}{\di t}=k\left(\con{A}_0-x\right)\left(\con{B}_0-x\right)\]
    上式移项即可得
    \[\dfrac{\dx}{\left(\con{A}_0-x\right)\left(\con{B}_0-x\right)}=k\di t\]
    当$\con{A}_0=\con{B}_0$时,情形与一种反应物类似,这里不再赘述.当$\con{A}_0\neq\con{B}_0$时,%
    为了便于求积分,我们对左边的式子稍作拆分即可得
    \[\dfrac{1}{\con{B}_0-\con{A}_0}\left(\dfrac{1}{\con{A}_0-x}-\dfrac{1}{\con{B}_0-x}\right)\di x=k\di t\]
    $t=0$时$x=0$,将上式两端求定积分即可得
    \[\dfrac{1}{\con{B}_0-\con{A}_0}\left(\ln\dfrac{\con{A}}{\con{A}_0-x}-\ln\dfrac{\con{B}_0}{\con{B}_0-x}\right)=kt\]
    我们将$\con{A}=\con{A}_0-x$和$\con{B}=\con{B_0}-x$代入上式并整理,就可得
    \[\ln\dfrac{\con{B}/\con{B}_0}{\con{A}/\con{A}_0}=\left(\con{B}_0-\con{A}_0\right)kt\]
    因此,$\ln\dfrac{\con{B}}{\con{A}}$对时间$t$作图应当得到一条直线.
\end{derivation}
\begin{theorem}[7B.1.6 二级反应的积分速率方程I]
    对于二级反应\ce{A + B -> P},当$\con{A}_0\neq\con{B}_0$时,其积分速率方程为
    \[\dfrac{1}{\con{B}_0-\con{A}_0}\left(\ln\dfrac{\con{A}}{\con{A}_0-x}-\ln\dfrac{\con{B}_0}{\con{B}_0-x}\right)=kt\]
    其中$x=\dfrac{\xi}{V}$.上式亦可以写作
    \[\ln\dfrac{\con{B}/\con{B}_0}{\con{A}/\con{A}_0}=\left(\con{B}_0-\con{A}_0\right)kt
    \ \ \text{或}\ \ 
    \ln\dfrac{\con{B}}{\con{A}}=\left(\con{B}_0-\con{A}_0\right)kt+\ln\dfrac{\con{B}_0}{\con{A}_0}\]

\end{theorem}
\vspace{8pt}
\Section{7B.2 积分速率方程表}
\indent 一些简单的整数级反应的积分速率方程,我们已经在\tbf{7B.1}中介绍得比较清楚了.%
对于更为复杂的反应,不过是求解更复杂的微分方程,这需要你掌握相当高的数学功底和技巧.%
因此,我们在这里仅给出一些反应积分速率方程,其推导过程(不要求掌握)附在本节之后以供参考.
\begin{theorem}[7B.2.1 积分速率方程表]
    \begin{enumerate}[topsep=0pt,parsep=0pt,itemsep=0pt,partopsep=0pt,label=\tbf{\arabic*.}]
        \item 零级反应\ce{A -> P}
            \[v=k\ \ \ \ \ \con{A}=\con{A}_0-kt\]
        \item 一级反应\ce{A -> P}
            \[v=k\con{A}\ \ \ \ \ \con{A}=\con{A}_0\e^{-kt}\]
        \item 二级反应\ce{A -> P}
            \[v=k\con{A}^2\ \ \ \ \ \con{A}=\dfrac{\con{A}_0}{1+k\con{A}_0t}\]
        \item 二级反应\ce{A + B -> P}
            \[v=k\con{A}\con{B}\ \ \ \ \ \ln\dfrac{\con{B}/\con{B}_0}{\con{A}/\con{A}_0}=\left(\con{B}_0-\con{A}_0\right)kt\]
        \item $n$级反应\ce{A -> P}($n\geqslant2$)
            \[v=k\con{A}^n\ \ \ \ \ \dfrac{1}{n-1}\left(\dfrac{1}{\con{A}^{n-1}}-\dfrac{1}{\con{A}_0^{n-1}}\right)=kt\]
        \item 三级反应\ce{A + 2B -> P}
            \[v=k\con{A}\con{B}^2\ \ \ \ \ 
            \dfrac{2\con{P}}{\left(2\con{A}_0-\con{B}_0\right)\left(\con{B}_0-2\con{P}\right)\con{B}_0}
            +\dfrac{1}{\left(2\con{A}_0-\con{B}_0\right)^2}\ln\dfrac{\con{A}_0\left(\con{B}_0-2\con{P}\right)}
            {\left(\con{A}_0-\con{P}\right)\con{B}_0}\]
        
    \end{enumerate}
\end{theorem}
一般而言,掌握前四个就已经足够.我们现在给出\tbf{5.}和\tbf{6.}的证明.
\begin{problem}[P.7B.1]
    求$n$级反应\ce{A -> P}的积分速率方程.
\end{problem}
\begin{proof}
    我们有
    \[v=-\dfrac{\di\con{A}}{\di t}=k\con{A}^n\]
    移项可得
    \[\dfrac{\di\con{A}}{\con{A}^n}=-kt\]
    由于$\displaystyle\int\dfrac{\di x}{x^n}=-\dfrac{1}{(n-1)x^{n-1}}$,于是对上式两端求定积分可得
    \[\dfrac{1}{n-1}\left(\dfrac{1}{\con{A}^{n-1}}-\dfrac{1}{\con{A}_0^{n-1}}\right)=kt\]

\end{proof}
\begin{problem}[P.7B.1]
    求三级反应\ce{A + 2B -> P}的积分速率方程.
\end{problem}
\begin{proof}
    设$x=\dfrac{\xi}{V}$,于是
    \[v=\dfrac{\di x}{\di t}=k\con{A}\con{B}^2\]
    将$\con{A}=\con{A}_0-x,\con{B}=\con{B}_0-2x$代入可得
    \[\dfrac{\di x}{\left(\con{A}_0-x\right)\left(\con{B}_0-2x\right)^2}=k\di t\]
    若$\con{B}_0=2\con{A_0}$,那么这与单反应物的三级反应一致.否则,不妨设
    \[\dfrac{1}{\left(\con{A}_0-x\right)\left(\con{B}_0-2x\right)^2}
    =\dfrac{a}{\con{A}_0-x}+\dfrac{b}{\con{B}_0-2x}+\dfrac{c}{\left(\con{B}_0-2x\right)^2}\]
    比较系数可得
    \[a=\dfrac{1}{\left(2\con{A}_0-\con{B}_0\right)^2}\ \ \ \ \ 
    b=-\dfrac{2}{\left(2\con{A}_0-\con{B}_0\right)^2}\ \ \ \ \ 
    c=\dfrac{2}{2\con{A}_0-\con{B}_0}\]
    于是对原式两边积分可得
    \[a\ln\dfrac{\con{A}_0}{\con{A}_0-x}
    +\dfrac{b}{2}\ln\dfrac{\con{B}_0}{\con{B}_0-2x}
    +\dfrac{c}{2}\left(\dfrac{1}{\con{B}_0}-\dfrac{1}{\con{B}_0-2x}\right)=kt\]
    将$a,b,c$代入上式即可得
    \[\dfrac{1}{\left(2\con{A}_0-\con{B}_0\right)^2}
    \left(\ln\dfrac{\con{A}_0}{\con{A}_0-x}-\ln\dfrac{\con{B}_0}{\con{B}_0-2x}\right)
    +\dfrac{1}{2\con{A}_0-\con{B}_0}\left(\dfrac{1}{\con{B}_0}-\dfrac{1}{\con{B}_0-2x}\right)=kt\]
    考虑到$x=\con{P}$,于是代入并整理可得
    \[\dfrac{2\con{P}}{\left(2\con{A}_0-\con{B}_0\right)\left(\con{B}_0-2\con{P}\right)\con{B}_0}
    +\dfrac{1}{\left(2\con{A}_0-\con{B}_0\right)^2}\ln\dfrac{\con{A}_0\left(\con{B}_0-2\con{P}\right)}
    {\left(\con{A}_0-\con{P}\right)\con{B}_0}\]

\end{proof}
看到这里,你应当发现所有与此类似的整数级反应实际上不过是有理因式的积分罢了.%
这样的积分都可以通过待定系数法得出解析解.我们不希望你与这样枯燥乏味的数学做过多纠缠,%
而应当把精力放在与化学关系更大的地方,所以这一节的内容仅作为参考.
\end{document}