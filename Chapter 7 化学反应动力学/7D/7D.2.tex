\documentclass{ctexart}
\usepackage{PhysicalChemistryNote}

\begin{document}\pagestyle{plain}
\Section{7D.2 聚合反应动力学}
\indent 聚合反应是把低分子量的单体转化成高分子量的聚合物的过程,%
聚合物具有低分子量单体所不具备的可塑性等重要性能,可广泛地用作塑料,纤维,橡胶,涂料,黏合剂等用途.%
聚合物是由一种以上的结构单元(单体)构成的,由单体经重复反应合成的高分子化合物.\\
\indent 聚合反应最特殊的一点在于,我们一般不以产物浓度衡量反应进行的状况,转而用聚合度这一概念.
\begin{definition}[7D.2.1 聚合度]
    \tbf{聚合度}即每个聚合物分子中留存的单体的平均数目,记作$DP$.
\end{definition}
根据机理的不同,我们可以将聚合反应分为如下两类.
\begin{definition}[7D.2.2 逐步聚合与链聚合]
    \tbf{逐步聚合}是指带有两个或多个官能团的单体相互反应,逐步生成二聚体,三聚体,寡聚物以最终形成高分子聚合物的聚合反应.%
    缩聚反应一般通过逐步聚合进行.\\
    \tbf{链聚合}是指中间体与单体反应,每次增长一个长度的聚合反应.%
    加聚反应一般通过链聚合进行,这其中最典型的是自由基链聚合.
\end{definition}
我们先来考虑逐步聚合的速率方程.为了简便考虑,假定只有一种单体参与逐步聚合.%
不妨假定是羟基羧酸\ce{HO-R-COOH}发生缩聚反应.我们现在来推导这一聚合反应的速率方程.
\begin{derivation}
    直接考虑聚合过程显然有些麻烦,因为任意长度的两条链都有可能发生反应形成一条新的链.%
    但有一点是可以确定的,即每一次反应都会使得\ce{-OH}和\ce{COOH}减少一个.因此,%
    我们可以考虑用\ce{-COOH}官能团(记作\ce{A})的浓度衡量反应的进度.\\
    一般的酯化反应对于醇和羧酸均为一级,即
    \[v=k\con{R_1OH}\con{R_2COOH}\]
    对于\ce{HO-R-COOH}而言,每个分子(以及聚合形成的链)都有一个\ce{-COOH}和\ce{-OH},因此有
    \[v=-\dfrac{\di\con{A}}{\di t}=k\con{A}^2\]
    这是一个典型的二级反应,由\tbf{7B.1.5}可知它的积分速率方程
    \[\con{A}=\dfrac{\con{A}_0}{1+k\con{A}_0t}\]
    由于每个聚合物分子都仅在端基含有一个\ce{-COOH},因此聚合物的平均链长与\ce{-COOH}的数目的乘积应当是定值.%
    反应开始时体系中均为单体,聚合度为$1$.于是就有
    \[DP\cdot\con{A}=\con{A}_0\]
    于是
    \[DP=\dfrac{\con{A}_0}{\con{A}}=1+k\con{A}_0t\]
    可见聚合度随着时间线性增长.\\
    我们也可以使用尚未参与反应的\ce{A}的比例$p$衡量聚合反应进行的程度.这样就有
    \[p=\dfrac{\con{A}_0-\con{A}}{\con{A}_0}=\dfrac{k\con{A}_0t}{1+k\con{A}_0t}\]
    以及
    \[DP=\dfrac{1}{1-p}\]

\end{derivation}
这样,我们就知道在这样的简单逐步聚合中,聚合度随时间线性增长这一事实.%
如果你对更加复杂的体系(例如有多种反应物)感兴趣,也可以用相似的步骤推导它们的速率方程.\\
\indent 我们在前面还给出了聚合度$DP$与反应程度$p$的关系,即$D=\dfrac{1}{1-p}$.%
Carothers于1935年提出了在各种体系中$DP$与$p$的关系.
\begin{theorem}[7D.2.3 Carothers方程]
    逐步聚合中,两种等物质的量的单体形成完全线性的聚合物(或者一种单体自身聚合)时,%
    聚合度$DP$与反应程度$p$满足
    \[DP=\dfrac{1}{1-p}\]
    如果一种单体相对过量,则有
    \[DP=\dfrac{1+r}{1+r-2rp}\]
    其中$r$是较少量单体和较多单体的基团比或物质的量之比.
\end{theorem}
你可以自行推导上述结论.\\
\indent 现在,我们来考虑另一种聚合机理——链聚合.它的过程与我们在\tbf{7D.1}中提到的直链反应十分相似,%
我们假定引发剂为\ce{In},聚合单体为\ce{M},则反应机理可以表述如下.
\begin{tightcenter}
    \ce{In ->T[$k_i$] 2R.}\\
    \ce{M + R. ->T[fast] M_1.}\\
    \ce{M_{$n$}. + M ->T[$k_p$] M_{$n+1$}.}\\
    \ce{M_{$n$}. + M_{$m$}. ->T[$k_t$] M_{$n+m$}}
\end{tightcenter}
引发剂形成的自由基\ce{R.}由于其高活泼性,容易在与\ce{M}反应之前就发生分解.%
因此,我们设参与第二个反应的\ce{R.}的比例为$f$.同时,为了简化体系,我们在链终止中只考虑偶联终止.现在我们来推导体系的速率方程.
\begin{derivation}
    我们记\ce{M.}为体系中任意长度的聚合物中间体(这是推导过程中最重要的一步,由于不同长度的中间体在动力学上并无显著不同,因此我们可以将它们视同一种物质).\\
    引发过程的决速步为引发剂\ce{In}的裂解;链增长步骤不改变\ce{M.}的总浓度.因此,对\ce{M.}稳态近似可得
    \[\dfrac{\di\con{M.}}{\di t}=2fk_i\con{In}-2k_t\con{M.}^2\]
    从而
    \[\con{M.}=\sqrt{\dfrac{fk_i\con{In}}{k_t}}\]
    \ce{M}主要在链增长过程中被消耗,于是其消耗速率
    \[-\dfrac{\di\con{M}}{\di t}=k_p\con{M.}\con{M}=k_p\sqrt{\dfrac{fk_i}{k_t}}\con{M}\con{In}^{\frac12}\]
    现在考虑聚合物的链长.我们先不考虑终止方式,仅考虑引发与增长过程.%
    显然,在链增长时被消耗的\ce{M}的数目与用于引发的\ce{R.}的数目之比就是每个链在终止之前的平均长度.我们把它记为动力学链长$\lambda$,即有
    \[\lambda
    =\dfrac{n\left(\text{消耗的\ce{M}}\right)}{n\left(\text{用于引发的\ce{R.}}\right)}
    =\dfrac{v\left(\text{链增长}\right)}{v\left(\text{链引发}\right)}
    =\dfrac{v\left(\text{链增长}\right)}{v\left(\text{链终止}\right)}
    =\dfrac{k_p\con{M.}\con{M}}{2k_t\con{M.}^2}
    =\dfrac{k_p\con{M}}{2k_t\con{M.}}\]
    代入$\con{M.}$的表达式即有
    \[\lambda=\dfrac{k_p}{2\sqrt{fk_ik_t}}\con{M}\con{In}^{-\frac12}\]
    由于这里的终止方式是偶联终止,因此产物事实上由两条链构成.于是聚合度
    \[DP=2\lambda=\dfrac{k_p}{\sqrt{fk_ik_t}}\con{M}\con{In}^{-\frac12}\]
    这就是链聚合的聚合度的表达式.对于不同的引发剂和终止方式,上式略有不同,%
    但整体上的推导方式是相似的.\\
    从上式也可以看出,单体\ce{M}的浓度越高,引发剂\ce{In}的浓度越低,聚合度越大.
\end{derivation}
\end{document}