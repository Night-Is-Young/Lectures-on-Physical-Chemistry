\documentclass{ctexart}
\usepackage{PhysicalChemistryNote}

\begin{document}\pagestyle{plain}
\noindent\tbf{\LARGE 7D 反应机理示例}\vspace{15pt}\\
\indent 在这一节中,我们将综合运用你的数学与化学知识来推导各种反应的速率方程,%
并加深你对\tbf{7C}中学习的理论知识的印象与实用的技巧.\vspace{12pt}\\
\Section{7D.1 链反应}
\Part{链反应的基本概念}
\indent 在化学动力学中有一类特殊的反应,只需用热,光或辐射等方法使反应引发,%
体系就能通过活性组分(通常是自由基或原子)相继发生一系列的连续反应,%
像链条一样自动地发展下去.
\begin{definition}[7D.1.1 链反应]
    \tbf{链反应}(又称\tbf{连锁反应}),是指反应的产物或副产物又可作为其他反应的原料,%
    从而使反应反复发生.在化学中,链反应通常在光,热,辐射或引发剂作用下,反应中交替产生活性中间体(如自由原子或自由基),%
    从而使反应一直进行下去.
\end{definition}
按照活性物质数量的变化,链反应主要有三个过程.
\begin{definition}[7D.1.2 链反应的过程]
    在链反应中,产生活性中间体的过程称为\tbf{链引发},%
    活性中间体与反应物分子反复作用生成产物的过程称为\tbf{链增长}或\tbf{链传递},%
    活性中间体最后湮灭的过程称为\tbf{链终止}.
\end{definition}
一般的链增长过程中,一个活性中间体产生一个新的活性中间体.例如\ce{Cl.}与\ce{H2}的反应:
\begin{tightcenter}
    \ce{Cl. + H2 -> HCl + H.}
\end{tightcenter}
不过,在部分链增长过程中,一个活性中间体也可能产生数个活性中间体.例如\ce{H.}与\ce{O2}的反应:
\begin{tightcenter}
    \ce{H. + O2 -> HO. + O.}
\end{tightcenter}
据此,我们可以按照链增长的性质对链反应进行分类.
\begin{definition}[7D.1.3 直链反应与支链反应]
    一个活性中间体只能产生一个新的活性中间体的反应称为\tbf{直链反应},%
    可以产生两个或多个新的活性中间体的反应称为\tbf{支链反应}.
\end{definition}
我们将在接下来对这些链反应的速率方程进行详细地讨论.\vspace{4pt}\\
\Part{直链反应——\ce{H2}与卤素单质的自由基反应}
\indent 对中间体的研究表明,\ce{H2}与\ce{X2}(其中$\ce{X}=\ce{Cl},\ce{Br},\ce{I}$)在光照或加热下的化合反应%
的机理是不同的.我们先从最简单的\ce{H2}与\ce{Cl2}的反应开始.
\begin{derivation}\setcounter{equation}{0}
    \ce{H2}与\ce{Cl2}通过自由基反应生成\ce{HCl}的反应机理如下.
    \begin{tightcenter}
        \ce{Cl2 <=>T[$k_1$][$k_{-1}$] 2Cl.}\\
        \ce{Cl. + H2 ->T[$k_2$] HCl + H.}\\
        \ce{H. + Cl2 ->T[$k_3$] HCl + Cl.}
    \end{tightcenter}
    由于产物\ce{HCl}十分稳定,因此忽略后两个反应的逆反应.\\
    体系中的不稳定中间体为\ce{H.}与\ce{Cl.},分别对它们稳态近似有
    \begin{equation}
        \dfrac{\di\con{H.}}{\di t}=k_2\con{Cl.}\con{H2}-k_3\con{H.}\con{Cl2}=0
    \end{equation}
    \begin{equation}
        \dfrac{\di\con{Cl.}}{\di t}=2k_1\con{Cl2}-2k_{-1}\con{Cl.}^2-k_2\con{Cl.}\con{H2}+k_3\con{H.}\con{Cl2}=0
    \end{equation}
    将(2)减去(1)可得
    \begin{equation}
        2k_1\con{Cl2}-2k_{-1}\con{Cl.}^2=0
    \end{equation}
    于是
    \begin{equation}
        \con{Cl.}=\sqrt{\dfrac{k_1}{k_{-1}}\con{Cl2}}
    \end{equation}
    由(1)可得
    \begin{equation}
        \dfrac{\di\con{HCl}}{\di t}=k_2\con{Cl.}\con{H2}+k_3\con{H.}\con{Cl2}=2k_2\con{Cl.}\con{H2}
    \end{equation}
    将(4)代入(5)可得
    \begin{equation}
        \dfrac{\di\con{HCl}}{\di t}=2k_2\con{Cl.}\con{H2}=2k_2\sqrt{\dfrac{k_1}{k_{-1}}}\con{H2}\con{Cl2}^{\frac12}
    \end{equation}
    因此反应对\ce{H2}为一级,对\ce{Cl2}为二分之一级.
\end{derivation}

\end{document}