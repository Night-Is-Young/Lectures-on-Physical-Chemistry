\documentclass{ctexart}
\usepackage{PhysicalChemistryNote}

\begin{document}\pagestyle{plain}
\setcounter{footnote}{0}
\begin{center}
    \tbf{\Huge Chapter 7 化学反应动力学}
\end{center}\vspace{15pt}

\indent 想象一下:你小心翼翼地把面团塞进烤箱,等待它膨胀成蓬松的面包——但如果你打开烤箱的瞬间,%
面团就“唰”地变成了焦炭,或者干脆像被施了冰冻咒一样毫无变化,厨房大概会变成灾难现场吧?%
好在现实中,大多数化学反应都像一位慢性子的老管家,不慌不忙地按照自己的节奏工作.%
而化学动力学的使命,就是破解这位管家藏在围裙口袋里的“日程表”,%
搞清楚它为何有时慢吞吞,有时又像被踩了尾巴的猫一样蹿得飞快(比如嘭一声炸开的爆米花).\\
\indent 如果说热力学是化学界的“预言家”,只告诉你“面包最终会不会烤熟”,%
那么动力学就是那个举着秒表,贴着烤箱玻璃偷看的厨师.%
它不仅关心反应的终点,更执着于追踪过程中每一个微妙的时间刻度:为什么加一撮酵母能让面团在半小时内膨胀,而不是半年?%
为什么夏天牛奶变质的速度总让冰箱成为人类最伟大的发明之一?%
这些看似平常的现象背后,都藏着分子世界速度与激情的故事.\\
\indent 准备好了吗?带上你的好奇心(和计算器),我们要推开一扇新世界的大门了——%
在这里,时间不再是钟表的滴答声,而是分子们踢踏舞步的节奏,是能量起伏的山丘地图,更是人类掌控物质变换的终极秘籍.%
祝你好运!
\end{document}