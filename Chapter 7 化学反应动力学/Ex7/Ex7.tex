\documentclass{ctexart}
\usepackage{PhysicalChemistryNote}

\begin{document}\pagestyle{plain}
\noindent\tbf{\LARGE Ex7 习题}\vspace{15pt}\\
\indent 我们将在本章的习题中给出更多的例题以供你巩固动力学知识.
\setcounter{Pcounter}{0}
\stepcounter{Pcounter}
\begin{problem}[P.7.\arabic{Pcounter}]
    Heck偶联反应为代表的过渡金属催化的偶联反应.这一反应的通式如下.
    \begin{tightcenter}
        \ce{ArBr + RCH=CH2 ->T[Pd catalyst][AcONa] ArCH=CHR + AcOH + NaBr}
    \end{tightcenter}
    这一反应的机理可以简化如下.
    \begin{tightcenter}
        \ce{E + A <=>T[$k_1$][$k_{-1}$] EA} \\
        \ce{EA + B ->T[$k_2$] E + P}
    \end{tightcenter}
    其中\ce{E}为\ce{Pd}催化剂,\ce{A}为溴代物,\ce{B}为原料烯烃,\ce{P}为产物烯烃.
    \begin{enumerate}[label=\tbf{\arabic{Pcounter}-\arabic*},topsep=0pt,parsep=0pt,itemsep=0pt,partopsep=0pt]
        \item 推导总反应速率$r$(以生成产物\ce{P}的速率记)的表达式.
        \item 记$\con{ex}=\con{B}_0-\con{A}_0$为两种底物的起始浓度之差,$\con{E}_0$为催化剂的总浓度.试证明反应速率$r$可以写成如下形式
            \[r=a\dfrac{\con{ex}\con{A}+\con{A}^2}{1+b\con{A}}\con{E}_0\]
            并求出系数$a,b$的表达式.
    \end{enumerate}
\end{problem}
\begin{solution}
    \begin{enumerate}[label=\tbf{\arabic{Pcounter}-\arabic*},topsep=0pt,parsep=0pt,itemsep=0pt,partopsep=0pt]
        \item 这事实上不过是米氏方程的变形.我们对\ce{EA}稳态近似可得
            \[\dfrac{\di\con{EA}}{\di t}=k_1\con{E}\con{A}-k_{-1}\con{EA}-k_2\con{EA}\con{B}=0\]
            于是
            \[\con{EA}=\dfrac{k_1\con{E}\con{A}}{k_{-1}+k_2\con{B}}\]
            由物料守恒$\con{E}+\con{EA}=\con{E}_0$可得
            \[\con{EA}\left(1+\dfrac{k_{-1}+k_2\con{B}}{k_1\con{A}}\right)=\con{E}_0\]
            于是
            \[r=k_2\con{EA}\con{B}=\dfrac{k_1k_2\con{E}_0\con{A}\con{B}}{k_{-1}+k_1\con{A}+k_2\con{B}}\]
        \item 根据计量数有$\con{ex}=\con{B}_0-\con{A}_0=\con{B}-\con{A}$.代入第一问的结果可得
            \[r=\dfrac{k_1k_2\con{E}_0\con{A}\left(\con{A}+\con{ex}\right)}{k_{-1}+k_1\con{A}+k_2\left(\con{A}+\con{ex}\right)}
            =\dfrac{k_1k_2}{k_{-1}+k_2\con{ex}}\cdot\dfrac{\con{ex}\con{A}+\con{A}^2}{1+\dfrac{k_1+k_2}{k_{-1}+k_2\con{ex}}\con{A}}\con{E}_0\]
            于是有
            \[a=\dfrac{k_1k_2}{k_{-1}+k_2\con{ex}}\ \ \ \ \ b=\dfrac{k_1+k_2}{k_{-1}+k_2\con{ex}}\]
            这样,由于$\con{ex}$和$\con{E}_0$都是已知的量,我们就可以通过测定$\con{A}$得出反应速率$r$.
    \end{enumerate}
\end{solution}
\stepcounter{Pcounter}
\begin{problem}[P.7.\arabic{Pcounter}]
    烯烃的硼氢化是有机合成中的常用反应.简单烯烃与\ce{BH3.THF}或\ce{BH3.SMe2}的反应通常在较低的温度下即可快速完成.%
    \ce{9-BBN}(9-硼杂双环[3.3.1]壬烷)是一种对烯烃的硼氢化反应具有更好区域选择性的试剂.%
    在非路易斯碱性溶剂中\ce{9-BBN}主要以二聚体\ce{(9-BBN)2}的形式存在.\\
    二聚体\ce{(9-BBN)2}对烯烃的硼氢化通常均经历二聚体生成单体的对峙反应和单体与烯烃加成的单向反应,%
    但其反应速率受到烯烃空间位阻影响,差异显著.例如,在$25\tccentigrade$条件下,%
    以\ce{CCl4}为溶剂,\ce{(9-BBN)2}与过量的四取代或单取代烯烃进行的硼氢化反应对于烯烃和\ce{(9-BBN)2}的反应级数如下表所示.
    \vspace{-5pt}\begin{table}[H]\centering
        \begin{tabular}{|c|c|c|}
            \hline
                烯烃                    &对烯烃的反应级数   &对\ce{(9-BBN)2}的反应级数 \\\hline
                2,3-二甲基丁-2-烯       &$1$    &$\frac12$ \\\hline
                1-己烯                  &$0$    &$1$    \\\hline
        \end{tabular}
    \end{table}\vspace{-15pt}
    试据此通过合理的近似或假设推导两种烯烃硼氢化的反应速率方程.
\end{problem}
\begin{solution}
    我们将烯烃记作\ce{A},产物记作\ce{P},\ce{9-BBN}记作\ce{B}.题目中所述的反应机理可以表述如下.
    \begin{tightcenter}
        \ce{B2 <=>T[$k_1$][$k_{-1}$] 2B} \\
        \ce{B + A ->T[$k_2$] P}
    \end{tightcenter}
    体系中的不稳定中间体为\ce{B}.我们对\ce{B}稳态近似可得
    \[\dfrac{\di\con{B}}{\di t}=2k_1\con{B2}-2k_{-1}\con{B}^2-k_2\con{A}\con{B}=0\]
    题目中已经明确地说明总反应级数的不同源于烯烃位阻的不同,即影响第二步反应的速率.\\
    如果烯烃\ce{A}的位阻较大,那么\ce{A}与\ce{B}的反应的活化能较大,第二步反应较难,即$k_2$较小.%
    进一步地,如果$k_2\ll k_{-1}$,就有$k_2\con{A}\ll k_{-1}\con{B}$,进一步化简上述稳态近似就有
    \[2k_1\con{B2}=2k_{-1}\con{B}^2\]
    于是
    \[\con{B}=\sqrt{\dfrac{k_1}{k_{-1}}\con{B2}}\]
    于是反应的速率
    \[\dfrac{\di\con{P}}{\di t}=k_2\con{A}\con{B}=k_2\sqrt{\dfrac{k_1}{k_{-1}}}\con{B2}^{\frac12}\con{A}\]
    对烯烃\ce{A}为一级,对\ce{(9-BBN)2}为二分之一级,这就对应\ce{A}为2,3-二甲基丁-2-烯的情形.\\
    如果烯烃\ce{A}的位阻较小,那么\ce{A}与\ce{B}的反应的活化能较小,第二步反应容易进行,即$k_2$较大.%
    进一步地,如果$k_2\gg k_{-1}$,就有$k_2\con{A}\gg k_{-1}\con{B}$,上述稳态近似就有
    \[2k_1\con{B2}=k_2\con{A}\con{B}\]
    于是反应的速率
    \[\dfrac{\di\con{P}}{\di t}=k_2\con{A}\con{B}=2k_1\con{B2}\]
    对烯烃\ce{A}为零级,对\ce{(9-BBN)2}为一级,这就对应\ce{A}为1-己烯的情形.
\end{solution}
\stepcounter{Pcounter}
\begin{problem}[P.7.\arabic{Pcounter}]
    铁卟啉(\ce{FeP})是细胞色素(P-450)的活性中心,具有将各种氧供体的氧原子活化并转移至底物的能力.%
    研究人员为模拟活体内的加氧酶催化$\beta$-胡萝卜素(\ce{$\beta$})分解为维生素\ce{A}(\ce{VA})的反应,%
    以\ce{FeP}为催化剂,间氯过氧化苯甲酸(\ce{CPBA})为氧化剂,研究了$\beta$-胡萝卜素的分解反应动力学.%
    研究中\ce{FeP}和\ce{CPBA}的浓度可视为不变.无论是否存在催化剂\ce{FeP},该分解反应对$\beta$-胡萝卜素均为一级反应.\\
    \tbf{实验A}.在无\ce{FeP}的情况下,$\beta$-胡萝卜素-间氯过氧化苯甲酸反应体系($\beta-\ce{CPBA}$)的反应机理$1$如下%
    (其中$\beta\ast\ce{CPBA}$为反应中间物,\ce{CBA}为间氯苯甲酸):
    \begin{tightcenter}
        \ce{$\beta$ + CPBA <=>T[$K_i'$] $\beta\ast\ce{CPBA}$}\\
        \ce{$\beta\ast\ce{CPBA}$ ->T[$k_2'$] VA + CBA}
    \end{tightcenter}
    \tbf{实验B}.以\ce{FeP}为催化剂,$\beta$-胡萝卜素-间氯过氧化苯甲酸-铁卟啉反应体系($\beta-\ce{CPBA}-\ce{FeP}$)的反应机理$2$如下%
    (其中$\ce{FeOP}\ast\ce{CBA},\beta\ast\ce{FeOP}\ast\ce{CBA},\beta\ast\ce{CPBA}$为反应中间物):
    \begin{tightcenter}
        \ce{FeP + CPBA <=>T[$K_1$] $\ce{FeOP}\ast\ce{CBA}$}\\
        \ce{$\ce{FeOP}\ast\ce{CBA}$ + $\beta$ <=>T[$K_2$] $\beta\ast\ce{FeOP}\ast\ce{CBA}$}\\
        \ce{$\beta\ast\ce{FeOP}\ast\ce{CBA}$ ->T[$k_1$] VA + FeP + CBA}\\
        \ce{$\beta$ + CPBA <=>T[$K_i$] $\beta\ast\ce{CPBA}$}\\
        \ce{$\beta\ast\ce{CPBA}$ ->T[$k_2$] VA + CBA}
    \end{tightcenter}
    对该体系的实验结果进行曲线拟合,可得以下数据(其中$k_{\text{obs}}$为反应的表观速率常数).
    \vspace{-5pt}\begin{table}[H]\centering
        \begin{tabular}{|c|c|c|c|}
            \hline
                $T/\text{K}$ & $k_1/\text{s}^{-1}$ & $k_2/\text{s}^{-1}$ & $k_{\text{obs}}/\text{s}^{-1}$ \\ \hline
                $293.2$ & $4.869\times10^{-3}$ & $1.350\times10^{-4}$ & $5.865\times10^{-4}$ \\ \hline
                $301.2$ & $7.731\times10^{-3}$ & $2.398\times10^{-4}$ & $9.795\times10^{-4}$ \\ \hline
        \end{tabular}
    \end{table}\vspace{-15pt}
    \begin{enumerate}[label=\tbf{\arabic{Pcounter}-\arabic*},topsep=0pt,parsep=0pt,itemsep=0pt,partopsep=0pt]
        \item 对$\beta-\ce{CPBA}$体系,测得
            \[k_{\text{obs}}'(293.2\K)=4.795\times10^{-4}\text{ s}^{-1}\ \ \ \ \ k_{\text{obs}}'(301.2\K)=8.285\times10^{-4}\text{ s}^{-1}\]
            求反应的表观活化能$E_{a,\text{obs}}'$.
        \item 根据反应机理$2$,推导$\beta\ast\ce{CPBA}\ast\ce{FeP}$体系中$-\dfrac{\di[\beta]}{\di t}$与$[\beta]$间关系的速率方程,并给出$k_{\text{obs}}$的表达式.
        \item 已知$\beta\ast\ce{CPBA}\ast\ce{FeP}$体系反应的表观活化能$E_{\text{obs}}=47.07\kJm$和机理$2$中第三步的活化能$E_{a,1}=42.43\kJm$,计算机理$2$中第五步的活化能$E_{a,2}$.
        \item 分别根据以下条件,说明$\beta-\ce{CPBA}$和$\beta\ast\ce{CPBA}\ast\ce{FeP}$中哪一个体系的反应更有利.
            \begin{enumerate}[label=\tbf{\arabic{Pcounter}-4-\arabic*},topsep=0pt,parsep=0pt,itemsep=0pt,partopsep=0pt,leftmargin=10pt]
                \item $E_{a,1}$与$E_{a,2}$的结果.
                \item $E_{a,\text{obs}}$与$E_{a,\text{obs}}'$的结果.
            \end{enumerate}
    \end{enumerate}
\end{problem}
\begin{solution}
    \begin{enumerate}[label=\tbf{\arabic{Pcounter}-\arabic*},topsep=0pt,parsep=0pt,itemsep=0pt,partopsep=0pt]
        \item 根据Arrhenius方程有
            \[\ln\dfrac{k_{\text{obs}}'\left(T_1\right)}{k_{\text{obs}}'\left(T_2\right)}=\dfrac{E_{a,\text{obs}}'}{R}\left(\dfrac{1}{T_2}-\dfrac{1}{T_1}\right)\]
            代入题中数据可得
            \[\ln\dfrac{4.795\times10^{-4}}{8.285\times10^{-4}}=\dfrac{E_{a,\text{obs}}'}{8.314\JmK}\left(\dfrac{1}{293.2\K}-\dfrac{1}{301.2\K}\right)\]
            解得
            \[E_{a,\text{obs}}=50.19\kJm\]
        \item 我们在\tbf{7D}的酶促反应的示例中大多是推导产物的生成速率,而这道题则有些不同.我们要求反应物$\beta$的消耗速率$-\dfrac{\di[\beta]}{\di t}$.%
            并且,由于题目中已经说明酶\ce{FeP}的浓度可视为不变,因此我们并不能使用一般的酶的物料守恒求解.\\
            我们遇到的最大的困难在于有关$\beta$的反应全部出现在快速平衡中,而题目并未给出快速平衡的速率常数,这使得我们无法直接写出$-\dfrac{\di[\beta]}{\di t}$的显式表达式.\\
            因此,我们需要采取一些间接的方法.如果我们用另一物质的浓度表示$[\beta]$,然后将这一表达式对时间$t$求导,即可用这一物质的消耗或生成速率表达$-\dfrac{\di[\beta]}{\di t}$.\\
            由于这一体系的净反应为\ce{$\beta$ + CPBA -> VA + CBA},因此$\beta$的物料守恒为
            \[[\beta]_0=\con{VA}+[\beta]+[\beta\ast\ce{FeOP}\ast\ce{CBA}]+[\beta\ast\ce{CBA}]\]
            直接对此式求导仍然是不可行的,因为后两个中间体的浓度也会随着$[\beta]$的变化而变化.我们需要写出这一关系.%
            根据平衡态假设可知
            \[[\beta\ast\ce{FeOP}\ast\ce{CBA}]=K_1K_2\con{FeP}\con{CPBA}[\beta]\]
            \[[\beta\ast\ce{CBA}]=K_i\con{CPBA}[\beta]\]
            代入上述物料守恒可得
            \[[\beta]_0=\con{VA}+\left(1+K_1K_2\con{FeP}\con{CPBA}+K_i\con{CPBA}\right)[\beta]\]
            我们将上式对时间$t$求导可得
            \[-\dfrac{\di[\beta]}{\di t}=\dfrac{1}{1+K_1K_2\con{FeP}\con{CPBA}+K_i\con{CPBA}}\dfrac{\di\con{VA}}{\di t}\]
            又因为
            \[\begin{aligned}
                \dfrac{\di\con{VA}}{\di t}
                &= k_1[\beta\ast\ce{FeOP}\ast\ce{CBA}]+k_2[\beta\ast\ce{CBA}] \\
                &= \left(k_1K_1K_2\con{FeP}\con{CPBA}+k_2K_i\con{CPBA}\right)[\beta]
            \end{aligned}\]
            于是
            \[-\dfrac{\di[\beta]}{\di t}=\dfrac{k_1K_1K_2\con{FeP}\con{CPBA}+k_2K_i\con{CPBA}}{1+K_1K_2\con{FeP}\con{CPBA}+K_i\con{CPBA}}[\beta]\]
            于是
            \[k_{\text{obs}}=\dfrac{\left(k_1K_1K_2\con{FeP}+k_2K_i\right)\con{CPBA}}{1+\left(K_1K_2\con{FeP}+K_i\right)\con{CPBA}}\]
            如果你认为一个$\beta$可以变成一个\ce{VA},从而认为$-\dfrac{\di[\beta]}{\di t}=\dfrac{\con{VA}}{\di t}$,那就大错特错了.这道题最重要的地方就在于%
            中间体的浓度也会随时间而变化,并且其浓度都与$[\beta]$成正比关系,因此需要统一代入.
        \item 这一小问的做法与\tbf{\arabic{Pcounter}-1}是完全一致的,我们代入数据即可得
            \[\ln\dfrac{1.350\times10^{-4}}{2.398\times10^{-4}}=\dfrac{E_{a,2}}{8.314\JmK}\left(\dfrac{1}{293.2\K}-\dfrac{1}{301.2\K}\right)\]
            解得
            \[E_{a,2}=52.73\kJm\]
        \item 这一小问要求我们从两个角度分析\ce{FeP}对$\beta$氧化为\ce{VA}的反应是否有利.比较两个不同的反应机理,%
            既可以从具体决速步的活化能判断,也可以从有无催化剂时反应的表观活化能判断.
            \begin{enumerate}[label=\tbf{\arabic{Pcounter}-4-\arabic*},topsep=0pt,parsep=0pt,itemsep=0pt,partopsep=0pt,leftmargin=10pt]
                \item 从反应机理来看,机理$1$的步骤与机理$2$的后两步完全相同,这意味着这两个反应的速率常数和活化能相同(我们的计算确实支持这一点).%
                    但对于机理$2$有$E_{a,1}<E_{a,2}$,这意味着反应可以经过一个活化能更小的步骤进行,因此\ce{FeP}对这一反应是有利的.
                \item 从表观活化能来看有$E_{a,\text{obs}}<E_{a,\text{obs}}'$,这意味着通过机理$2$进行的反应的总体的活化能更小,这也可以得出\ce{FeP}对这一反应是有利的.
            \end{enumerate}
    \end{enumerate}
\end{solution}
\stepcounter{Pcounter}
\begin{problem}[P.7.\arabic{Pcounter} 岩石变化动力学]
    斜方辉石\ce{[(Mg,Fe)2Si2O6]}是地壳和上地幔的主要组成矿物之一,在其晶体结构中包含有两种不同的硅氧四面体\ce{(SiO3)2-}链,%
    分别称作\ce{A}链和\ce{B}链,基于这两种链的排布而形成了两种结构有差异的八面体空隙M1和M2,二者比例相同.%
    \ce{Mg^2+}和\ce{Fe^2+}便分布在这些八面体空隙中.由于M1和M2的空隙性质不同,%
    导致\ce{Mg^2+}和\ce{Fe^2+}对其占据的选择性不同,\ce{Fe^2+}倾向于占据M2位置.%
    一定条件下,两种离子可以发生不同位置的交换反应:
    \begin{tightcenter}
        \ce{Fe^2+({M1}) + Mg^2+({M2}) <=> Fe^2+({M2}) + Mg^2+({M1})}
    \end{tightcenter}
    为简便起见,\ce{Fe^2+({M1})},\ce{Mg^2+({M2})},\ce{Fe^2+({M2})}和\ce{Mg^2+({M1})}分别写作Fe(1),Fe(2),Mg(1)和Mg (2).%
    上述反应达平衡时,分配系数$K_D$为
    \[K_D=\dfrac{\chi_{\ce{Fe(2)}}\chi_{\ce{Mg(1)}}}{\chi_{\ce{Fe(1)}}\chi_{\ce{Mg(2)}}}\]
    其中,$\chi$为各离子占据相应位置的摩尔分数,例如:$\chi_{\ce{Fe(2)}}$为\ce{Fe^2+}离子占据M2位置的摩尔分数,其他同理.\\
    选择某一矿物样品,在$873\K$下进行处理,利用X射线衍射结合穆斯堡尔谱监测反应进行过程中上述物种占据不同位置情况随时间的变化,数据列入下表中.
    \vspace{-5pt}\begin{table}[H]\centering
        \begin{tabular}{|c|c|c|c|c|c|}
            \hline
                编号 & $t/\text{min}$ & $\chi_{\ce{Fe(1)}}$ & $\chi_{\ce{Mg(2)}}$ & $\chi_{\ce{Fe(2)}}$ & $\chi_{\ce{Mg(1)}}$ \\ \hline
                1   & 0     & 0.00450   & 0.9807    & 0.0174    & 0.9769 \\ \hline
                2	& 600   & 0.00420   & 0.9804    & 0.0176    & 0.9771 \\ \hline
                3	& 1920  & 0.00380   & 0.9801	& 0.0179	& 0.9774 \\ \hline
                4	& 3720	& 0.00361	& 0.9798	& 0.0183	& 0.9778 \\ \hline
                5	& 6000	& 0.00335	& 0.9795	& 0.0185	& 0.9780 \\ \hline
                6	& 11760	& 0.00281	& 0.9790	& 0.0191	& 0.9786 \\ \hline
                7	& 20300	& 0.00261	& 0.9788	& 0.0193	& 0.9788 \\ \hline
                8	& 29700	& 0.00233	& 0.9785	& 0.0195	& 0.9790 \\ \hline
                9	& 48165	& 0.00232	& 0.9785	& 0.0195	& 0.9790 \\ \hline
        \end{tabular}
    \end{table}\vspace{-15pt}
    \begin{enumerate}[label=\tbf{\arabic{Pcounter}-\arabic*},topsep=0pt,parsep=0pt,itemsep=0pt,partopsep=0pt]
        \item 计算分配系数$K_D$.(提示:合理判断并选择表中的数据.)
        \item 记上述反应的正,逆反应的速率常数分别为$k_1$和$k_{-1}$.假设正,逆反应速率表达形式均与基元反应类似,即反应速率分别与占据相应位置的各离子的摩尔分数$\chi$成正比.%
            据此,写出$K_D$与$k_1$和$k_{-1}$的关系式。
        \item 利用上表中起始阶段的数据(采用编号1到4的数据进行处理),计算$k_1$和$k_{-1}$的值.%
            (提示:可将\ce{Mg^2+}的摩尔分数视作常数,取0.9780.)
    \end{enumerate}
\end{problem}
\begin{solution}
    \begin{enumerate}[label=\tbf{\arabic{Pcounter}-\arabic*},topsep=0pt,parsep=0pt,itemsep=0pt,partopsep=0pt]
        \item 我们需要寻找平衡状态的数据以计算$K_D$.观察编号为8和9的数据,可以发现各物种的摩尔分数已经近似地不再变化,即达到平衡状态.%
            选取编号$9$的数据,可得
            \[K_D=\dfrac{\chi_{\ce{Fe(2)}}\chi_{\ce{Mg(1)}}}{\chi_{\ce{Fe(1)}}\chi_{\ce{Mg(2)}}}
            =\dfrac{0.9795\cdot0.0195}{0.9785\cdot0.00232}=8.41\]
            若选取编号$8$的数据得到$K_D=8.44$,亦可.
        \item 平衡时,正逆反应速率相等,即有
            \[k_1\chi_{\ce{Fe(2)}}\chi_{\ce{Mg(1)}}=k_{-1}\chi_{\ce{Fe(1)}}\chi_{\ce{Mg(2)}}\]
            又因为分配系数的定义为
            \[K_D=\dfrac{\chi_{\ce{Fe(2)}}\chi_{\ce{Mg(1)}}}{\chi_{\ce{Fe(1)}}\chi_{\ce{Mg(2)}}}\]
            结合上述两式可得
            \[K_D=\dfrac{k_1}{k_{-1}}\]
            事实上,这和一般的基元反应的平衡常数与正逆反应速率常数的关系是完全一致的.%
            在本题中,我们只不过不使用浓度,转而使用摩尔分数来表示反应速率.
        \item 为了求出摩尔分数随时间的关系,我们需要求出反应体系的积分速率方程.\\
            首先,体系的微分速率方程为
            \[\dfrac{\di\chi_{\ce{Fe(2)}}}{\di t}=-k_1\chi_{\ce{Fe(2)}}\chi_{\ce{Mg(1)}}+k_{-1}\chi_{\ce{Fe(1)}}\chi_{\ce{Mg(2)}}\]
            这一方程具有四个变量,难以进行进一步的积分运算,因此我们需要尽量减少和统一变量.%
            为此,首先将\ce{Mg^2+}的摩尔分数视作常数(不妨记作$\chi_{\ce{Mg}}$)后就有
            \[\dfrac{\di\chi_{\ce{Fe(2)}}}{\di t}=-k_1\chi_{\ce{Fe(2)}}\chi_{\ce{Mg}}+k_{-1}\chi_{\ce{Fe(1)}}\chi_{\ce{Mg}}
            =0.9780\left(-k_1\chi_{\ce{Fe(2)}}+k_{-1}\chi_{\ce{Fe(1)}}\right)\]
            随后,在\tbf{\arabic{Pcounter}-2}中我们知道$k_1$与$k_{-1}$的等量关系.将\tbf{\arabic{Pcounter}-1}中的$K_D$代入可得
            \[k_1=8.41 k_{-1}\]
            将此式代入速率方程可得
            \[\dfrac{\di\chi_{\ce{Fe(2)}}}{\di t}=0.9780k_{-1}\left(\chi_{\ce{Fe(1)}}-8.41\chi_{\ce{Fe(2)}}\right)\]
            现在我们来思考$\chi_{\ce{Fe(1)}}$和$\chi_{\ce{Fe(2)}}$的关系.体系中的\ce{Fe^2+}的总量是守恒的,即有
            \[n(\ce{Fe(1)})+n(\ce{Fe(2)})=n_0(\ce{Fe^2+})\]
            我们将两种空隙的数目,以及\ce{Fe^2+}占两种空隙的比例代入可得
            \[n(\ce{{M1}})\chi_{\ce{Fe(1)}}+n(\ce{{M2}})\chi_{\ce{Fe(2)}}=n_0(\ce{Fe^2+})\]
            由于题目中给出M1与M2的比例相同,因此有$n(\ce{{M1}})=n(\ce{{M2}})$.于是有
            \[\chi_{\ce{Fe(1)}}+\chi_{\ce{Fe(2)}}=\dfrac{n_0(\ce{Fe^2+})}{n(\ce{{M1}})}=2\chi_{\ce{Fe}}\]
            我们将$t=0$时的摩尔分数代入上式可得
            \[\chi_{\ce{Fe(1)}}+\chi_{\ce{Fe(2)}}=0.0219\]
            将这一等量关系代入速率方程以消去$\chi_{\ce{Fe(1)}}$,即可得到
            \[\dfrac{\di\chi_{\ce{Fe(2)}}}{\di t}=k_{-1}\left(0.1801-9.20\chi_{\ce{Fe(2)}}\right)\]
            现在我们终于得到了一个单变量的微分方程.对其移项积分可得
            \[\ln\left(0.1801-9.20\chi_{\ce{Fe(2)}}\right)=-9.20k_{-1}t+C\]
            其中$C$为积分常数.右边是关于$t$的一次函数,因此我们可以采取线性回归的方式求出其斜率,进而得出$k_{-1}$.%
            令$y=\ln\left(0.1801-9.20\chi_{\ce{Fe(2)}}\right)$,$x=t$,我们将题目中的数据代入并线性回归可得
            \[y=-1.42\times10^{-4}x-3.91\]
            于是
            \[k_{-1}=\dfrac{-1.42\times10^{-4}}{-9.20}\text{ min}^{-1}=1.54\times10^{-5}\text{ min}^{-1}\]
            \[k_1=8.41\times1.54\times10^{-5}\text{ min}^{-1}=1.30\times10^{-4}\text{ min}^{-1}\]
            如果采取$\chi_{\ce{Fe(1)}}$作为变量,得出的结果可能略有不同,但基本与此差距不大.\\
            这一道题目主要考察消去变量的思想.对于一个有多个变量的微分方程,我们既可以根据实际情况将某些变量近似地看作常量,%
            也可以根据化学反应内在的守恒关系减少变量的数目.在这道题中,这两种方法都需要使用.%
            最终,我们得出了一个简单的常微分方程,采取线性回归即可得出最终的结论.
    \end{enumerate}
\end{solution}
\end{document}